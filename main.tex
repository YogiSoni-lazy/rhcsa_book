\documentclass[14pt,fleqn]{extbook} % Default font size and left-justified equations
\usepackage{graphicx}
\usepackage{setspace}
\input{structure.tex} % Insert the commands.tex file which contains the majority of the structure behind the template
\usepackage{color}
\definecolor{light}{rgb}{0.5, 0.5, 0.5}
\def\light#1{{\color{light}#1}}
\usepackage{multicol}
\usepackage{parskip}
\usepackage{fancyhdr}
\usepackage{tabulary}
\usepackage{adjustbox}
\usepackage{subfig}
\usepackage{xcolor}
\usepackage{tcolorbox}
\usepackage{amssymb}
\usepackage{afterpage}
\tcbuselibrary{breakable}
\usepackage[printwatermark]{xwatermark}
%\usepackage[dvipsnames]{xcolor}

%\usepackage[framemethod=tikz]{mdframed}

\newcommand\blankpage{%
	\null
	\thispagestyle{empty}%
	\addtocounter{page}{-1}%
	\newpage}

\fancyfoot[R]
{
	\includegraphics[scale=0.6]{content/logo.png}%
}

%\fancyfoot[L]{Left footer}\fancyfoot[C]{Centre footer}\fancyfoot[R]{Right footer}


\thispagestyle{plain}
%\hypersetup{pdftitle={Title},pdfauthor={Author}} % Uncomment and fill out to include PDF metadata for the author and title of the book

%----------------------------------------------------------------------------------------
\setstretch{1.25}
%\definecolor[new][h=9A957A, a=1, t=.3]
\definecolor{mycolor}{RGB}{255, 174, 0}

%\newwatermark*[allpages,color=new!50,angle=45,scale=3,xpos=-20,ypos=0]{DRAFT}

%\usepackage{draftwatermark}
%\SetWatermarkLightness{ 0.9 }
%\SetWatermarkText{Lavatech\newline Technology}
%\SetWatermarkScale{0.3}

%%%%%%%%%%%%%%%%%%%%%%%   This block adds watermark  %%%%%%%%%%%%%%%%%
\newsavebox\mybox
\savebox\mybox{\tikz[color=mycolor,opacity=0.2]\node{lavatechtechnology.com};}
\newwatermark*[
allpages,
angle=45,
scale=2.5,
xpos=-20,
ypos=15
]{\usebox\mybox}
%%%%%%%%%%%%%%%%%%%%%%%   This block adds watermark  %%%%%%%%%%%%%%%%%


%\newsavebox\mybox
%\savebox\mybox{\tikz[color=red,opacity=0.3]\node{Lavatech};}
%\newwatermark*[
%allpages,
%angle=45,
%scale=6,
%xpos=-20,
%ypos=15
%]{\usebox\mybox}


\begin{document}

%----------------------------------------------------------------------------------------
%	TITLE PAGE
%----------------------------------------------------------------------------------------

\begingroup
\thispagestyle{empty} % Suppress headers and footers on the title page
\begin{tikzpicture}[remember picture,overlay]
\node[inner sep=0pt] (background) at (current page.center) {\includegraphics[width=\paperwidth, height=\paperheight]{cover_new.pdf}};
\end{tikzpicture}
\vfill
\endgroup

%----------------------------------------------------------------------------------------
%	COPYRIGHT PAGE
%----------------------------------------------------------------------------------------

\newpage

~\vfill
\thispagestyle{empty}

\noindent Copyright \copyright\ 2022 Lavatech Technology\\ % Copyright notice

The contents of this course and all its modules and related materials, including handouts are
Copyright ©

No part of this publication may be stored in a retrieval system, transmitted or reproduced in any way, including, but not limited to, photocopy, photograph, magnetic, electronic or other record, without the prior written permission of Lavatech Technology.

If you believe Lavatech Technology training materials are being used, copied, or otherwise improperly distributed please e-mail: 
\newline
\textbf{info@lavatechtechnology.com}

\noindent \textsc{Published by Lavatech Technology}\\ % Publisher

\noindent \textit{lavatechtechnology.com}\\ % URL

%\noindent Licensed under the Creative Commons Attribution-NonCommercial 3.0 Unported License (the ``License''). You may not use this file except in compliance with the License. You may obtain a copy of the License at \url{http://creativecommons.org/licenses/by-nc/3.0}. Unless required by applicable law or agreed to in writing, software distributed under the License is distributed on an \textsc{``as is'' basis, without warranties or conditions of any kind}, either express or implied. See the License for the specific language governing permissions and limitations under the License.\\  License information, replace this with your own license (if any)
%
\noindent \textit{January 2022} % Printing/edition date

\afterpage{\blankpage}
\afterpage{\blankpage}


%----------------------------------------------------------------------------------------
%	TABLE OF CONTENTS
%----------------------------------------------------------------------------------------


\usechapterimagetrue % If you don't want to include a chapter image, use this to toggle images off - it can be enabled later with \usechapterimagetrue

\chapterimage{image1.png} % Table of contents heading image

\pagestyle{empty} % Disable headers and footers for the following pages


\tableofcontents


\cleardoublepage % Forces the first chapter to start on an odd page so it's on the right side of the book

\pagestyle{fancy} % Enable headers and footers again

%----------------------------------------------------------------------------------------
%	PART One
%----------------------------------------------------------------------------------------

%\part{System Admin Level I}

%----------------------------------------------------------------------------------------
%	CHAPTER 1
%----------------------------------------------------------------------------------------

\afterpage{\blankpage}
\begin{flushleft}
	\bigskip
	\begin{figure}[h!]
		\centering
		\includegraphics[scale=.38]{content/chapter1/images/career.png}
	\end{figure}
\end{flushleft}

\newpage


\chapterimage{index2.png} % Chapter heading image
\chapter{Introduction}
%-----------------------
\section{Introduction to Operating System}\index{Introduction to Operating System}
\input{content/chapter1/1.1.tex}
\subsection{What is an Operating System (OS)?}\index{Introduction to Operating System!What is an Operating System (OS)?}

\begin{flushleft}
	An Operating System (OS) is an interface between a user and computer hardware.
	\bigskip
	\begin{figure}[h!]
		\centering
		\includegraphics[scale=.3]{content/chapter1/images/os.jpg}
		\caption{Operating System}
		\label{fig:OS}
	\end{figure}
\end{flushleft}

\newpage


\subsection{Architecture of OS}\index{Introduction to Operating System!Architecture of OS}
\begin{flushleft}
	An OS consists of:
	\begin{itemize}
		\item Kernel
		\item Shell
		\item Applications
	\end{itemize}
	\begin{figure}[h!]
		\centering
		\includegraphics[scale=.08]{content/chapter1/images/os_structure.jpg}
		\caption{OS architecture}
		\label{fig:OS_Structure}
	\end{figure}
	Let's see each of these in detail.
	
	\newpage
	
	\begin{enumerate} 
		\item \textbf{Kernel} - 
		\begin{itemize}
			\item Establish communication between hardware and software.
			\begin{figure}[h!]
				\centering
				\includegraphics[scale=.4]{content/chapter1/images/kernel.png}
				\caption{Kernel}
				\label{fig:kernel}
			\end{figure}
			\item Kernel performs:
			\begin{itemize}
				\item \textbf{Process Management}: Manage CPU processes.
				\item \textbf{Device Management}: Interface between hardware and process.
				\item \textbf{Memory Management}: Manage RAM.
				\item \textbf{Handling system calls}: Provides the interface between the processes and OS.
			\end{itemize}			
		\end{itemize}
		\bigskip
		
		\item \textbf{Shell} - An interface to kernel that takes commands from user and executes them.
		\bigskip
		
		\item \textbf{Utilities/applications} - Programs that can be directly used by users.
	\end{enumerate}
\end{flushleft}

\newpage




\subsection{Types of OS}\index{Introduction to Operating System!Types of OS}
\setlength{\columnsep}{20pt}
\begin{flushleft}
	There are 3 types of OS:
	\begin{enumerate}
		\item 
		\begin{multicols}{2}
			\textbf{Server OS} - Designed for server computers that runs 24X7.
			\newline
			Eg:
			\begin{itemize}
				\item Linux server
				\item Windows server
				\item Mac OS X server
			\end{itemize}
			\vfill \null
			\columnbreak
			\includegraphics[scale=.03]{content/chapter1/images/server.jpg}
		\end{multicols}
		\vspace{-25pt}

\bigskip
		
		\item 
		\begin{multicols}{2}
			\textbf{Desktop OS} - Designed for personal computer.
			\newline
			Eg:
			\begin{itemize}
				\item Windows
				\item Mac OS
			\end{itemize}
			\vfill \null			
			\columnbreak
			\includegraphics[scale=.03]{content/chapter1/images/desktop.jpg}
		\end{multicols}
		\vspace{-25pt}
		
\bigskip
		
		\item 
		\begin{multicols}{2}
			\textbf{Mobile OS} - Designed to run on mobile devices.
			\newline
			Eg:
			\begin{itemize}
				\item IPhone OS
				\item Windows mobile
				\item Anroid
			\end{itemize}
			\vfill \null			
			\columnbreak
			\includegraphics[scale=.12]{content/chapter1/images/mobile.jpg}
		\end{multicols}
		\vspace{-25pt}

		%\bigskip
	\end{enumerate}
\end{flushleft}

\newpage


\subsection{Practice}\index{Introduction to Operating System!Practice}
\begin{flushleft}
	
	\bigskip
	
	\begin{figure}[h!]
		\centering
		\includegraphics[scale=.2]{content/practise.jpg}
	\end{figure}	
	\begin{enumerate}
		\item \textbf{What is an Operating System? (Select all that applies.)}
		\begin{enumerate}[label=(\alph*)]
			\item A software used for only gaming purpose.
			\item Collection of software that acts as an interface between a user and computer hardware. %correct
			\item OS consists of kernel to interact with hardware, shell and utilities. %correct
			\item OS is a software used for accounting purpose. %correct
			%\item \light{Handling system calls}
		\end{enumerate}
		\bigskip
		\bigskip
		\item \textbf{Which of the following are the functions of a kernel? (Select all that applies.)}
		\begin{enumerate}[label=(\alph*)]
			\item Perform memory management by keeping track of memory. %correct
			\item Perform hardware management. %correct
			\item Establish communication between hardware and software. %correct
			\item Control all processes of the OS. %correct
		\end{enumerate}
		\bigskip
		\bigskip
		\item \textbf{State whether true or false. Kernel is an interface between hardware and software.}
		\begin{enumerate}[label=(\alph*)]
			\item True %correct
			\item False
		\end{enumerate}
	\end{enumerate}
	

\end{flushleft}

\newpage

\section{Introduction to Linux}\index{Introduction to Linux}
\setlength{\columnsep}{3pt}
\begin{flushleft}
	\bigskip
	\bigskip
	\begin{tcolorbox}[breakable,notitle,boxrule=1pt,colback=black,colframe=black]
		\color{white}
		\bigskip
		In this section, you are going to learn:
		\begin{enumerate}
			\item \textbf{What is Open Source Software (OSS)?}
			\item \textbf{What is Linux OS?}
			\item \textbf{Linux OS V/S Windows OS}
			\item \textbf{What is console, terminal, command line \& shell?}
			\item \textbf{Linux distributions}
		\end{enumerate}	
		Finally, there will be a \textbf{small excerise} on these topics to check your knowledge.
		\bigskip
	\end{tcolorbox}
	\bigskip
	\bigskip
	\begin{multicols}{2}
		\vspace*{\fill}
		\vspace*{\fill}
		\vspace*{\fill}
		\vspace*{\fill}
		\vspace*{\fill}
		\vspace*{\fill}
		\vfill \null
		\columnbreak
		So let's get started....
		\includegraphics[scale=0.07]{content/linux_section.png}
	\end{multicols}
\end{flushleft}
\newpage

\subsection{What is Open Source Software (OSS)?}\index{Introduction to Linux!What is open source software(OSS)?}
\setlength{\columnsep}{5pt}
\begin{flushleft}
	\paragraph{}
	\begin{itemize}
		\item Open source software (OSS) is code that is publicly accessible.
		\item OSS is:
		\begin{itemize}
			\item \textbf{Free to use}
			\item \textbf{Free to modify}
			\item \textbf{Free to distribute}
		\end{itemize}
	\end{itemize}	
	\bigskip
	\bigskip
	\bigskip
	\begin{figure}[h!]
		\centering
		\includegraphics[scale=.1]{content/chapter1/images/open.png}
		\caption{Open Source Software Logo}
		\label{fig:opensource}
	\end{figure}
	Some of the OSS:
	\begin{figure}[h!]
		\centering
		\includegraphics[scale=.4]{content/chapter1/images/opensource.png}
		\caption{Open Source Softwares}
		\label{fig:opensource1}
	\end{figure}
	
\end{flushleft}
\newpage

\subsection{What is Linux OS?}\index{Introduction to Linux!Linux V/S Windows}
\setlength{\columnsep}{5pt}
\begin{flushleft}
	\paragraph{}
	\begin{itemize}
		\item Linux® is an open source operating system. 
		\item Created as a hobby by Linus Torvalds in 1991, as alternative of the MINIX OS (which is variant of Unix).
		\item Developed for \textbf{ personal computers, servers, mainframes, mobile devices and embedded devices}.
	\end{itemize}	
	\begin{figure}[h!]
		\centering
		\includegraphics[scale=.5]{content/chapter1/images/tux.png}
		\caption{Tux, the Linux mascot, by Larry Ewing}
		\label{fig:mascot}
	\end{figure}	
\end{flushleft}

\newpage

\subsection{Linux OS v/s Windows OS}\index{Introduction to Linux!Linux V/S Windows}
\setlength{\columnsep}{5pt}

\begin{flushleft}
	\paragraph{}
		%\begin{tabulary}{1.0\textwidth}{C|C|C|p{10em}}
		
		
		\begin{tabulary}{1.0\textwidth}{|p{14em}|p{14em}|}
			\toprule
			\textbf{Linux OS} & \textbf{Windows OS}\\
			\midrule
			Open source OS & Closed source OS \\
			\hline
			Free of cost & Not free \\
			\hline
			File names are case-sensitive & File names are case-insensitive \\
			\hline
			More efficient in resource usage & Less efficient in resource usage\\
			\hline
			More security, no need to install anti-virus & Less security, need to install anti-virus\\
			\hline
			Used for devops, programming, database, cloud-computing, big data hadoop etc & Used mostly for house hold purpose\\
			\bottomrule
		\end{tabulary}

		\label{tab:example} % Unique label used for referencing the table in-text
		%\addcontentsline{toc}{table}{Table \ref{tab:example}} % Uncomment to add the table to the table of contents
		

				
\end{flushleft}

\newpage
\subsection{Understanding console, terminal, command line and shell}\index{Introduction to Linux!Understanding console, terminal, command line and shell}
\setlength{\columnsep}{5pt}
\begin{flushleft}
	\bigskip
	\bigskip
	\paragraph{What is a console and terminal?}
	\begin{itemize}
		\item A console is a device with a screen and keyboard combined inside it.
		\item Terminal is the software program inside the console.
			\begin{figure}[h!]
			\centering
			\includegraphics[width=5cm]{content/chapter1/images/console.png}
			\caption{VT terminals - The first console with terminal}%
			\label{fig:example}%
			\end{figure}		
	\end{itemize}
% figure side by side
%	\begin{figure}[h!]
%		\centering
%		\subfloat[\centering VT terminals]{{
%		\includegraphics[width=10cm]{content/chapter1/images/console.png}}}%
%		\qquad
%		\subfloat[\centering RS-232 connector to connect console to terminal]{{\includegraphics[width=5cm]{content/chapter1/images/connector.jpeg} }}%
%		\caption{2 VT terminal and connector}%
%		\label{fig:example}%
%	\end{figure}

	\paragraph{What is a command line?}
	\begin{itemize}
		\item It is a blank line and cursor on the screen, allowing the user to type commands to execute.
	
		\begin{figure}[h!]
			\centering
			\includegraphics[scale=.18]{content/chapter1/images/commandline.png}
			\caption{Linux command line}
			\label{fig:commandline}
		\end{figure}	
	\end{itemize}
	\paragraph{What is a shell?}
	\begin{itemize}
		\item Shell is an interface to kernel.
		\item It executes Linux commands \& display it's result.
		\item Eg:
		\begin{itemize}
			\item Shell in Linux OS: bash, fish, zsh, ksh, sh, tsch
			\item Shell in Windows OS: PowerShell, pwsh
		\end{itemize}
	\end{itemize}		
\end{flushleft}

\newpage

%\subsection{Linux in server industry}\index{Introduction to Linux!Linux in server industry}
%\setlength{\columnsep}{5pt}
\begin{flushleft}
	\bigskip
	\bigskip
	\paragraph{What is a console and terminal?}
	\begin{itemize}
		\item A console is a device with a screen and keyboard combined inside it.
		\item Terminal is the software program inside the console.
			\begin{figure}[h!]
			\centering
			\includegraphics[width=5cm]{content/chapter1/images/console.png}
			\caption{VT terminals - The first console with terminal}%
			\label{fig:example}%
			\end{figure}		
	\end{itemize}
% figure side by side
%	\begin{figure}[h!]
%		\centering
%		\subfloat[\centering VT terminals]{{
%		\includegraphics[width=10cm]{content/chapter1/images/console.png}}}%
%		\qquad
%		\subfloat[\centering RS-232 connector to connect console to terminal]{{\includegraphics[width=5cm]{content/chapter1/images/connector.jpeg} }}%
%		\caption{2 VT terminal and connector}%
%		\label{fig:example}%
%	\end{figure}

	\paragraph{What is a command line?}
	\begin{itemize}
		\item It is a blank line and cursor on the screen, allowing the user to type commands to execute.
	
		\begin{figure}[h!]
			\centering
			\includegraphics[scale=.18]{content/chapter1/images/commandline.png}
			\caption{Linux command line}
			\label{fig:commandline}
		\end{figure}	
	\end{itemize}
	\paragraph{What is a shell?}
	\begin{itemize}
		\item Shell is an interface to kernel.
		\item It executes Linux commands \& display it's result.
		\item Eg:
		\begin{itemize}
			\item Shell in Linux OS: bash, fish, zsh, ksh, sh, tsch
			\item Shell in Windows OS: PowerShell, pwsh
		\end{itemize}
	\end{itemize}		
\end{flushleft}

\newpage

\subsection{Linux distributions}\index{Introduction to Linux!Linux distributions}
\setlength{\columnsep}{5pt}
\begin{flushleft}
	\paragraph{}
	\begin{itemize}
		\item A Linux distribution (or distro) is made from the \textbf{Linux kernel and collection of software}.
		\item Almost one thousand Linux distributions exist.
		\item Free and community managed distributions are:
		\begin{figure}[h!]
			\centering
			\includegraphics[scale=0.4]{content/chapter1/images/distro.png}
			\caption{Linux distributions}
			\label{fig:distro1}
		\end{figure}
		
		\item Popular commercially backed distributions are:
		\begin{figure}[h!]
			\centering
			\includegraphics[scale=0.3]{content/chapter1/images/distro2.png}
			\caption{Commercial Linux distributions}
			\label{fig:distro2}
		\end{figure}

	\end{itemize}

	\newpage
	\paragraph{What is upstream and downstream?}
	\begin{itemize}
		\item The term 'upstream' refers to the \textbf{original version of a software}.
		\item Downstream is the \textbf{refined product code} based on original software version.
		\item Eg:
		\begin{itemize}
			\item \textbf{Fedora} is the upstream to \textbf{Red Hat Enterprise Linux (RHEL)}.
			\item \textbf{Debian} is the upstream to \textbf{Ubuntu}.
		\end{itemize}
	\end{itemize}
\end{flushleft}
\newpage

\subsection{Practice}\index{Introduction to Linux!Practice}
\setlength{\columnsep}{3pt}
\begin{flushleft}
	\paragraph{}

	\bigskip

	\begin{figure}[h!]
		\centering
		\includegraphics[scale=.2]{content/practise.jpg}
	\end{figure}	
	
	\begin{enumerate}
		\item \textbf{Which of the following are Open Source Software (OSS)? (Select all that applies.)}
		\begin{enumerate}[label=(\alph*)]
			\item Python %correct
			\item Linux % correct
			\item Apache webserver %correct
			\item MySQL %correct
		\end{enumerate}
		\bigskip
		\bigskip
		\item \textbf{State whether true or false. OSS is free to use, develop, modify and distribute for personal and professional use.}
		\begin{enumerate}[label=(\alph*)]
			\item True  %correct
			\item False
		\end{enumerate}
		\bigskip
		\bigskip
		\item \textbf{Select all statement true for Linux OS.}
		\begin{enumerate}[label=(\alph*)] 
			\item Linux is OSS developed by Linus Torvalds. %correct
			\item Linux OS cannot be used for house-hold use.
			\item Linux OS is variant of MINIX OS.    %correct
			\item Linux is desgined for servers and mainframes computer.   %correct
		\end{enumerate}
		\bigskip
		\bigskip
		\newpage
		\item \textbf{Select all statement that are true about Linux and Windows OS.}
		\begin{enumerate}[label=(\alph*)]
			\item File names are case sensitive in Linux OS and case-insensitive in Windows OS.  %correct
			\item Linux OS is less secure than Windows OS.   
			\item Linux OS is OSS while Windows OS is closed source OS.   %correct
			\item Windows OS is expensive compared to Linux OS.   %correct
		\end{enumerate}
		\bigskip
		\bigskip
		\item \textbf{Select all statement that are true about shell.}
		\begin{enumerate}[label=(\alph*)]
			\item Bash and ksh are types of shell used in Linux OS.  %correct
			\item Shell takes command from terminal and supplies it to kernel for processing.  %correct
			\item Shell is an interface to kernel. %correct
			\item Powershell is an example of shell used in Windows OS.  %correct
		\end{enumerate}
		\bigskip
		\bigskip
		\item \textbf{State whether true or false. Fedora is upstream of RHEL.}
		\begin{enumerate}[label=(\alph*)]
			\item True   %correct
			\item False   
		\end{enumerate}
	\end{enumerate}

	
\end{flushleft}
\newpage



%--------------	--------


%--------------------------------------------------------------------------
%	CHAPTER 2
%-------------------------------------------------------------------------

\chapterimage{index3.png} % Table of contents heading image
\chapter{Linux Basics}
%-----------------------
\section{Linux directory structure}\index{Linux directory structure}
\input{content/chapter2/2.1.tex}
\subsection{Filesystem Hierarchy Standard (FHS)}\index{Linux directory structure!Filesystem Hierarchy Standard (FHS)}

\begin{flushleft}
	\begin{itemize}
		\item If you’re coming from Windows, you must be aware of C: drive, D: drive etc.
		\item In Linux, there is \textbf{no} C: or D: drive. 
		\item Linux have standard directory structure called Filesystem Hierarchy Standard (FHS).
	\end{itemize}
	
	\begin{tcolorbox}[breakable,notitle,boxrule=-1pt,colback=yellow,colframe=yellow]
		\color{black}
		\bigskip
		Note: Folder and directory means the same!
		\bigskip
	\end{tcolorbox}
	
	\bigskip
	\begin{figure}[h!]
		\centering
		\includegraphics[scale=.24]{content/chapter2/images/fhs.jpg}
		\caption{Filesystem Hierarchy Standard (FHS)}
		\label{fig:fhs}
	\end{figure}
\end{flushleft}

\newpage


\subsection{Important directories in FHS}\index{Linux directory structure!Important Directory You Should Know About}

\begin{flushleft}
	\begin{enumerate}
		\item \textbf{/} 
		\begin{itemize}
			\item "/" is \textbf{top most directory} or \textbf{root directory} or \textbf{starting point} of FHS. 
		\end{itemize}
		\item \textbf{/home}
		\begin{itemize}
			\item Contains \textbf{home folder for each user} having their personal data \& configuration files.
			\item Eg: Home directory for user \textbf{jack} is \textbf{/home/jack}.
			\begin{figure}[h!]
				\centering
				\includegraphics[scale=.3]{content/chapter2/images/path.png}
				\caption{Path separator}
				\label{fig:path}
			\end{figure}
			\begin{tcolorbox}[breakable,notitle,boxrule=-0pt,colback=yellow,colframe=yellow]
				\color{black}
				\textbf{Note:}
				\begin{itemize}
					\item Symbol \textbf{"$\sim$"} refers to the user's home directory.
					\item Eg: If you are logged in as user \textbf{"jack"}, then \textbf{"$\sim$"} meant \textbf{"/home/jack"}.
					\item Eg: If you are logged in as user \textbf{"jill"}, then \textbf{$\sim$} meant \textbf{"/home/jill"}.
					\end{itemize}
			\end{tcolorbox}
			
		\end{itemize}
		\item \textbf{/bin}
		\begin{itemize}
			\item Contains user binaries (programs).
			\item Eg: cp, mkdir, pwd, ls, rmdir etc.
		\end{itemize}
		\item \textbf{/sbin}
		\begin{itemize}
			\item Contains system administration binaries.
			\item Eg: fdisk, useradd, userdel, ifconfig etc.
		\end{itemize}
	\newpage
		\item \textbf{/usr}
		\begin{itemize}
			\item Contains read-only commands, libraries and data.
			\begin{itemize}
				\item \textbf{/bin} is link to \textbf{/usr/bin}
				\item \textbf{/sbin} is link to \textbf{/usr/sbin}
				\item \textbf{/lib} is link to \textbf{/usr/lib}
			\end{itemize}
		\end{itemize}
		\item \textbf{/etc}
		\begin{itemize}
			\item Contains system-wide configuration files.
			\item Eg: 
			\begin{itemize}
				\item /etc/fstab
				\item /etc/sysconfig/network-script/
			\end{itemize}
		\end{itemize}
			\item \textbf{/opt}
			\begin{itemize}
				\item Opt stands for optional.
				\item Used by proprietary 3rd party software to store their settings.
			\end{itemize}
		\item \textbf{/tmp}
		\begin{itemize}
			\item Contains temporary files of system.
			\item Files under this are deleted when system is rebooted.
		\end{itemize}
		\item \textbf{/var}
		\begin{itemize}
			\item Var stands for variable files.
			\item /var includes:
			\begin{itemize}
				\item System log files: \textbf{/var/log}
				\item Packages and database files: \textbf{/var/lib}
				\item Emails: \textbf{/var/mail}
				\item Lock files: \textbf{/var/lock}
				\item Temporary files needed across reboots: \textbf{/var/tmp}
			\end{itemize}
		\end{itemize}
			\item \textbf{/mnt}
		\begin{itemize}
			\item Temporary mount directory.
		\end{itemize}
		\item \textbf{/boot}
		\begin{itemize}
			\item Contains grub2 bootloader, kernel, initramfs etc. needed during system boot up.
		\end{itemize}
		\item \textbf{/media}
		\begin{itemize}
			\item Mounts temporary media device like USB drive.
		\end{itemize}
	\newpage
		\item \textbf{/proc}
		\begin{itemize}
			\item Contains running process informations.
			\item Eg: /proc/\{pid\} directory contains information about the process with that particular process.
		\end{itemize}
		\item \textbf{/run}
		\begin{itemize}
			\item Used for run-time variable data.
			\item Eg: Socket files, Process IDs etc.
			\item Difference between \textbf{/run} \& \textbf{/tmp}:
			\begin{itemize}
				\item Data in \textbf{/run always get deleted at next boot}
				\item Data in /tmp may or may not get deleted at next boot.
			\end{itemize}
		\end{itemize}
	\end{enumerate}
	
\end{flushleft}

\newpage

\subsection{General Linux Shortcuts}
\setlength{\columnsep}{5pt}

\begin{flushleft}
	\paragraph{}
	%\begin{tabulary}{1.0\textwidth}{C|C|C|p{10em}}
	
	\textbf{Useful command line-editing shortcuts}	
	\begin{tabulary}{1.0\textwidth}{|p{10em}|p{18em}|}
		\toprule
		\textbf{Shortcut} & \textbf{Description}\\
		\midrule
		\textbf{Ctrl + l} & Clear the terminal screen. \\
		\hline
		\textbf{Ctrl + a} & Jump to the beginning of the command line. \\
		\hline
		\textbf{Ctrl + e} & Jump to the beginning of the command line. \\
		\hline
		\textbf{Ctrl + u} & Clear from the cursor to the beginning of the command line. \\
		\hline
		\textbf{Ctrl + k} & Clear from the cursor to the end of the command line. \\
		\hline
		\textbf{Ctrl + r} & Search the history list of commands for a pattern. \\
		\hline
		\textbf{Ctrl + Shift + c} & Copy text from terminal. \\
		\hline
		\textbf{Ctrl + Shift + v} & Paste copied text on terminal. \\
		\hline
		\textbf{Ctrl + Shift + t} & Open new terminal. \\
		\hline
		\textbf{Alt + Tab} & Switch between applications. \\
		\bottomrule
	\end{tabulary}

	
	\label{tab:example} % Unique label used for referencing the table in-text
	%\addcontentsline{toc}{table}{Table \ref{tab:example}} % Uncomment to add the table to the table of contents
	
	
	
\end{flushleft}

\newpage
\subsection{Practice}\index{Linux directory structure!Practice}
\setlength{\columnsep}{3pt}
\begin{flushleft}
	\paragraph{}

	\bigskip

	\begin{figure}[h!]
		\centering
		\includegraphics[scale=.2]{content/practise.jpg}
	\end{figure}	
	
	\begin{enumerate}
		\item \textbf{Does Linux OS have same directory structure as Windows OS?}
		\begin{enumerate}[label=(\alph*)]
			\item Yes
			\item No    %correct
		\end{enumerate}
		\bigskip
		\bigskip
		\item \textbf{Which of the following is the top most directory in Linux OS?}
		\begin{enumerate}[label=(\alph*)]
			\item /         %correct
			\item /home
			\item /root
			\item /boot
		\end{enumerate}
		\bigskip
		\bigskip
		\item \textbf{Which of the following directories contains user binaries \& system administration binaries? }
		\begin{enumerate}[label=(\alph*)]
			\item /bin             %correct
			\item /usr/bin %correct
			\item /usr/sbin %correct
			\item /sbin %correct
		\end{enumerate}
		\bigskip
		\bigskip
		\item \textbf{State whether true or false. All files and directories under /tmp may or may not get deleted on system reboot.}
		\begin{enumerate}[label=(\alph*)]
			\item True          %correct
			\item False
		\end{enumerate}
		\bigskip
		\bigskip
		\item \textbf{Which of the following files/folders are present under /var directory?}
		\begin{enumerate}[label=(\alph*)]
			\item /var/tmp   %correct
			\item /var/log   %correct
			\item /var/mail  %correct
			\item /var/spool  %correct
		\end{enumerate}
		\bigskip
		\bigskip
		\item \textbf{Which of the following directory is used for auto-mounting CD-ROM or USB drive?}
		\begin{enumerate}[label=(\alph*)]
			\item /mnt
			\item /media  %correct
			\item /tmp
			\item /run
		\end{enumerate}
		\bigskip
		\bigskip
		\item \textbf{Which of the following directory stores information about optional third party application like VLC media player.}
		\begin{enumerate}[label=(\alph*)]
			\item /mnt
			\item /media
			\item /opt     %correct
			\item /run
		\end{enumerate}
		\bigskip
		\bigskip
		\item \textbf{The system process in running state stores it's process id and other related information in which of the following directory?}
		\begin{enumerate}[label=(\alph*)]
			\item /mnt
			\item /proc  %correct
			\item /tmp
			\item /run
		\end{enumerate}
			\bigskip
			\bigskip
		\item \textbf{Where is the kernel of Linux OS stored?}
		\begin{enumerate}[label=(\alph*)]
			\item /bin
			\item /usr
			\item /proc
			\item /boot    %correct
		\end{enumerate}  
	\end{enumerate}

	
\end{flushleft}
\newpage



\section{Linux commands}\index{Linux commands}
\input{content/chapter2/2.2.tex}
\subsection{Basic commands}\index{Linux commands!Basic commands}


\begin{flushleft}
	Before starting with basic commands, let's first understand a few things.
	\paragraph{What is command prompt?}
	\begin{itemize}
		\item Command prompt is also known as shell prompt.
		\item Commands are entered in a terminal at the shell prompt. 
		\begin{figure}[h!]
			\centering
			\includegraphics[scale=.5]{content/chapter2/images/command_prompt1.jpg}
			\caption{Command Prompt}
			\label{fig:command_prompt1}
		\end{figure}
		\begin{figure}[h!]
			\centering
			\includegraphics[scale=.5]{content/chapter2/images/command_prompt2.jpg}
			\caption{Command Prompt}
			\label{fig:command_prompt2}
		\end{figure}
		\item The shell prompt lists:
		\begin{itemize}
			\item Username who logged in
			\item Server hostname
			\item Current directory 
			\item \textbf{"\$"} prompt (if you are normal user) or \textbf{"\#"} prompt (if you are root user)
		\end{itemize}		
	

	\end{itemize}
		
	\newpage

	\paragraph{Command syntax}
	
	\begin{itemize}
		\item Commands entered at the shell prompt have three parts:
		\bigskip
			\begin{tcolorbox}[breakable,notitle,boxrule=-0pt,colback=pink,colframe=pink]
			\color{black}
			\fontdimen2\font=1em
			\bigskip
			Syntax:  command [options] [argument]
			\fontdimen2\font=4pt
			\bigskip
		\end{tcolorbox}
		Explaination:
		\begin{itemize}
			\item \textbf{command}: name of command
			\item \textbf{options}: start with one or two dashes (eg: -a or –all)
			\item \textbf{arguments}: a target that the command should operate on
			\item \textbf{"[]"} means optional.
		\end{itemize}

		\item
		Eg:
		\begin{figure}[h!]
			\centering
			\includegraphics[scale=.65]{content/chapter2/images/command_prompt3.jpg}
			\caption{Command syntax}
			\label{fig:command_prompt3}
		\end{figure}
	\end{itemize}

	\newpage
	\paragraph{Now let's get started with some actual commands.}
	
	\begin{enumerate}
		\item \textbf{pwd}: Shows the user’s present working directory.
		\newline
		Eg:
		\begin{tcolorbox}[breakable,notitle,boxrule=-0pt,colback=black,colframe=black]
			\color{green}
			\fontdimen2\font=1em
			\# pwd
			\fontdimen2\font=4pt
		\end{tcolorbox}
		\bigskip
		\bigskip

		\item \textbf{clear}: Clears the contents of the screen.
		\newline	
		Eg:
		\begin{tcolorbox}[breakable,notitle,boxrule=-0pt,colback=black,colframe=black]
			\color{green}
			\fontdimen2\font=1em
			\# clear
			\fontdimen2\font=4pt
		\end{tcolorbox}
		\bigskip
		\begin{tcolorbox}[breakable,notitle,boxrule=-0pt,colback=orange,colframe=orange]
			\color{black}
			\textbf{Tip:} You can use the keyboard shortcut \textbf{CTRL+L} as well to clear the screen.
		\end{tcolorbox}
	
		\bigskip
		\bigskip
		
		\item \textbf{mkdir}: Create a directory.
		\bigskip
		\begin{tcolorbox}[breakable,notitle,boxrule=1pt,colback=pink,colframe=pink]
			\color{black}
			\fontdimen2\font=1em
			Syntax:  mkdir [options] foldername
			\fontdimen2\font=4pt
		\end{tcolorbox}
		Eg:
		\bigskip
		\begin{tcolorbox}[breakable,notitle,boxrule=-0pt,colback=black,colframe=black]
			\color{green}
			\fontdimen2\font=1em
			\# mkdir /home/jack/test
			\fontdimen2\font=4pt
		\end{tcolorbox}
		Option of \textbf{mkdir} command:
		\begin{itemize}
			\item -p: Creates a directory and all its parents too.
			\newline
			Eg:
			\begin{tcolorbox}[breakable,notitle,boxrule=-0pt,colback=black,colframe=black]
				\color{green}
				\fontdimen2\font=1em
				\# mkdir -p /home/jack/test/a/b/c
				\fontdimen2\font=4pt
			\end{tcolorbox}
		\end{itemize}
		\bigskip

		\item \textbf{ls}: Lists the content of a directory.
		\bigskip
		\begin{tcolorbox}[breakable,notitle,boxrule=1pt,colback=pink,colframe=pink]
			\color{black}
			\fontdimen2\font=1em
			Syntax:  ls [options] [foldername]
			\fontdimen2\font=4pt
		\end{tcolorbox}
	
		Eg:
		\bigskip
		\begin{tcolorbox}[breakable,notitle,boxrule=-0pt,colback=black,colframe=black]
			\fontdimen2\font=1em
			\color{yellow}
            \# List current directory content
            \newline
            \color{green}
			\# ls        
			\newline
			\newline
			\color{yellow}
			\# List content of /home directory
			\newline
			\color{green}
			\# ls /home                
			\fontdimen2\font=4pt
		\end{tcolorbox}
	
		Options with \textbf{ls} command-
		\begin{itemize}		
			\item \textbf{-l}: List more details of contents inside directory.
			\newline
			Eg:
			\begin{tcolorbox}[breakable,notitle,boxrule=-0pt,colback=black,colframe=black]
				\color{green}
				\$ ls -l /home
				\color{white}
				\small
				\fontdimen2\font=1em
				\newline
				total 24
				\newline
				drwxr-xr-x    3    jack   jack    4096 Feb2 17:02 jack
				\newline
				drwx------  2 root   root   16384 Dec8 14:39 lost+found
			\end{tcolorbox}
			Output explaination:
			\begin{itemize}
			\item Column 1 – Type \& permissions of file
				\item Column 2 - Number of links
				\item Column 3 - Owner of the file
				\item Column 4 - Group under which the file belongs
				\item Column 5 - Size of file
				\item Column 6 - Date of last update
				\item Column 7 – Last updated time of file
				\item Column 8 - Name of file/directory
			\end{itemize}
			\item \textbf{-d}: Shows information about directory rather than listing.
			\newline
			Eg:
			\begin{tcolorbox}[breakable,notitle,boxrule=-0pt,colback=black,colframe=black]
				\color{green}
				\# ls -ld 
				\color{white}
				%\small
				\fontdimen2\font=1em
				\newline
				drwxr-xr-x 3 jack jack 4096 Feb  2 17:02 .
				\fontdimen2\font=4pt
			\end{tcolorbox}
		
			\item \textbf{-a}: Shows all files, including hidden files. 
			\bigskip
			\begin{tcolorbox}[breakable,notitle,boxrule=-0pt,colback=yellow,colframe=yellow]
				\color{black}
				Note: Names of hidden files/folders begin with a dot.
			\end{tcolorbox}
			
			Eg:
			\begin{tcolorbox}[breakable,notitle,boxrule=-0pt,colback=black,colframe=black]
				\color{green}
				\fontdimen2\font=1em
				\# ls -a
				\color{white}
				\newline
				.  ..  .bash\_logout  .bashrc  Desktop  .profile
				\fontdimen2\font=4pt
			\end{tcolorbox}
		\end{itemize}
		\bigskip
		\bigskip

		\newpage
		\item \textbf{cd}: Used to switch between directories.
		\bigskip
		\begin{tcolorbox}[breakable,notitle,boxrule=1pt,colback=pink,colframe=pink]
			\color{black}
			\fontdimen2\font=1em
			Syntax:  cd [foldername]
			\fontdimen2\font=4pt
		\end{tcolorbox}
		Eg:
		\begin{tcolorbox}[breakable,notitle,boxrule=-0pt,colback=black,colframe=black]
			\fontdimen2\font=1em
			\color{yellow}
			\# \textbf{cd} without any argument \textbf{switch to user's home directory}.
			\newline
			\color{green}
			\# cd
			\fontdimen2\font=4pt
		\end{tcolorbox}
		Special characters that can be used with "cd" command:
		\begin{itemize}
			\item "." means current directory
			\item ".." means parent directory
			\item "–" would take you to previous working directory
			\item {"$\sim$"} would take you to home directory of the user
		\end{itemize}
		Eg:
		\begin{tcolorbox}[breakable,notitle,boxrule=-0pt,colback=black,colframe=black]
			\color{green}
			\fontdimen2\font=1em
			\# cd /home/
			\newline
			\# cd ..
			\newline
			\# cd -
			\newline
			\# cd {$\sim$}
			\fontdimen2\font=4pt
		\end{tcolorbox}

		\bigskip
		\bigskip
		\textbf{Absolute Path \& Relative Path}: 
		\begin{itemize}
			\item \textbf{Absolute path}: Starts with root directory (i.e "/") \& is complete path. 
			\item \textbf{Relative path}: It is relative to current location \& does not start with root directory (i.e "/").
			\begin{figure}[h!]
				\centering
				\includegraphics[scale=.45]{content/chapter2/images/path2.png}
				\caption{Absolute \& relative path}
				\label{fig:path2}
			\end{figure}	
		\end{itemize}
	
		\newpage		
		\item \textbf{cat}: Used to display contents of a file and also to create one.
		\begin{enumerate}[label=(\alph*)]
			\item Creating a file:
			\newline
			Eg:
			\begin{tcolorbox}[breakable,notitle,boxrule=-0pt,colback=black,colframe=black]
				\color{green}
				\fontdimen2\font=1em
				\# cat > hello.txt
				\newline
				hi
				\newline
				welcome to unix 
				\newline
				\color{yellow}
				\# End by pressing Ctrl + d
			\fontdimen2\font=4pt
			\end{tcolorbox}
			\item Displaying a file:
			\newline
			Eg:
			\begin{tcolorbox}[breakable,notitle,boxrule=-0pt,colback=black,colframe=black]
				\color{green}
				\fontdimen2\font=1em
				\# cat hello.txt
				\fontdimen2\font=4pt
			\end{tcolorbox}
		\end{enumerate}


		\bigskip
		\bigskip
		\item \textbf{touch}: Performs 2 functions:
		\begin{enumerate}[label=(\alph*)]
		\item Update timestamp of file, if it exists.
		\item Creates a new file, if it does not exists.
		\newline
		Eg:
		\begin{tcolorbox}[breakable,notitle,boxrule=-0pt,colback=black,colframe=black]
			\color{green}
			\fontdimen2\font=1em
			\# touch abc.txt
			\fontdimen2\font=4pt
		\end{tcolorbox}
		\end{enumerate}
		
		You can create multiple files using brace expansion.
		\newline
		Eg:
		\begin{tcolorbox}[breakable,notitle,boxrule=-0pt,colback=black,colframe=black]
			\color{green}
			\fontdimen2\font=1em
			\# touch {Sunday,Monday,Tuesday,Wednesday}.txt
			\newline
			\# touch file{1..3}.txt
			\newline
			\# touch file{a..h}.txt
			\newline
			\# touch file{a,b}{1,2}.txt
			\fontdimen2\font=4pt
		\end{tcolorbox}
		
		\bigskip
		\bigskip
		\item \textbf{cp}: Copy file or directory.
		\bigskip
		\begin{tcolorbox}[breakable,notitle,boxrule=1pt,colback=pink,colframe=pink]
			\color{black}
			\fontdimen2\font=1em
			Syntax: cp source destination
			\fontdimen2\font=4pt
		\end{tcolorbox}
		To copy multiple files to a directory:
			\bigskip
			\begin{tcolorbox}[breakable,notitle,boxrule=-0pt,colback=black,colframe=black]
				\color{green}
				\fontdimen2\font=1em
				\# cp file1 file2 file3 backup
				\fontdimen2\font=4pt
			\end{tcolorbox}

		Options with \textbf{cp} command:
		\begin{enumerate}[label=(\alph*)]
				\item \textbf{-r}: Copy directories recursively including all its files and subdirectories.
				\newline
				Eg:
				\begin{tcolorbox}[breakable,notitle,boxrule=-0pt,colback=black,colframe=black]
					\color{green}
					\fontdimen2\font=1em
					\# cp -r folder1  newfolder
					\fontdimen2\font=4pt
			\end{tcolorbox}
			\item \textbf{Wildcard character "*"}: Copy all files starting with a specific name.
			\bigskip
			\begin{tcolorbox}[breakable,notitle,boxrule=-0pt,colback=black,colframe=black]
				\color{green}
				\fontdimen2\font=1em
				\# cp file* backup
				\fontdimen2\font=4pt
			\end{tcolorbox}
		\end{enumerate}
		\bigskip
		\bigskip		

		
		\item \textbf{mv}: Used to move or rename the file/folder.
		\bigskip
		\begin{tcolorbox}[breakable,notitle,boxrule=1pt,colback=pink,colframe=pink]
			\color{black}
			\fontdimen2\font=1em
			Syntax:  mv source destination
			\fontdimen2\font=4pt
		\end{tcolorbox}
		\begin{enumerate}[label=(\alph*)]
		\item Renames file(or directory) when the \textbf{location of source and destination is same.}
		\bigskip
		\begin{tcolorbox}[breakable,notitle,boxrule=1pt,colback=black,colframe=black]
			\color{green}
			\fontdimen2\font=1em
			\# mv old-file new-file
			\fontdimen2\font=4pt
		\end{tcolorbox}
		\item Moves a files/directory to different location if \textbf{source and destination is different.}
		\bigskip
		\begin{tcolorbox}[breakable,notitle,boxrule=-0pt,colback=black,colframe=black]
			\color{green}
			\fontdimen2\font=1em
			\# mv oldfile new-file-location
			\fontdimen2\font=4pt
		\end{tcolorbox}
		\end{enumerate}
		\bigskip
		\bigskip
	
		\item \textbf{rm}: Deletes one or more files/folders.
		\bigskip
		\begin{tcolorbox}[breakable,notitle,boxrule=1pt,colback=pink,colframe=pink]
			\color{black}
			\fontdimen2\font=1em
			Syntax:  rm filename
			\fontdimen2\font=4pt
		\end{tcolorbox}
		Eg:
		\begin{tcolorbox}[breakable,notitle,boxrule=-0pt,colback=black,colframe=black]
			\color{green}
			\fontdimen2\font=1em
			\# rm one.txt
			\fontdimen2\font=4pt
		\end{tcolorbox}
		Options with \textbf{rm} command:
		\begin{enumerate}[label=(\alph*)]
			\item \textbf{-r}: Remove directories and their contents recursively.
			\bigskip
			\begin{tcolorbox}[breakable,notitle,boxrule=-0pt,colback=black,colframe=black]
				\color{green}
				\fontdimen2\font=1em
				\# rm -r foldername
				\fontdimen2\font=4pt
			\end{tcolorbox}
		\end{enumerate}	
		\bigskip
		\bigskip
		\newpage
		\item \textbf{rmdir}: Deletes only empty directory.
		\bigskip
		\begin{tcolorbox}[breakable,notitle,boxrule=1pt,colback=pink,colframe=pink]
			\color{black}
			\fontdimen2\font=1em
			Syntax:  rmdir directory
			\fontdimen2\font=4pt
		\end{tcolorbox}
		Eg:
		\begin{tcolorbox}[breakable,notitle,boxrule=-0pt,colback=black,colframe=black]
			\color{green}
			\fontdimen2\font=1em
			\# rmdir /tmp/project/
			\fontdimen2\font=4pt
		\end{tcolorbox}
		\bigskip
		\bigskip
		\item \textbf{df}: df stands for \textbf{disk free}. Displays information about disk usage.
		\bigskip
		\begin{tcolorbox}[breakable,notitle,boxrule=1pt,colback=pink,colframe=pink]
			\color{black}
			\fontdimen2\font=1em
			Syntax:  df
			\fontdimen2\font=4pt
		\end{tcolorbox}
		Options with \textbf{df} command:
		\begin{enumerate}[label=(\alph*)]
			\item \textbf{-h}: Display disk-usage size in GBs, MBs etc.
			\newline
			Eg:
			\begin{tcolorbox}[breakable,notitle,boxrule=1pt,colback=black,colframe=black]
				\color{green}
				\fontdimen2\font=1em
				\# df -h
				\fontdimen2\font=4pt
			\end{tcolorbox}
			\item \textbf{-T}: Print filesystem of each partition.
			\newline
			Eg:
			\begin{tcolorbox}[breakable,notitle,boxrule=-0pt,colback=black,colframe=black]
				\color{green}
				\fontdimen2\font=1em
				\# df -T
				\fontdimen2\font=4pt
			\end{tcolorbox}
		\end{enumerate}	
		\bigskip
		\bigskip
		
		\end{enumerate}
\end{flushleft}

\newpage


\subsection{Advance commands}\index{Linux commands!Advance commands}

\begin{flushleft}
	
	\begin{enumerate}
		\item \textbf{whoami}: Display username who is logged in.
		\newline
		Eg:
		\begin{tcolorbox}[breakable,notitle,boxrule=-0pt,colback=black,colframe=black]
			\color{green}
			\# whoami
			\newline
			\color{white}
			jack
		\end{tcolorbox}
		\bigskip
		\bigskip
		
		\item \textbf{users}: Display all the username who are currently logged in.
		\newline
		Eg:
		\begin{tcolorbox}[breakable,notitle,boxrule=-0pt,colback=black,colframe=black]
			\color{green}
			\# users
			\newline
			\fontdimen2\font=1em
			\color{white}
			jack jill lavatech
			\fontdimen2\font=4pt
		\end{tcolorbox}
		
		\bigskip
		
		\item \textbf{who}: Displays users currently logged with more details.
		\newline
		Eg:
		\begin{tcolorbox}[breakable,notitle,boxrule=-0pt,colback=black,colframe=black]
			\color{green}
			\# who
			\newline
			\color{white}
			\fontdimen2\font=1em
			jack   :0           2022-02-02 14:21 (:0)
			\newline
			jill :1           2022-02-03 13:52 (:1)
			\newline
			lavatech pts/0  2022-02-03 14:03 (192.168.0.105)
			\fontdimen2\font=4pt
		\end{tcolorbox}				
		Output explaination:
			\begin{itemize}
				\item Column 1 - Login name
				\item Column 2 - Login device (TTY or pts)
				\item Column 3 - Login date
				\item Column 4 - Login time 
				\item Column 5 - Local device or remote IP address from where the user is logged in
			\end{itemize}
		\bigskip
		\begin{tcolorbox}[breakable,notitle,boxrule=-0pt,colback=yellow,colframe=yellow]
			\color{black}
			\textbf{Note:} 
			\begin{itemize}
				\item TTY stands for \textbf{teletypewriter}: It is an input device that allows alphanumeric character to be typed in and sent to a computer.
				\item The pts/0 is telling which \textbf{"pseudo terminal"} the user is logged in on. 
			\end{itemize}
		\end{tcolorbox}
		Options with \textbf{who} command:
		\begin{itemize}
			\item \textbf{-H}: Prints the column headers.
			\newline
			Eg:
			\begin{tcolorbox}[breakable,notitle,boxrule=-0pt,colback=black,colframe=black]
				\color{green}
				\$ who -H 
			\end{tcolorbox}
			\item \textbf{-b}: Display the time and date of the last reboot.
			\newline
			Eg:
			\begin{tcolorbox}[breakable,notitle,boxrule=-0pt,colback=black,colframe=black]
				\color{green}
				\# who -b
			\end{tcolorbox}
		\end{itemize}
		\bigskip
		\item \textbf{w}: Shows who is logged and what they are doing.
			\newline
			Eg:
			\begin{tcolorbox}[breakable,notitle,boxrule=-0pt,colback=black,colframe=black]
				\color{green}
				\# w
				\color{white}
				\small
				\fontdimen2\font=1em
				\newline
				14:27:00 up 1 day, 5 min,  2 users,  load average: 1.86, 2.44, 2.70
				\newline
				\fontdimen2\font=1.2em
				USER     TTY      FROM             LOGIN@   IDLE   JCPU   PCPU WHAT
				\newline
				jack   :0       :0               Wed14   ?xdm?  14:25m  0.03s /usr/lib/g
				\newline
				lavatech :1       :1               13:52   ?xdm?  14:25m  0.00s /usr/lib/g
				\newline
				\fontdimen2\font=4pt
			\end{tcolorbox}
			
			Output explaination:
			\begin{itemize}
				\item Line 1 - Shows below details:
				\begin{itemize}
					\item Current time
					\item How long the system has been running.
					\item How many users are currently logged in.
					\item System load averages for the past 1, 5, and 15 minutes.
				\end{itemize}
				\item Line 2 - Header
				\item Line 3 - Login name, the tty name, the remote host, login time, idle time, JCPU, PCPU, and the command line of their current process.
			\end{itemize}
			\bigskip
			\bigskip
			\textbf{JCPU} - Time used by all processes attached to the tty.
			\newline
			\textbf{PCPU} - Time used by the current process, named in the "what" field.
		\bigskip
		\item \textbf{uname}: Displays the name of OS.
		\newline
		Eg:
		\begin{tcolorbox}[breakable,notitle,boxrule=-0pt,colback=black,colframe=black]
			\color{green}
			\fontdimen2\font=1em
			\# uname
			\newline
			\color{white}
			Linux
			\fontdimen2\font=4pt
		\end{tcolorbox}
		Options with \textbf{uname} command:
		\begin{itemize}
			\item \textbf{-r}: Displays the current release of OS.
			\newline
			Eg:
			\begin{tcolorbox}[breakable,notitle,boxrule=-0pt,colback=black,colframe=black]
				\color{green}
				\fontdimen2\font=1em
				\# uname -r
				\color{white}
				\newline
				3.6.18-194.el8
				\fontdimen2\font=4pt
			\end{tcolorbox}
			\item \textbf{-a}: Displays:
			\begin{itemize}
				\item Kernel name 
				\item System name 
				\item Kernel release
				\item Kernel version
				\item Machine processor 
				\item Hardware platform
			\end{itemize} 
			Eg:
			\begin{tcolorbox}[breakable,notitle,boxrule=-0pt,colback=black,colframe=black]
				\color{green}
				\fontdimen2\font=1em
				\# uname -a
				\color{white}
				\newline
				Linux lavatech 5.13.0-27-generic \#29~20.04.1-Ubuntu SMP Fri Jan 14 00:32:30 UTC 2022 x86\_64 x86\_64 x86\_64 GNU/Linux
				\fontdimen2\font=4pt
			\end{tcolorbox}
			\item \textbf{-n}: Prints the hostname of machine.
			\newline
			Eg:
			\begin{tcolorbox}[breakable,notitle,boxrule=-0pt,colback=black,colframe=black]
				\color{green}
				\fontdimen2\font=1em
				\# uname -n
				\fontdimen2\font=4pt
			\end{tcolorbox}
		\end{itemize}
		\bigskip
		
		\item \textbf{uptime}: Display belows details in one line:
		\begin{itemize}
			\item Current time
			\item How long the system has been running
			\item How many users are currently logged in
			\item System load averages for the past 1, 5, and 15 minutes
		\end{itemize}
		Eg:
		\begin{tcolorbox}[breakable,notitle,boxrule=-0pt,colback=black,colframe=black]
			\color{green}
			\fontdimen2\font=1em
			\# uptime
			\newline
			\color{white}
			 14:48:33 up 1 day, 27 min,  2 users,  load average: 2.58, 2.60, 2.76
			\fontdimen2\font=4pt
		\end{tcolorbox}
		\bigskip
		\bigskip
		\item \textbf{timedatectl}: Control the system time and date
		\bigskip
		\begin{tcolorbox}[breakable,notitle,boxrule=-0pt,colback=pink,colframe=pink]
			\color{black}
			\fontdimen2\font=1em
			Syntax: timedatectl [options] [arguments]
			\fontdimen2\font=4pt
		\end{tcolorbox}
		Eg:
		\begin{figure}[h!]
			\centering
			\includegraphics[scale=.45]{content/chapter2/images/timedate.png}
			\caption{Sample output}
			\label{fig:path25}
		\end{figure}
		\newline	
		Options with \textbf{timedatectl} command:
		\begin{itemize}
			\item \textbf{list-timezones}: Displays all available timezones.
			\newline
			Eg:
			\begin{tcolorbox}[breakable,notitle,boxrule=-0pt,colback=black,colframe=black]
				\color{green}
				\fontdimen2\font=1em
				\# timedatectl list-timezones
				\fontdimen2\font=4pt
			\end{tcolorbox}
			\item \textbf{set-timezone}: Sets timezone.
			\newline
			Eg:
			\begin{tcolorbox}[breakable,notitle,boxrule=-0pt,colback=black,colframe=black]
				\color{green}
				\fontdimen2\font=1em
				\# timedatectl set-timezone America/Jamaica
				\fontdimen2\font=4pt
			\end{tcolorbox}
		\end{itemize}	
		\bigskip
		\bigskip
		\item \textbf{date}: Display or set the date.
		\begin{enumerate}[label=(\alph*)]
			\item To display the date:
			\newline
			Eg:
			\begin{tcolorbox}[breakable,notitle,boxrule=-0pt,colback=black,colframe=black]
				\color{green}
				\fontdimen2\font=1em
				\# date
				\newline
				\color{white}
				Saturday 30 April 2022 12:24:24 PM IST
				\fontdimen2\font=4pt
			\end{tcolorbox}
			\item To set the date, the format is:
			\bigskip
			\begin{tcolorbox}[breakable,notitle,boxrule=1pt,colback=pink,colframe=pink]
				\color{black}
				\fontdimen2\font=1em
				Syntax: date "+\%formatters"
				\fontdimen2\font=4pt
			\end{tcolorbox}
			Eg:
			\bigskip
			\begin{tcolorbox}[breakable,notitle,boxrule=-0pt,colback=black,colframe=black]
				\color{green}
				\fontdimen2\font=1em
				\# date '+\%a \%h \%d \%T'
				\fontdimen2\font=4pt
			\end{tcolorbox}
		
			Valid date formatters:
			\newline
			\newline
			\begin{tabulary}{1.0\textwidth}{|p{8em}|p{16em}|}
				\toprule
				\textbf{Date Formatter} & \textbf{Description}\\
				\midrule
				\%m & month of year (01-12) \\
				\hline
				\%n & prints output to new line \\
				\hline
				\%d & day of month (01-31) \\
				\hline
				\%y & last two digits of year (00-99) \\
				\hline
				\%D & date as mm/dd/yy \\
				\hline
				\%H & hour (00-23) \\
				\hline
				\%M & minute (00-59) \\
				\hline
				\%S & second (00-59) \\
				\hline
				\%T & time as HH:MM:SS \\
				\hline
				\%j & day of year (001-366) \\
				\hline
				\%w & day of week (0-6) Sunday is 0 \\
				\hline
				\%a & abbreviated weekday (Sun-Sat) \\
				\hline
				\%h & abbreviated month (Jan-Dec) \\
				\hline
				\%r & 12-hour time w/ AM/PM (e.g., "03:59:42 PM")\\
				\bottomrule
			\end{tabulary}
		
			Options with \textbf{date} command:
			
			\begin{itemize}
				\item \textbf{-s datestring}: Sets the time and date to the value specified in the datestring only if you are root user.
				\newline
				Eg:
				\bigskip
				\begin{tcolorbox}[breakable,notitle,boxrule=-0pt,colback=black,colframe=black]
					\color{green}
					\fontdimen2\font=1em
					\# date -s '11/20/2003 12:48:00'
					\fontdimen2\font=4pt
				\end{tcolorbox}
				\item \textbf{-d datestring}: Display the specified date instead of actual current system date.
				\newline
				Eg:
				\bigskip
				\begin{tcolorbox}[breakable,notitle,boxrule=-0pt,colback=black,colframe=black]
					\color{green}
					\fontdimen2\font=1em
					\# date -d "last friday"
					\newline
					\# date -d "next friday"
					\newline
					\# date -d "yesterday"
					\fontdimen2\font=4pt
				\end{tcolorbox}
			\end{itemize}
			
		\end{enumerate}
		\bigskip
		\bigskip
		\item \textbf{cal}: Prints a calendar for the current month.
			\bigskip
			\begin{tcolorbox}[breakable,notitle,boxrule=-0pt,colback=pink,colframe=pink]
				\color{black}
				\fontdimen2\font=1em
				Syntax: cal [options] [[month] year]
				\fontdimen2\font=4pt
			\end{tcolorbox}
			Options with \textbf{cal} command:
			\begin{itemize}
				\item \textbf{-j}: Display julian dates (days numbered 1 to 365, starting from January 1).
				\newline
				Eg: To display calender for month-12 and year-2022 with julian dates.
				\begin{tcolorbox}[breakable,notitle,boxrule=-0pt,colback=black,colframe=black]
					\color{green}
					\fontdimen2\font=1em
					\# cal -j 12 2022
					\fontdimen2\font=4pt
				\end{tcolorbox}
				
				\begin{figure}[h!]
					\centering
					\includegraphics[scale=0.4]{content/chapter2/images/cal1.png}
					\caption{Sample output}
					\label{fig:cal1}
				\end{figure}

				\item \textbf{-m}: Display specific month.
				\newline
				Eg: To display calender of month-12.
				\begin{tcolorbox}[breakable,notitle,boxrule=-0pt,colback=black,colframe=black]
					\color{green}
					\fontdimen2\font=1em
					\# cal -m 12
					\fontdimen2\font=4pt
				\end{tcolorbox}
				
				\begin{figure}[h!]
					\centering
					\includegraphics[scale=0.4]{content/chapter2/images/cal2.png}
					\caption{Sample output}
					\label{fig:cal2}
				\end{figure}
				
			
				\item \textbf{-y}: Display entire year.\newline
				Eg: To display calender of year-2022.
			\begin{tcolorbox}[breakable,notitle,boxrule=-0pt,colback=black,colframe=black]
				\color{green}
				\fontdimen2\font=1em
				\# cal -y 2022
				\fontdimen2\font=4pt
			\end{tcolorbox}
			\begin{figure}[h!]
				\centering
				\includegraphics[scale=0.4]{content/chapter2/images/cal3.png}
				\caption{Sample output}
				\label{fig:cal3}
			\end{figure}
			
			\end{itemize}

	\newpage
		\item \textbf{ifconfig}: Display the IP address of server.
		\newline
		Eg:
		\begin{tcolorbox}[breakable,notitle,boxrule=1pt,colback=black,colframe=black]
			\color{green}
			\fontdimen2\font=1em
			\# ifconfig
			\fontdimen2\font=4pt
		\end{tcolorbox}
			\bigskip
			\bigskip		
		\newpage
		\item \textbf{hostname}: Displays the fully qualified name of server.
		\newline
		Eg:
		\begin{tcolorbox}[breakable,notitle,boxrule=1pt,colback=black,colframe=black]
			\color{green}
			\fontdimen2\font=1em
			\# hostname
			\fontdimen2\font=4pt
		\end{tcolorbox}
		\bigskip
		\bigskip
		
		\item \textbf{free}: Displays amount of free and used memory in the system in bytes.
		\newline
			Options with \textbf{free} command:
			\begin{itemize}
				\item \textbf{-k}: Show the output in Kilobytes
				\item \textbf{-m}: Show the output in Megabytes
				\item \textbf{-g}: Show the output in Gegabytes
			\end{itemize}
		Eg:
		\begin{tcolorbox}[breakable,notitle,boxrule=1pt,colback=black,colframe=black]
			\color{green}
			\fontdimen2\font=1em
			\# free 
			\newline
			\# free -k
			\newline
			\# free -m
			\newline
			\# free -g
			\fontdimen2\font=4pt
		\end{tcolorbox}
		
		
		\bigskip
		\bigskip
		\item \textbf{shutdown}: Shutdown the machine immediately or schedule a shutdown using 24 hour format.
		\bigskip
		\begin{tcolorbox}[breakable,notitle,boxrule=1pt,colback=pink,colframe=pink]
			\color{black}
			\fontdimen2\font=1em
			Syntax:  shutdown [options] [time in 24 hour format]
			\fontdimen2\font=4pt
		\end{tcolorbox}
		\bigskip
		\begin{itemize}
			\item Eg: Shutdown immediately -
			\begin{tcolorbox}[breakable,notitle,boxrule=1pt,colback=black,colframe=black]
				\color{green}
				\fontdimen2\font=1em
				\# shutdown -h now
				\newline
				or
				\newline
				\# poweroff
				\fontdimen2\font=4pt
			\end{tcolorbox}
			\item Eg: Reboot the system immediately -
			\bigskip
			\begin{tcolorbox}[breakable,notitle,boxrule=1pt,colback=black,colframe=black]
				\color{green}
				\fontdimen2\font=1em
				\# shutdown -r now
				\newline
				\color{green}
				or
				\newline
				\color{green}
				\# reboot
				\fontdimen2\font=4pt
			\end{tcolorbox}
			\item Eg: Restart OS at specific time, like at 5:30 pm -
			\bigskip
			\begin{tcolorbox}[breakable,notitle,boxrule=1pt,colback=black,colframe=black]
				\color{green}
				\fontdimen2\font=1em
				\# shutdown 17:30
				\fontdimen2\font=4pt
			\end{tcolorbox}
			\item Eg: Shutdown OS at specific time, like at 5:30 pm -
			\bigskip
			\begin{tcolorbox}[breakable,notitle,boxrule=1pt,colback=black,colframe=black]
				\color{green}
				\fontdimen2\font=1em
				\# shutdown -h 17:30
				\fontdimen2\font=4pt
			\end{tcolorbox}
		\end{itemize}

		\bigskip
		\bigskip
		

		\item \textbf{which}: Shows the full path of the command.
		\bigskip
		\begin{tcolorbox}[breakable,notitle,boxrule=1pt,colback=pink,colframe=pink]
			\color{black}
			\fontdimen2\font=1em
			Syntax: which command\_name
			\fontdimen2\font=4pt
		\end{tcolorbox}
		Eg:
		\bigskip
		\begin{tcolorbox}[breakable,notitle,boxrule=1pt,colback=black,colframe=black]
			\color{green}
			\fontdimen2\font=1em
			\# which cal
			\newline
			\color{white}
			/usr/bin/cal
			\fontdimen2\font=4pt
		\end{tcolorbox}

		
		\bigskip
		\bigskip
		
		\item \textbf{whereis}: Locate the binary, source, and manual page files for a command.
		\bigskip
		\begin{tcolorbox}[breakable,notitle,boxrule=1pt,colback=pink,colframe=pink]
			\color{black}
			\fontdimen2\font=1em
			Syntax: whereis command\_name
			\fontdimen2\font=4pt
		\end{tcolorbox}
		Eg:
		\bigskip
		\begin{tcolorbox}[breakable,notitle,boxrule=1pt,colback=black,colframe=black]
			\color{green}
			\fontdimen2\font=1em
			\# whereis cal
			\newline
			\color{white}
			cal: /usr/bin/cal /usr/share/man/man1/cal.1.gz
			\fontdimen2\font=4pt
		\end{tcolorbox}
		\bigskip
		\bigskip
		

		\item \textbf{sleep}: Suspends the shell by making it inactive for specified seconds.
		\begin{tcolorbox}[breakable,notitle,boxrule=1pt,colback=pink,colframe=pink]
			\color{black}
			\fontdimen2\font=1em
			Syntax: sleep seconds
			\fontdimen2\font=4pt
		\end{tcolorbox}
		Eg:
		\bigskip
		\begin{tcolorbox}[breakable,notitle,boxrule=1pt,colback=black,colframe=black]
			\color{green}
			\fontdimen2\font=1em
			\# sleep 10
			\fontdimen2\font=4pt
		\end{tcolorbox}
		\bigskip
		\bigskip	

				
		\item \textbf{history}: Shows last few commands fired by the current user.
		\newline
		Eg:
		\begin{figure}[h!]
			\centering
			\includegraphics[scale=.4]{content/chapter2/images/history.png}
			\caption{history command output}
			\label{fig:h6}
		\end{figure}
	\newline
		Options using \textbf{history} command:
		\newline
		\textbf{-c}: Clear the history.
			\begin{tcolorbox}[breakable,notitle,boxrule=1pt,colback=pink,colframe=pink]
				\color{black}
				\fontdimen2\font=1em
				Syntax:  history -c
				\fontdimen2\font=4pt
			\end{tcolorbox}
		Shortcuts using \textbf{history} command:
		\newline
		\textbf{!}: Used to repeat the command executed in past, using it's history number.
		\begin{tcolorbox}[breakable,notitle,boxrule=1pt,colback=pink,colframe=pink]
			\color{black}
			\fontdimen2\font=1em
			Syntax:  !history-no
			\fontdimen2\font=4pt
		\end{tcolorbox}
		Eg:
		\begin{figure}[h!]
			\centering
			\includegraphics[scale=.2]{content/chapter2/images/history2.png}
			\caption{history command output}
			\label{fig:h88}
		\end{figure}

		\newpage
		\item \textbf{ping}: Used to check if a machine is reachable or not on the network.
		\newline
		\begin{tcolorbox}[breakable,notitle,boxrule=1pt,colback=pink,colframe=pink]
			\color{black}
			\fontdimen2\font=1em
			Syntax:  ping ip-address/hostname
			\fontdimen2\font=4pt
		\end{tcolorbox}
		Eg:
		\begin{tcolorbox}[breakable,notitle,boxrule=1pt,colback=black,colframe=black]
			\color{green}
			\fontdimen2\font=1em
			\# ping 192.168.1.0
			\newline
			\# ping www.google.com
			\fontdimen2\font=4pt
		\end{tcolorbox}
		\bigskip
		\begin{tcolorbox}[breakable,notitle,boxrule=1pt,colback=yellow,colframe=yellow]
			\color{black}
			Note: Ping works continuously until ctrl+c is pressed.
		\end{tcolorbox}
		\bigskip
		Options with \textbf{ping} command:
		\newline
		\textbf{-cN}: Send data N number of times using ping command.	
		\newline
		Eg:
		\begin{figure}[h!]
			\centering
			\includegraphics[scale=.35]{content/chapter2/images/ping.png}
			\caption{ping command output}
			\label{fig:h1}
		\end{figure}
		\bigskip
		\bigskip
		\item \textbf{man}: Display manual pages of commands.
		\bigskip
		\begin{tcolorbox}[breakable,notitle,boxrule=1pt,colback=pink,colframe=pink]
			\color{black}
			\fontdimen2\font=1em
			Syntax: man command\_name
			\fontdimen2\font=4pt
		\end{tcolorbox}
		Eg:
		\bigskip
		\begin{tcolorbox}[breakable,notitle,boxrule=1pt,colback=black,colframe=black]
			\color{green}
			\fontdimen2\font=1em
			\# man date
			\fontdimen2\font=4pt
		\end{tcolorbox}
		Options with \textbf{man} command:
		\newline
		\textbf{-k}: Search for man page using keyword
		\begin{tcolorbox}[breakable,notitle,boxrule=1pt,colback=pink,colframe=pink]
			\color{black}
			\fontdimen2\font=1em
			Syntax: man -k keyword
			\fontdimen2\font=4pt
		\end{tcolorbox}
		Eg:
		\begin{figure}[h!]
			\centering
			\includegraphics[scale=.5]{content/chapter2/images/man2.png}
			\caption{Command Prompt}
			\label{fig:command_prompt2}
		\end{figure}

	
		\bigskip
		\bigskip


		\item \textbf{whatis}: Provides very brief descriptions of the command.
		\bigskip
		\begin{tcolorbox}[breakable,notitle,boxrule=1pt,colback=pink,colframe=pink]
			\fontdimen2\font=1em
			\color{black}
			Syntax: whatis command\_name
			\fontdimen2\font=4pt
		\end{tcolorbox}
		Eg:
		\bigskip
		\begin{tcolorbox}[breakable,notitle,boxrule=1pt,colback=black,colframe=black]
			\fontdimen2\font=1em
			\color{green}
			\# whatis cal
			\color{white}
			\newline
			cal (1)              - displays a calendar and the date of Easter
			\fontdimen2\font=4pt
		\end{tcolorbox}
		
		\bigskip
		\bigskip
		
		\item \textbf{echo}: Display custom message.
		\bigskip
		\begin{tcolorbox}[breakable,notitle,boxrule=1pt,colback=pink,colframe=pink]
			\color{black}
			\fontdimen2\font=1em
			Syntax: echo "message"
			\fontdimen2\font=4pt
		\end{tcolorbox}
		Eg:
		\bigskip		
		\begin{tcolorbox}[breakable,notitle,boxrule=1pt,colback=black,colframe=black]
			\color{green}
			\fontdimen2\font=1em
			\# echo "lavatech technology training institute"
			\fontdimen2\font=4pt
		\end{tcolorbox}
		Executing command along with custom message:
		\begin{tcolorbox}[breakable,notitle,boxrule=1pt,colback=pink,colframe=pink]
			\color{black}
			\fontdimen2\font=1em
			Syntax: echo "message \$(command)"
			\newline
			or
			\newline
			Syntax: echo "message `command`"
			\fontdimen2\font=4pt
		\end{tcolorbox}
		Eg:
		\bigskip		
		\begin{tcolorbox}[breakable,notitle,boxrule=1pt,colback=black,colframe=black]
			\color{green}
			\fontdimen2\font=1em
			\# echo "lavatech technology training institute, date: \$(date)"
			\newline
			\# echo "lavatech technology training institute, date: `date`"
			\fontdimen2\font=4pt
		\end{tcolorbox}
		\bigskip
		\bigskip	
		
		\item \textbf{alias}: Sets an alias (similar to nickname) for a command.
		\bigskip
		\begin{tcolorbox}[breakable,notitle,boxrule=1pt,colback=pink,colframe=pink]
			\color{black}
			\fontdimen2\font=1em
			Syntax: alias 'shortcutname=command'
			\fontdimen2\font=4pt
		\end{tcolorbox}
		Eg:
		\bigskip
		\begin{tcolorbox}[breakable,notitle,boxrule=1pt,colback=black,colframe=black]
			\color{green}
			\fontdimen2\font=1em
			\# alias 'cls=clear'           
			\fontdimen2\font=4pt
		\end{tcolorbox}
		\bigskip
		\bigskip	
		
		\item \textbf{unalias}: Remove an alias.
		\bigskip
		\begin{tcolorbox}[breakable,notitle,boxrule=1pt,colback=pink,colframe=pink]
			\color{black}
			\fontdimen2\font=1em
			Syntax: unalias shortcutname
			\fontdimen2\font=4pt
		\end{tcolorbox}
		Eg:
		\bigskip
		\begin{tcolorbox}[breakable,notitle,boxrule=1pt,colback=black,colframe=black]
			\color{green}
			\fontdimen2\font=1em
			\# unalias cls
			\fontdimen2\font=4pt
		\end{tcolorbox}
		\bigskip
		\bigskip	
		

		
	\end{enumerate}
	
\end{flushleft}

\newpage


\subsection{Practice}\index{Linux commands!Practice}
\setlength{\columnsep}{3pt}
\begin{flushleft}
	\paragraph{}
	
	\bigskip
	
	\begin{figure}[h!]
		\centering
		\includegraphics[scale=.2]{content/practise.jpg}
	\end{figure}	
	
	\begin{enumerate}
		\item \textbf{Which of the following command is used to list content of the directory? (Select all that applies.)}
		\begin{enumerate}[label=(\alph*)]
			\item pwd
			\item mkdir /test1
			\item ls                  % correct
			\item ls -l /home         % correct
		\end{enumerate}
		\bigskip
		\bigskip
		\item \textbf{What are hidden files in Linux?}
		\begin{enumerate}[label=(\alph*)]
			\item File/directory having name starting the "*" are hidden files.         
			\item File/directory having name starting with "." are hidden files. %corect
			\item File/directory having name starting with ".,." are hidden files.
			\item None of above
		\end{enumerate}
		\bigskip
		\bigskip
		\item \textbf{Which of the following command is used to create directory along with parent directory?}
		\begin{enumerate}[label=(\alph*)]
			\item mkdir /a/b/c
			\item mkdir -d /a/b/c 
			\item mkdir -p /a/b/c  %correct
			\item mkdir -t /a/b/c
		\end{enumerate}
		\bigskip
		\bigskip
		\item \textbf{If "cd" command is used without any options, where does it take you?}
		\begin{enumerate}[label=(\alph*)]
			\item Root directory
			\item Home directory % correct
			\item Parent directory 
			\item Present working directory
		\end{enumerate}
		\bigskip
		\bigskip
		\item \textbf{Which of the following command can be used to create a file? (Select all that applies.)}
		\begin{enumerate}[label=(\alph*)]
			\item cat    %correct
			\item mkdir
			\item cd
			\item touch    %correct
		\end{enumerate}
		\bigskip
		\bigskip
		\item \textbf{What option of "cp" command is used to copy a directory recursively?}
		\begin{enumerate}[label=(\alph*)]
			\item cp -d /source\_folder /destination\_folder
			\item cp -p /source\_folder /destination\_folder
			\item cp -r /source\_folder /destination\_folder  %correct
			\item cp -t /source\_folder /destination\_folder
		\end{enumerate}
		\bigskip
		\bigskip
		\item \textbf{Which of the following command is used to rename or move a file/directory?}
		\begin{enumerate}[label=(\alph*)]
			\item cp
			\item ls
			\item mv %correct
			\item df 
		\end{enumerate}
		\bigskip
		\bigskip
		\item \textbf{Which of the following command is used to delete an empty directory? (Select all that applies.)}
		\begin{enumerate}[label=(\alph*)]
			\item rm -r /foldername   %correct
			\item rmdir /foldername  % correct
			\item rmdir -e /foldername
			\item rm /foldername
		\end{enumerate}
		\bigskip
		\bigskip
		\item \textbf{What command is used to display disk usage in Linux? (Select all that applies.)}
		\begin{enumerate}[label=(\alph*)]
			\item df     %correct
			\item df -T  %correct
			\item df -h  %correct
			\item df -Th %correct
		\end{enumerate}
		\bigskip
		\bigskip
		\item \textbf{Which of the following command is used to display details of all logged in users? (Select all that applies.)}
		\begin{enumerate}[label=(\alph*)]
			\item users  %correct
			\item who  %correct
			\item w  %correct
			\item whoami
		\end{enumerate}  
		\bigskip
		\bigskip
		\item \textbf{Which of the following command displays details of the OS?}
		\begin{enumerate}[label=(\alph*)]
			\item who
			\item w
			\item uname  %correct
			\item df
		\end{enumerate}  
		\bigskip
		\bigskip
		\item \textbf{What command displays the JCPU and PCPU in Linux?}
		\begin{enumerate}[label=(\alph*)]
			\item who
			\item w          %correct
			\item uptime
			\item alais
		\end{enumerate}  
		\bigskip
		\bigskip
		\newpage
		\item \textbf{Select all the valid "date" command examples.}
		\begin{enumerate}[label=(\alph*)]
			\item date -d "yesterday" %correct
			\item date     %correct
			\item date '+\%H'    %correct
			\item date -s "12/20/2022" %correct
		\end{enumerate}  
		\bigskip
		\bigskip
		\item \textbf{Which of the following command is used to display memory usage?}
		\begin{enumerate}[label=(\alph*)]
			\item df 
			\item uptime
			\item w
			\item free -m  %correct
		\end{enumerate}  
		\bigskip
		\bigskip
		\item \textbf{What command is used to create shortcut of any command?}
		\begin{enumerate}[label=(\alph*)]
			\item which   
			\item ping
			\item alais       %correct
			\item echo
		\end{enumerate}
		\bigskip
		\bigskip
		\item \textbf{Which of the following command is used to display IP address?}
		\begin{enumerate}[label=(\alph*)]
			\item history
			\item ifconfig
			\item uptime
			\item free -g  %correct
		\end{enumerate}  
		\bigskip
		\bigskip
		\item \textbf{What is use of "ping" command?}
		\begin{enumerate}[label=(\alph*)]
			\item Check IP address of OS.
			\item Check if the IP address is reachable or not.  % correct
			\item Monitor networing.
			\item Display details about the OS.
		\end{enumerate}  
		\bigskip
		\bigskip
		\item \textbf{What command is used to display full path of any command?}
		\begin{enumerate}[label=(\alph*)]
			\item which       %correct
			\item whereis 
			\item how
			\item echo
		\end{enumerate}  
	\end{enumerate}
	
	
\end{flushleft}
\newpage

\afterpage{\blankpage}
%-----------------------

%----------------------------------------------------------------------------------------
%	CHAPTER 3
%----------------------------------------------------------------------------------------
\chapterimage{index4.png} % Table of contents heading image
\chapter{Editors in Linux}
%-----------------------
\section{Gedit \& vi/vim editor}
\setlength{\columnsep}{3pt}
\begin{flushleft}
	\bigskip
	\bigskip
	\begin{tcolorbox}[breakable,notitle,boxrule=1pt,colback=black,colframe=black]
		\color{white}
		\bigskip
		In this section, you are going to learn:
		\begin{enumerate}
			\item \textbf{What is vi/vim editor?}
			\item \textbf{Using vim editor:}
			\begin{itemize}
				\item \textbf{How to move around in the file?}
				\item \textbf{How to searching for text in the file?}
				\item \textbf{How to save or not save the file?}
				\item \textbf{Other related function related to file editing}
			\end{itemize}

		\end{enumerate}	
		\bigskip
		Finally, there will be a \textbf{small excerise} on these topics to check your knowledge.
		\bigskip
	\end{tcolorbox}
	
	
	\bigskip
	\bigskip
	
	\begin{multicols}{2}
		\vspace*{\fill}
		\vspace*{\fill}
		\vspace*{\fill}
		\vspace*{\fill}
		\vspace*{\fill}
		\vspace*{\fill}
		\vspace*{\fill}
		\vspace*{\fill}
		\vspace*{\fill}
		
		\vfill \null
		\columnbreak
		So let's get started....
		\includegraphics[scale=0.08]{content/linux_section.png}
	\end{multicols}	
	
\end{flushleft}

\newpage


\subsection{Editors in Linux}
\setlength{\columnsep}{3pt}
\begin{flushleft}
	\bigskip
	There are 2 major types of editors in Linux:
	\begin{itemize}
		\item \textbf{Grahpical editor}
		\begin{itemize}
			\item \textbf{gedit}: 	Gedit application is a full-featured text editor.
			\begin{tcolorbox}[breakable,notitle,boxrule=0pt,colback=pink,colframe=pink]
				\color{black}
				\fontdimen2\font=1em
				Syntax: gedit filename
				\fontdimen2\font=4pt
			\end{tcolorbox}
			Eg:
			\begin{figure}[h!]
				\centering
				\includegraphics[scale=0.4]{content/chapter3/images/gedit.png}
				\caption{Sample output}
				\label{fig:cal32}
			\end{figure}
			
		\end{itemize}
		\item \textbf{Text-based editor}
		\begin{itemize}
			\item \textbf{vi or vim}:  Text based editor used in Linux and Mac OS. Let's see more on this.
		\end{itemize}
	\end{itemize}
\end{flushleft}

\newpage






\subsection{Vi/vim in detail}
\input{content/chapter3/3.1.0.tex}
\subsection{Moving around file using vi/vim}

\begin{flushleft}
	
	There are two modes using which you can navigate \& edit a file in vim:
	\begin{itemize}

		\item \textbf{Command mode}: 
		\begin{itemize}
			\item \textbf{Press Esc} to enter command mode.
			\item This is default mode.
			\item File cannot be edited under this mode.
			\item Below keystrokes will move the cursor:
			\begin{itemize}	
				\item \textbf{\$}: Move cursor to the end of the line.
				\item \textbf{17G}: Move cursor to line 17 (i.e type 17 and press shift + g).
				\item \textbf{G}: Move cursor to the last line (i.e press shift + g).
			\end{itemize}
			
		\end{itemize}

		\newline
		\bigskip
		
		\bigskip
		\bigskip
		\item \textbf{Insert mode}: 
		\begin{itemize}
			\item \textbf{Press i or a or o} to enter insert mode.
			\item This mode is used to edit the file.
			\item Below keystrokes will allow you to enter this mode:
			\begin{itemize}
				\item \textbf{i}: Insert text just \textbf{before} the current cursor position.
				\item \textbf{a}: Insert text just \textbf{after} the current cursor position.
				\item \textbf{o}: Insert text into a \textbf{new line} below current line.
				\item \textbf{I}: Insert text at the \textbf{beginning} of the current line.
				\item \textbf{A}: Insert text at the \textbf{end} of the current line.
				\item \textbf{O}: Insert text into a \textbf{new line} above current line.
			\end{itemize}
			
		\end{itemize}

					
	
		
		

	\end{itemize}
	
	
\end{flushleft}

\newpage

\subsection{Searching for text using vi/vim}

\begin{flushleft}

You can search a specific word in vim editor:
\newline
\textbf{Forward search:}
\begin{itemize}
	\item \textbf{Press ESC} to switch to command mode.
	\item \textbf{Press "/"} and type the word to search.
	\newline
	Eg: \textbf{/bunny and press enter}: Jump forward to the first occurrence of the string "bunny" after current cursor position.
	\item \textbf{press n}: Jumps to the next occurrence of the word searched by "/".
	\item \textbf{press N}: Jumps to the previous occurrence of the word searched by “/”.
\end{itemize}

\bigskip\bigskip
\textbf{Backward search:}
\begin{itemize}
	\item \textbf{Press ESC} to switch to command mode.
	\item \textbf{Press "?"} and type the word to search.
	\newline
	Eg: \textbf{?bunny and press enter}: Jump backward to the first occurrence of the string "bunny" before current cursor position.
	\item \textbf{press n}: Jumps to the next occurrence of the word searched by "?".
	\item \textbf{press N}: Jumps to the previous occurrence of the word searched by “?”.
\end{itemize}


\end{flushleft}
\newpage
\subsection{Saving and/or quitting the file using vi/vim}

\begin{flushleft}
	
	
	\bigskip
	File can by saved and/or quitted in vim editor. 
	\newline
	To do this, \textbf{switch to command mode (by pressing Esc)} and then do any of the following:
	\begin{itemize}
		\item \textbf{:wq} or \textbf{:x} and press enter-  \textbf{Save \& quit} the edited file.
		\item \textbf{:q!} and press enter - \textbf{Quit} the editor without saving changes.
		\item \textbf{:w} and press enter - \textbf{Save} the edited file without quitting.
	\end{itemize}
		
\end{flushleft}
\newpage
\subsection{More functions in vi/vim}

\begin{flushleft}
	
	
	\bigskip
	Other useful functions in command mode (\textbf{press Esc to switch to command mode}) are:
	\begin{itemize}
		\item \textbf{u} - Undo the last change to the file (and type "u" again to re-do the change).
		\item \textbf{U} - Undo all changes to the current line.
		\item \textbf{yy} - Copy a line.
		\item \textbf{p} - Paste a line.
		\item \textbf{dd} - Delete a line.
		\item \textbf{yy} - Copy a single line.
		\item \textbf{Nyy} - Copy N number of lines. Eg: To copy 5 lines, press 5yy.
		\item \textbf{p} - Paste a single line.
		\item \textbf{Np} - Paste N number of times a copied line. Eg: To paste a line 5 times, press 5p.
		\item \textbf{dd} - Delete a single line.
		\item \textbf{Ndd} - Delete N number of lines. Eg: To delete 5 lines, press 5dd.
		\item \textbf{:set nu and press enter} - Show all line numbers .
		\item \textbf{:set nonu and press enter} - Removes all line numbers.
		\item \textbf{:s/Joe/Bob and press enter} - Substitute first occurrence of Joe in current line by Bob.
		\item \textbf{:s/Joe/Bob/g and press enter} - Substitute all occurrences of Joe in current line by Bob.
		\item \textbf{:\%s/Joe/Bob/g and press enter} - Substitute every "Joe" to "Bob" throughout the
		document.
	\end{itemize}
	
\end{flushleft}
\newpage
\subsection{Practice}
\setlength{\columnsep}{3pt}
\begin{flushleft}
	\paragraph{}
	
	\bigskip
	
	\begin{figure}[h!]
		\centering
		\includegraphics[scale=.2]{content/practise.jpg}
	\end{figure}	
	
	\begin{enumerate}
		\item \textbf{Which of the following combination is used to save and exit a file in vim editor? (Select all that applies.)}
		\begin{enumerate}[label=(\alph*)]
			\item :x       %correct
			\item :wq      %correct
			\item :wq!                  % correct
			\item :w
		\end{enumerate}
		\bigskip
		\bigskip
		\item \textbf{Which of the following keystroke is used to undo changes in vim editor? (Select all that applies.)}
		\begin{enumerate}[label=(\alph*)]
			\item Press "z" in command mode      
			\item Press "U" in command mode    %correct
			\item Press "u" in command mode   %correct
			\item Press "s" in command mode
		\end{enumerate}
		\bigskip
		\bigskip
		\item \textbf{Which of the following keystroke is used to enter in \textbf{inset mode} in vim editor? (Select all that applies.)}
		\begin{enumerate}[label=(\alph*)]
			\item Press "a" %correct
			\item Press "i" %correct
			\item Press "o" %correct
			\item Press "z"
		\end{enumerate}
		\bigskip
		\bigskip
		\item \textbf{Which of the following keystroke is used to enter in \textbf{command mode} in vim editor?}
		\begin{enumerate}[label=(\alph*)]
			\item Press "CTRL"
			\item Press "ATL"
			\item Press "SHIFT"
			\item Press "ESC"  %correct
		\end{enumerate}
		\bigskip
		\bigskip
		\item \textbf{Which of the following keystroke is used to delete a line in vim editor?}
		\begin{enumerate}[label=(\alph*)]
			\item Press "d" in command mode      
			\item Press "y" in command mode      
			\item Press "p" in command mode      
			\item Press "dd" in command mode        %correct
		\end{enumerate}
		\bigskip
		\bigskip
		\item \textbf{Which of the following keystroke is used to copy 5 lines in vim editor?}
		\begin{enumerate}[label=(\alph*)]
			\item Press "5y" in command mode      
			\item Press "5yy" in command mode        %correct
			\item Press "5p" in command mode      
			\item Press "5dd" in command mode     
		\end{enumerate}	
		\bigskip
		\bigskip
		\item \textbf{Which of the following keystroke is used to paste a line 10 times in vim editor?}
		\begin{enumerate}[label=(\alph*)]
			\item Press "10p" in command mode      
			\item Press "10pp" in command mode        %correct
			\item Press "10y" in command mode      
			\item Press "10yy" in command mode     
		\end{enumerate}	
		\bigskip
		\bigskip
		\item \textbf{Which of the following string helps in setting line numbers in vim editor?}
		\begin{enumerate}[label=(\alph*)]
			\item \textbf{:set nu} in command mode   %correct
			\item \textbf{:set line} in command mode   
			\item \textbf{:s/nu} in command mode  
			\item \textbf{:set nonu} in command mode  
		\end{enumerate}
		\bigskip
		\bigskip
		\item \textbf{Which of the following string is used substitute a word in entire file in vim editor?}
		\begin{enumerate}[label=(\alph*)]
			\item \textbf{:replace/old\_word/new\_word/} in command mode   
			\item \textbf{:s/old\_word/new\_word/g} in command mode 
			\item \textbf{:\%s/old\_word/new\_word/g} in command mode    %correct
			\item \textbf{:set nu} in command mode  
		\end{enumerate}
	\end{enumerate}
	
	
\end{flushleft}
\newpage

\afterpage{\blankpage}

%-----------------------

%----------------------------------------------------------------------------------------
%	CHAPTER 4
%----------------------------------------------------------------------------------------
\chapterimage{index5.png} % Table of contents heading image
\chapter{User \& Group Administration}
%-----------------------
\section{User \& group basics}\index{User \& group basics}
\input{content/chapter4/4.1.tex}
\subsection{What is a user?}\index{User \& group basics!What is a user?}
\setlength{\columnsep}{3pt}
\begin{flushleft}
	\bigskip
	To login to any UNIX machine, you need a user. There are 3 types of users in Linux:

	\begin{figure}[h!]
		\centering
		\includegraphics[scale=0.20]{content/chapter4/images/users.png}
		\caption{User types}
		\label{fig:user}
	\end{figure}

	
	\begin{itemize}
		\item \textbf{Superuser i.e. root}: 
		\begin{itemize}
			\item Automatically created when you install Linux.
			\item Superuser can execute commands under \textbf{/usr/sbin \& /usr/bin}.
		\end{itemize}

		\item \textbf{System users}:
		\begin{itemize}
			\item Created automatically during application or software installation.
			\item Eg: When you install VLC media player in Linux, a user named "vlc" will be created automatically.
			\item These accounts exist to allow different services to interact with
			your computer.
		\end{itemize}
		
		\item \textbf{Normal or Local users}:
		\begin{itemize}
			\item Created by root or sudo user.
			\item Local user can execute commands under \textbf{/usr/bin}.
			\item Eg: jack, jill, ram, ravi etc.
		\end{itemize}
		
	\end{itemize}

\end{flushleft}

\newpage






\subsection{What is a group?}\index{User \& group basics!What is a group?}
\setlength{\columnsep}{3pt}
\begin{flushleft}
	\bigskip
	\begin{figure}[h!]
	\centering
	\includegraphics[scale=.2]{content/chapter4/images/user_group.jpg}
	\caption{User V/S Group}
	\label{fig:user_group}
	\end{figure}
	\bigskip

	\begin{itemize}
		\item Group is a collection of users.
		\item Users are grouped together for granting common permissions to all members of a group.
		\item A user can be a member of more than one group.
	\end{itemize}

\newpage
	\paragraph{Types of group}
	
	\begin{itemize}
		\item \textbf{Primary group}:
		\begin{itemize}
			\item Primary group are automatically generated while creating a user.
			\item A user becomes member of their primary group automatically.
			\item Primary group have same name as the username \& a unique group ID.
			\item \textbf{One user can have one and only one primary group}.
		\end{itemize} 
		\item \textbf{Secondary or supplementary group}:
		\begin{itemize}
			\item Secondary group are created separately with the help of \textbf{groupadd} command.
			\item Multiple users can be added to the secondary group.
			\item \textbf{One user can have more than one secondary groups}.
		\end{itemize} 
	\end{itemize}
	
	\begin{figure}[h!]
		\centering
		\includegraphics[scale=0.5]{content/chapter4/images/type_group.png}
		\caption{Primary and Secondary group}
		\label{fig:prime_secondary_group}
	\end{figure}
	


\end{flushleft}

\newpage


\subsection{Files containing user \& group details}
\setlength{\columnsep}{3pt}
\begin{flushleft}

	Below diagram shows important files and their short description relating to user \& group details:
	\begin{figure}[h!]
	\centering
	\includegraphics[scale=0.4]{content/chapter4/images/users6.png}
	\caption{Important user \& group files}
	\label{fig:prime_secondary_group12}
\end{figure}

Let's see each of these file in detail.

\newpage

\textbf{Structure of /etc/passwd file}
\begin{itemize}
	\item Fields in this file are separated by a ":" (colon).
	\begin{figure}[h!]
	\centering
	\includegraphics[scale=.2]{content/chapter4/images/53.png}
	\caption{Sample /etc/passwd entry}
	\label{fig:user_group1}
	\end{figure}	
	\newline
Explaination:
\begin{enumerate}
	\item \textbf{Username} - Case-sensitive login name of user.
	\item \textbf{Password placeholder} - "x" acts as placeholder. Encrypted password are now stored in \textbf{/etc/shadow} file.
	\item \textbf{User ID}: UID ranges:
		\begin{itemize}
		\item UID 0 is assigned to root user.
		\item UID 1-200 is a range of "system users" assigned to system processes by OS.
		\item UID 201-999 is a range of "system users" used by system processes.
		\item UID 1000+ is the range available for assignment to regular users.
	\end{itemize}
	\item \textbf{Group ID}
	\item \textbf{Comment} - More description about user.
	\item \textbf{Home directory} - Location of user's personal data.
	\item \textbf{Shell} - User login shell, default is \textbf{/bin/bash}.
\end{enumerate}

\end{itemize}	
\newpage



\textbf{Structure of the /etc/shadow file}
\begin{itemize}
	\item Fields in this file are separated by a ":" (colon).
	\begin{figure}[h!]
		\centering
		\includegraphics[scale=.4]{content/chapter4/images/pass4.png}
		\caption{Sample /etc/shadow entry}
		\label{fig:user_group}
	\end{figure}	
	\newline
	Explaination:
	\begin{enumerate}
		\item \textbf{Username} - A direct match to the username in the \textbf{/etc/passwd} file.
		\item \textbf{Encrypted Password} - Store password in encrypted format.
		\bigskip
		\begin{tcolorbox}[breakable,notitle,boxrule=-0pt,colback=yellow,colframe=yellow]
			\color{black}
			\textbf{Note:} 
			\begin{itemize}
				\item \textbf{“!”} in this entry means the account is disabled or locked. 
				\item \textbf{“!!”} mask means password is not assigned.
			\end{itemize}
		\end{tcolorbox}	
		Let's understand the encrypted password in detail.
		\newpage	
		\begin{figure}[h!]
			\centering
			\includegraphics[scale=.5]{content/chapter4/images/pass.png}
		\end{figure}
		There are three pieces of information stored in password:
		\begin{itemize}
			\item \textbf{Hashing algorithm}: The number 1 indicates an MD5 hash. The number 6 appears when a SHA-512 hash is used.
			\item \textbf{Salt value}: The random salt value used to encrypt the hash. Prevents two users with the same password from having identical entries in the /etc/shadow file.
			\item \textbf{Encrypted hash}
		\end{itemize}
		\bigskip
		\bigskip
		\item \textbf{Password age}:
		\begin{figure}[h!]
			\centering
			\includegraphics[scale=.2]{content/chapter4/images/age.png}
		\end{figure}
		\begin{enumerate}\addtocounter{enumi}{3}
			\item \textbf{Last password change} - The number of days (since January 1, 1970) since the password was last changed.
			\item \textbf{Minimum days before password change} - The number of days before password may be changed (0 indicates it may be changed at any time).
			\item \textbf{Maximum days before password change} - The number of days after which password must be changed (99999 indicates user can keep his or her password unchanged for many, many years).
			\item \textbf{Password change warning} - The number of days to warn user of an expiring password (7 for a full week).
			\item \textbf{Account active days} - The number of days an account remains active after a password has expired.
			\item \textbf{No. of days account is expired} - The account expiration date, represented as the number of days since 1970.01.01.
			\item This blank field is reserved for future use.
		\end{enumerate}
	\end{enumerate}
\end{itemize}	
	
\newpage

\textbf{Structure of the /etc/group file}
\begin{itemize}
	\item Fields in this file are separated by a ":" (colon).
	\begin{figure}[h!]
		\centering
		\includegraphics[scale=.2]{content/chapter4/images/54.png}
		\caption{Sample /etc/group entry}
		\label{fig:user_group}
	\end{figure}	
	\newline
	Explaination
	\begin{itemize}
		\item Column 1: \textbf{Groupname} - Name of group. It is case-sensitive. 
		\item Column 2: \textbf{Group password placeholder} - "x" acts as placeholder. Encrypted password are stored in \textbf{/etc/gshadow} file.
		\item Column 3: \textbf{Group ID} 
		\item Column 4: \textbf{Group members} - Usernames of users belonging to this group.
	\end{itemize}	
\end{itemize}	

\newpage
\textbf{Structure of the /etc/gshadow file}
\begin{itemize}
	\item Fields in this file are separated by a ":" (colon).
	\begin{figure}[h!]
		\centering
		\includegraphics[scale=.2]{content/chapter4/images/56.png}
		\caption{Sample /etc/gshadow entry}
		\label{fig:user_group}
	\end{figure}	
	\newline
	Explaination:
	\begin{itemize}
		\item Column 1: \textbf{Groupname} - Name of group. It is case-sensitive. 
		\item Column 2: \textbf{Encrypted Password} - Store password in encrypted format.
		\bigskip
		\begin{tcolorbox}[breakable,notitle,boxrule=-0pt,colback=yellow,colframe=yellow]
			\color{black}
			\textbf{Note:} 
			\begin{itemize}
				\item {“!”} in this entry means the account is disabled or locked. 
				\item {“!!”} mask means password is not assigned.
			\end{itemize}
		\end{tcolorbox}

		\item Column 3: \textbf{Administrators} - Users who can change the password.
		\item Column 4: \textbf{Group members} - Usernames of user belonging to this group.
	\end{itemize}	
\end{itemize}	

	
\end{flushleft}

\newpage


\subsection{Practice}\index{User \& group basics!Practice}
\setlength{\columnsep}{3pt}
\begin{flushleft}
	\paragraph{}
	
	\bigskip
	
	\begin{figure}[h!]
		\centering
		\includegraphics[scale=.2]{content/practise.jpg}
	\end{figure}	
	
	\begin{enumerate}
		\item \textbf{State whether true or false. One user can have one \& only one primary group and 0 or more secondary groups.}
		\begin{enumerate}[label=(\alph*)]
			\item True %correct
			\item False
		\end{enumerate}
		\bigskip
		\bigskip
		\item \textbf{Which of the following file contains user details?}
		\begin{enumerate}[label=(\alph*)]
			\item /etc/shadow
			\item /etc/skel
			\item /etc/group
			\item /etc/passwd %correct
		\end{enumerate}
		\bigskip
		\bigskip
		\item \textbf{Which of the following file contains user password details?}
		\begin{enumerate}[label=(\alph*)]
			\item /etc/group
			\item /etc/gshadow
			\item /etc/shadow  %correct
			\item /etc/passwd
		\end{enumerate}
		\bigskip
		\bigskip
		\item \textbf{State whether true or false. Default shell of user is /bin/bash.}
		\begin{enumerate}[label=(\alph*)]
			\item True  %correct
			\item False
		\end{enumerate}
		\bigskip
		\bigskip
		\item \textbf{State whether true or false. Group is collection of users.}
		\begin{enumerate}[label=(\alph*)]
			\item True  %correct
			\item False
		\end{enumerate}
		\bigskip
		\bigskip
		\item \textbf{Which of the following are valid fields of /etc/passwd file? (Select all that applies.)}
		\begin{enumerate}[label=(\alph*)]
			\item shell                  %correct
			\item home directory        %correct
			\item comment	%correct
			\item username	%correct
		\end{enumerate}
		\bigskip
		\bigskip
		\item \textbf{State whether true or false. Primary group is created automatically at the time of user creation.}
		\begin{enumerate}[label=(\alph*)]
			\item True       %correct
			\item False
		\end{enumerate}
		\bigskip
		\bigskip
		\item \textbf{State whether true or false. Secondary group is created automatically at the time of user creation.}
		\begin{enumerate}[label=(\alph*)]
			\item True       
			\item False  %correct
		\end{enumerate}
	\end{enumerate}
	
	
\end{flushleft}
\newpage




\section{User \& group commands}\index{User \& group commands}
\input{content/chapter4/4.2.tex}
\subsection{User commands}\index{User \& group commands!User commands}
\setlength{\columnsep}{3pt}
\begin{flushleft}
	
	Let's see some commands to create, update \& delete users in Linux OS.
	
	\paragraph{How to create a new user?}
	\bigskip
	\textbf{useradd}: Create a new user.
	\begin{tcolorbox}[breakable,notitle,boxrule=1pt,colback=pink,colframe=pink]
		\color{black}
		\fontdimen2\font=1em
		Syntax:  useradd username
		\fontdimen2\font=4pt
	\end{tcolorbox}
	On adding a new user, a user have below things set:
	\begin{itemize}
		\item Home directory: \textbf{/home/username}
		\item Default shell: \textbf{/bin/bash}
		\item Primary group: Named same as \textbf{username}
		\item Password: \textbf{No password is set}
	\end{itemize}
	
	\begin{tcolorbox}[breakable,notitle,boxrule=1pt,colback=yellow,colframe=yellow]
		\color{black}
		Note: Some defaults, such as the UID numbers and default password aging rules, are read from the \textbf{/etc/login.defs} file.
	\end{tcolorbox}
	
	Options with \textbf{useradd} command:
	\begin{enumerate}[label=(\alph*)]
		\item \textbf{–g}: Assign primary group.
		\bigskip
		\begin{tcolorbox}[breakable,notitle,boxrule=0pt,colback=pink,colframe=pink]
			\color{black}
			\fontdimen2\font=1em
			Syntax: useradd -g primary\_group\_name username
			\fontdimen2\font=4pt
		\end{tcolorbox}
		Eg:
		\bigskip
		\begin{tcolorbox}[breakable,notitle,boxrule=-0pt,colback=black,colframe=black]
			\color{green}
			\fontdimen2\font=1em
			\# useradd –g javadevl shekhar
			\fontdimen2\font=4pt
		\end{tcolorbox}
		
		\item \textbf{–G}: Assign secondary groups.
		\bigskip
		\begin{tcolorbox}[breakable,notitle,boxrule=0pt,colback=pink,colframe=pink]
			\color{black}
			\fontdimen2\font=1em
			Syntax: useradd -G secondary\_group\_name username
			\fontdimen2\font=4pt
		\end{tcolorbox}
		Eg:
		\bigskip
		\begin{tcolorbox}[breakable,notitle,boxrule=-0pt,colback=black,colframe=black]
			\color{green}
			\fontdimen2\font=1em
			\# useradd –G cdevl,perldevl shekhar
			\fontdimen2\font=4pt
		\end{tcolorbox}
		
		\newpage
		\item \textbf{–u}: Assign specific UID. UID should be more than 500 for normal users.
		\bigskip
		\begin{tcolorbox}[breakable,notitle,boxrule=0pt,colback=pink,colframe=pink]
			\color{black}
			\fontdimen2\font=1em
			Syntax: useradd –u UID username
			\fontdimen2\font=4pt
		\end{tcolorbox}
		Eg:
		\bigskip
		\begin{tcolorbox}[breakable,notitle,boxrule=-0pt,colback=black,colframe=black]
			\color{green}
			\fontdimen2\font=1em
			\# useradd –u 601 shekhar
			\fontdimen2\font=4pt
		\end{tcolorbox}
		
		
		\item \textbf{–s}: Change user shell.
		\bigskip
		\begin{tcolorbox}[breakable,notitle,boxrule=0pt,colback=pink,colframe=pink]
			\color{black}
			\fontdimen2\font=1em
			Syntax: useradd –s shell\_name username
			\fontdimen2\font=4pt
		\end{tcolorbox}
		Eg:
		\bigskip
		\begin{tcolorbox}[breakable,notitle,boxrule=-0pt,colback=black,colframe=black]
			\color{green}
			\fontdimen2\font=1em
			\# useradd –s /sbin/nologin shekhar
			\fontdimen2\font=4pt
		\end{tcolorbox}
		\bigskip
		\begin{tcolorbox}[breakable,notitle,boxrule=-0pt,colback=yellow,colframe=yellow]
			\color{black}
			Note: Shell \textbf{/sbin/nologin} will not allow username to login. Usually used for system users like ftp, squid etc.
		\end{tcolorbox}
		
		
		\item \textbf{–d}: Change user's home directory.
		\bigskip
		\begin{tcolorbox}[breakable,notitle,boxrule=0pt,colback=pink,colframe=pink]
			\color{black}
			\fontdimen2\font=1em
			Syntax: useradd –d directory\_name username
			\fontdimen2\font=4pt
		\end{tcolorbox}
		Eg:
		\bigskip
		\begin{tcolorbox}[breakable,notitle,boxrule=-0pt,colback=black,colframe=black]
			\color{green}
			\fontdimen2\font=1em
			\# useradd –d /mnt/shekar shekhar 
			\fontdimen2\font=4pt
		\end{tcolorbox}
	\end{enumerate}
	
	\bigskip
	\bigskip
	
	\newpage
	
	\paragraph{How to assign or change user’s password?}
	
	\bigskip
	\textbf{passwd}: Assign/change user's password.
	
	\begin{tcolorbox}[breakable,notitle,boxrule=0pt,colback=pink,colframe=pink]
		\color{black}
		\fontdimen2\font=1em
		Syntax: passwd username
		\fontdimen2\font=4pt
	\end{tcolorbox}
	On executing the command, you will be asked to set password twice.
	
	\begin{tcolorbox}[breakable,notitle,boxrule=0pt,colback=yellow,colframe=yellow]
		\color{black}
		Note: 
		\begin{itemize}
			\item Root user or superuser can set the password of any user.
			\item Local user can set their own password, but not of anyone else.
		\end{itemize}

	\end{tcolorbox}
	
	Options with \textbf{passwd} command:
	
	\begin{itemize}
		\item \textbf{- -stdin}: Change password by providing it on command line itself.
		\begin{tcolorbox}[breakable,notitle,boxrule=0pt,colback=pink,colframe=pink]
			\color{black}
			\fontdimen2\font=1em
			Syntax: echo “new\_password” | passwd ---stdin username
			\fontdimen2\font=4pt
		\end{tcolorbox}
		Eg:
		\bigskip
		\begin{tcolorbox}[breakable,notitle,boxrule=-0pt,colback=black,colframe=black]
			\color{green}
			\fontdimen2\font=1em
			\# echo “shekhar@12345” | passwd ---stdin shekhar
			\fontdimen2\font=4pt
		\end{tcolorbox}
	\end{itemize}
\newpage

\newpage
\paragraph{How to change user’s password attributes?}

\bigskip
\textbf{chage}: Changing the password aging information for user.

\begin{tcolorbox}[breakable,notitle,boxrule=0pt,colback=pink,colframe=pink]
	\color{black}
	\fontdimen2\font=1em
	Syntax: chage username
	\fontdimen2\font=4pt
\end{tcolorbox}
Eg:
\begin{figure}[h!]
	\centering
	\includegraphics[scale=.5]{content/chapter4/images/chage1.png}
	\caption{Sample output}
	\label{fig:command_prompt5}
\end{figure}

Options with \textbf{chage} command:

\begin{itemize}
	\item \textbf{-l}: Show account aging information.
	\begin{tcolorbox}[breakable,notitle,boxrule=0pt,colback=pink,colframe=pink]
		\color{black}
		\fontdimen2\font=1em
		Syntax: chage -l username
		\fontdimen2\font=4pt
	\end{tcolorbox}
	Eg:
	\bigskip
	\begin{figure}[h!]
		\centering
		\includegraphics[scale=.5]{content/chapter4/images/chage2.png}
		\caption{Sample output}
		\label{fig:command_prompt8}
	\end{figure}
\end{itemize}

\newpage

\paragraph{Switch between users}
\begin{itemize}
	\item \textbf{su}: Allows a user to switch to a different user account. If a username is not specified, the root account is implied.
		\begin{tcolorbox}[breakable,notitle,boxrule=0pt,colback=pink,colframe=pink]
		\color{black}
		\fontdimen2\font=1em
		Syntax: su [-] [username]
		\fontdimen2\font=4pt
	\end{tcolorbox}
	Eg:
	\bigskip
	\begin{tcolorbox}[breakable,notitle,boxrule=-0pt,colback=black,colframe=black]
		\color{green}
		\fontdimen2\font=1em
		\# su - jack
		\fontdimen2\font=4pt
	\end{tcolorbox}
	What is used of "-" in su command?
	\begin{itemize}
		\item The command \textbf{"su username"} starts a non-login shell, while the command \textbf{"su - username"} starts a login shell. 
		\item The main distinction is \textbf{"su -"} sets up the shell environment as if this were a clean login as that user, while \textbf{"su"} just starts a shell as that user with the current environment settings.
	\end{itemize}
\end{itemize}

\paragraph{Running command as root with sudo}
\begin{itemize}
	\item \textbf{sudo}: Allows a user to be permitted to run a command as root, or as another user,
	based on settings in the /etc/sudoers file.
	\begin{tcolorbox}[breakable,notitle,boxrule=0pt,colback=pink,colframe=pink]
		\color{black}
		\fontdimen2\font=1em
		Syntax: sudo command
		\fontdimen2\font=4pt
	\end{tcolorbox}
	Eg:
	\bigskip
	\begin{tcolorbox}[breakable,notitle,boxrule=-0pt,colback=black,colframe=black]
		\color{green}
		\fontdimen2\font=1em
		\$ sudo useradd raman
		\fontdimen2\font=4pt
	\end{tcolorbox}
\end{itemize}

	
	\newpage
	
	\paragraph{How to modify an existing user?}
	\bigskip
	\textbf{usermod}: Modify existing user.
	\newline
	Options with \textbf{usermod} command:
	\begin{enumerate}[label=(\alph*)]
		\item \textbf{–g}: Change user's primary group.
		\bigskip
		\begin{tcolorbox}[breakable,notitle,boxrule=0pt,colback=pink,colframe=pink]
			\color{black}
			\fontdimen2\font=1em
			Syntax: usermod –g primary\_group username
			\fontdimen2\font=4pt
		\end{tcolorbox}
		Eg:
		\bigskip
		\begin{tcolorbox}[breakable,notitle,boxrule=-0pt,colback=black,colframe=black]
			\color{green}
			\fontdimen2\font=1em
			\# usermod –g cdevl shekhar
			\fontdimen2\font=4pt
		\end{tcolorbox}
		
		\item \textbf{–G}: Change user's secondary group.
		\bigskip
		\begin{tcolorbox}[breakable,notitle,boxrule=0pt,colback=pink,colframe=pink]
			\color{black}
			\fontdimen2\font=1em
			Syntax: usermod -G secondary\_group\_name username
			\fontdimen2\font=4pt
		\end{tcolorbox}
		Eg:
		\bigskip
		\begin{tcolorbox}[breakable,notitle,boxrule=-0pt,colback=black,colframe=black]
			\color{green}
			\fontdimen2\font=1em
			\# usermod –G javadevl shekhar
			\fontdimen2\font=4pt
		\end{tcolorbox}
		
		
		\item \textbf{–L}: Lock (i.e. disable) user.
		\bigskip
		\begin{tcolorbox}[breakable,notitle,boxrule=0pt,colback=pink,colframe=pink]
			\color{black}
			\fontdimen2\font=1em
			Syntax: usermod -L username
			\fontdimen2\font=4pt
		\end{tcolorbox}		
		
		\item \textbf{–U}: Unlock user account.
		\bigskip
		\begin{tcolorbox}[breakable,notitle,boxrule=0pt,colback=pink,colframe=pink]
			\color{black}
			\fontdimen2\font=1em
			Syntax: usermod –U username
			\fontdimen2\font=4pt
		\end{tcolorbox}
		
		\item \textbf{–s}: Change user's login shell.
		\bigskip
		\begin{tcolorbox}[breakable,notitle,boxrule=0pt,colback=pink,colframe=pink]
			\color{black}
			\fontdimen2\font=1em
			Syntax: usermod –s shell\_name username
			\fontdimen2\font=4pt
		\end{tcolorbox}
		Eg:
		\bigskip
		\begin{tcolorbox}[breakable,notitle,boxrule=-0pt,colback=black,colframe=black]
			\color{green}
			\fontdimen2\font=1em
			\# usermod –s /bin/ksh shekhar
			\fontdimen2\font=4pt
		\end{tcolorbox}
		
		\newpage
		\item \textbf{–d}: Change user's home directory.
		\item \textbf{-m}: Create user's new home directory.
		\bigskip
		\begin{tcolorbox}[breakable,notitle,boxrule=0pt,colback=pink,colframe=pink]
			\color{black}
			\fontdimen2\font=1em
			Syntax: usermod –d directory\_name username -m
			\fontdimen2\font=4pt
		\end{tcolorbox}
		Eg:
		\bigskip
		\begin{tcolorbox}[breakable,notitle,boxrule=-0pt,colback=black,colframe=black]
			\color{green}
			\fontdimen2\font=1em
			\# usermod –d /opt/java shekhar -m
			\fontdimen2\font=4pt
		\end{tcolorbox}
	\end{enumerate}
	
	\bigskip
	\bigskip
	
	\newpage
	
	\paragraph{How to delete a user?}
	
	\bigskip
	\begin{tcolorbox}[breakable,notitle,boxrule=-0pt,colback=red,colframe=red]
		\color{white}
		Standard Practice: You should not delete a user if they leave	 the organization. You
		should lock their account.
	\end{tcolorbox}
	
	
	
	\textbf{userdel}: Delete user \& remove it's entry from \textbf{/etc/passwd} and \textbf{/etc/shadow} files.
	\begin{tcolorbox}[breakable,notitle,boxrule=0pt,colback=pink,colframe=pink]
		\color{black}
		\fontdimen2\font=1em
		Syntax: userdel username
		\fontdimen2\font=4pt
	\end{tcolorbox}
	Eg:
	\begin{tcolorbox}[breakable,notitle,boxrule=-0pt,colback=black,colframe=black]
		\color{green}
		\fontdimen2\font=1em
		\# userdel shekhar
		\fontdimen2\font=4pt
	\end{tcolorbox}

	\begin{tcolorbox}[breakable,notitle,boxrule=-0pt,colback=yellow,colframe=yellow]
		\color{black}
		Note: userdel command does not delete user's home directory by default.
	\end{tcolorbox}	

	Options with \textbf{userdel} command:
	\begin{itemize}
		\item \textbf{-r}: Delete user along with home directory.
		\begin{tcolorbox}[breakable,notitle,boxrule=0pt,colback=pink,colframe=pink]
			\color{black}
			\fontdimen2\font=1em
			Syntax: userdel -r username
			\fontdimen2\font=4pt
		\end{tcolorbox}
		Eg:
		\begin{tcolorbox}[breakable,notitle,boxrule=-0pt,colback=black,colframe=black]
			\color{green}
			\fontdimen2\font=1em
			\# userdel -r shekhar
			\fontdimen2\font=4pt
		\end{tcolorbox}
	\end{itemize}
	
\end{flushleft}

\newpage


\subsection{Content of /etc/skel directory}\index{User \& group commands!Content of /etc/skel directory}
\setlength{\columnsep}{3pt}
\begin{flushleft}

\textbf{/etc/skel} directory:
\begin{itemize}
	\item Content of this directory is \textbf{automatically copied over to a new user's home directory when it is created}.
	\item Users gets same intial settings and environment using content of this directory.
	\item Below are the files under \textbf{/etc/skel} directory that are copied:
	\begin{itemize}
		\item \textbf{/etc/skel/.bash\_profile} copied as \textbf{{$\sim$}/.bash\_profile}: 
		\begin{itemize}
			\item Configures the bash startup environment.
			\item You can the add environment variables to this file.
		\end{itemize}
		\bigskip
		\item \textbf{/etc/skel/.bashrc} copied as \textbf{{$\sim$}/.bashrc}: 
		\begin{itemize}
			\item In this file you can include commands you want to run when you start the bash shell.
			\item You can also add aliases such as: \textbf{rm='rm -i'}
		\end{itemize}
		\bigskip
		\item \textbf{/etc/skel/.bash\_logout} copied as \textbf{{$\sim$}/.bash\_logout}: 
		\newline
		This file is executed when you exit a bash shell.
	\end{itemize}	
	\bigskip
	
\end{itemize}

Another file containing user's login setting is \textbf{/etc/profile}. Note that this file is not copied in the user's home directory.





	
\end{flushleft}

\newpage


\subsection{Group commands}\index{User \& group commands!Group commands}
\setlength{\columnsep}{3pt}
\begin{flushleft}
	
	\bigskip
	\textbf{groupadd}: Create a new group.
	\begin{tcolorbox}[breakable,notitle,boxrule=1pt,colback=pink,colframe=pink]
		\color{black}
		Syntax:  groupadd groupname
	\end{tcolorbox}
	Eg:
	\begin{tcolorbox}[breakable,notitle,boxrule=-0pt,colback=black,colframe=black]
		\color{green}
		\fontdimen2\font=1em
		\# groupadd project
		\fontdimen2\font=4pt
	\end{tcolorbox}

		
	\textbf{groupdel}: Delete a group.
	\begin{tcolorbox}[breakable,notitle,boxrule=0pt,colback=pink,colframe=pink]
			\color{black}
			\fontdimen2\font=1em
			Syntax: groupdel groupname
			\fontdimen2\font=4pt
	\end{tcolorbox}
	Eg:
	\begin{tcolorbox}[breakable,notitle,boxrule=-0pt,colback=black,colframe=black]
		\color{green}
		\fontdimen2\font=1em
		\# groupdel project
		\fontdimen2\font=4pt
	\end{tcolorbox}

	
	
\end{flushleft}

\newpage

\subsection{Practice}\index{User \& group commands!Practice}
\setlength{\columnsep}{3pt}
\begin{flushleft}
	\paragraph{}
	
	\bigskip
	
	\begin{figure}[h!]
		\centering
		\includegraphics[scale=.2]{content/practise.jpg}
	\end{figure}	
	
	\begin{enumerate}
		\item \textbf{Which of the following "useradd" and "usermod" command option is used to assign primary group to user?}
		\begin{enumerate}[label=(\alph*)]
			\item \textbf{-g}   %correct
			\item \textbf{-G}
			\item \textbf{-S}
			\item \textbf{-s}
		\end{enumerate}
		\bigskip
		\bigskip
		\item \textbf{Which of the following command is used to assign password to user?}
		\begin{enumerate}[label=(\alph*)]
			\item chage  
			\item passwd %correct
			\item useradd
			\item usermod
		\end{enumerate}
		\bigskip
		\bigskip
		\item \textbf{Which of the following "userdel" command option is used to delete user along with home directory?}
		\begin{enumerate}[label=(\alph*)]
			\item \textbf{-R}
			\item \textbf{-m}  
			\item \textbf{-r}  %correct
			\item \textbf{-d}
		\end{enumerate}
		\bigskip
		\bigskip
		\item \textbf{Which of the following "useradd" and "usermod" command option is used to assign secondary group to user?}
		\begin{enumerate}[label=(\alph*)]
			\item \textbf{-g}
			\item \textbf{-G}    %correct
			\item \textbf{-S}
			\item \textbf{-s}
		\end{enumerate}
		\bigskip
		\bigskip
		\item \textbf{Which of the following actions are performed during user creation? (Select all that applies.)}
		\begin{enumerate}[label=(\alph*)]
			\item User home directory is created    %correct
			\item Default shell of user is /bin/bash    %correct
			\item Secondary group for user is created
			\item User is assigned primary group    %correct
		\end{enumerate}
		\bigskip
		\bigskip
		\item \textbf{Which of the following "useradd" command option is used to create and assign home directory to user.}
		\begin{enumerate}[label=(\alph*)]
			\item \textbf{-h} \& \textbf{-M}
			\item \textbf{-h} \& \textbf{-m}   %correct
			\item \textbf{-m} \& \textbf{H}
			\item \textbf{-H} \& \textbf{-M}
		\end{enumerate}
		\bigskip
		\bigskip
		\item \textbf{Which of the following "usermod" command option is used to change home directory of the user?}
		\begin{enumerate}[label=(\alph*)]
			\item \textbf{-D}
			\item \textbf{-h}
			\item \textbf{-d}   %correct
			\item \textbf{-m}
		\end{enumerate}
		\bigskip
		\bigskip
		\newpage
		\item \textbf{Which of the following "usermod" command option is used to lock and unlock user account?}
		\begin{enumerate}[label=(\alph*)]
			\item \textbf{-L} \& \textbf{U}  %correct
			\item \textbf{-l} \& \textbf{u} 
			\item \textbf{-d} \& \textbf{m}  
			\item \textbf{-c} \& \textbf{m}
		\end{enumerate}
	\end{enumerate}
	
	
\end{flushleft}
\newpage


%-----------------------

%----------------------------------------------------------------------------------------
%	CHAPTER 5
%----------------------------------------------------------------------------------------
\chapterimage{index6.png} % Table of contents heading image
\chapter{Permissions in Linux}
%-----------------------
\section{File/Directory permissions}\index{Permissions}
\input{content/chapter5/5.1.tex}
\subsection{Introduction to permissions}\index{Permissions!Introduction to permissions}
\setlength{\columnsep}{3pt}
\begin{flushleft}
	
	\begin{itemize}
		\item File/directory permissions are used to control who is able to read, write and execute
		a certain file/directory.
		\item Command to check file/directory permission:
		\bigskip
		\begin{tcolorbox}[breakable,notitle,boxrule=0pt,colback=pink,colframe=pink]
			\color{black}
			\fontdimen2\font=1em
			Syntax: ls -l file/directory
			\fontdimen2\font=4pt
		\end{tcolorbox}
		Eg:
		\bigskip
		\begin{tcolorbox}[breakable,notitle,boxrule=1pt,colback=black,colframe=black]
			\color{green}
			\fontdimen2\font=1em
			\$ ls -l /home/jack
			\color{white}
			\newline
			drwxrwxr-x 2 jack jack 4096 Feb  2 17:02 Desktop
			\fontdimen2\font=4pt
		\end{tcolorbox}
		\bigskip		
		\bigskip
			The first column of the output shows the permission.
		\begin{figure}[h!]
			\centering
			\includegraphics[scale=0.5]{content/chapter5/images/permission.png}
			\caption{"ls -l" output}
			\label{fig:Sample permission}
		\end{figure}
		
		Let's see this column in detail.
	\end{itemize}

	
	
	
\end{flushleft}

\newpage


\subsection{Type of files}\index{Permissions!Type of files}
\setlength{\columnsep}{3pt}
\begin{flushleft}

	First character from the permissions filed defines the file type. 
	
	\begin{figure}[h!]
		\centering
		\includegraphics[scale=0.8]{content/chapter5/images/file_type1.png}
		\caption{First column of permission}
		\label{fig:First column permission}
	\end{figure}
	
	
	There are seven type of files in Linux:

	\begin{figure}[h!]
		\centering
		\includegraphics[scale=0.65]{content/chapter5/images/perm.png}
		\caption{File types with example}
		\label{fig:Sample_permission2}
	\end{figure}
	
\end{flushleft}

\newpage


\subsection{Type of users}\index{Permissions!Type of users}
\setlength{\columnsep}{3pt}
\begin{flushleft}
		
	\begin{figure}[h!]
		\centering
		\includegraphics[scale=0.6]{content/chapter5/images/type_user1.png}
		\caption{User types}
		\label{fig:user_types}
	\end{figure}
	
	Let's map these user with the permissions:
	
	\begin{figure}[h!]
		\centering
		\includegraphics[scale=0.6]{content/chapter5/images/type_user3.png}
		\caption{Permission mapping}
		\label{fig:permission_mapping}
	\end{figure}
	
\end{flushleft}

\newpage


\subsection{Types of permissions}\index{Permissions!Type of permissions}
\setlength{\columnsep}{3pt}
\begin{flushleft}
	There are 3 permission available for all types of files:
	\begin{figure}[h!]
		\centering
		\includegraphics[scale=0.5]{content/chapter5/images/perm2.png}
		\bigskip
		\caption{File Permission}
		\label{fig:file_permission}
	\end{figure}

	Let's see what does this means for a normal file and directory.
	\newpage
	\paragraph{Read means:}
	\begin{figure}[h!]
		\centering
		\includegraphics[scale=0.4]{content/chapter5/images/8.png}
		\caption{Read Permission}
		\label{fig:read_permission}
	\end{figure}

	\paragraph{Write means:}
	\begin{figure}[h!]
		\centering
		\includegraphics[scale=0.4]{content/chapter5/images/9.png}
		\caption{Write Permission}
		\label{fig:write_permission}
	\end{figure}

	\paragraph{Execute means:}
	\begin{figure}[h!]
		\centering
		\includegraphics[scale=0.4]{content/chapter5/images/perm3.png}
		\caption{Execute Permission}
		\label{fig:execute_permission}
	\end{figure}



	
\end{flushleft}

\newpage


\subsection{Permission commands}\index{Permissions!Permission commands}
\input{content/chapter5/5.1.5.tex}
\subsection{Umask}\index{Permissions!Umask}
\input{content/chapter5/5.1.6.tex}
\subsection{Practice}\index{Understanding vi/vim editor!Practice}
\input{content/chapter5/5.1.7.tex}
%-----------------------

%----------------------------------------------------------------------------------------
%	CHAPTER 6
%----------------------------------------------------------------------------------------
\chapterimage{index7.png} % Table of contents heading image
\chapter{Advance Permissions}
%-----------------------
\section{Sudo user}\index{Sudo user}
\input{content/chapter6/6.3.tex}
\subsection{What is a sudo user?}\index{Sudo user!What is a sudo user?}
\input{content/chapter6/6.3.1.tex}
\subsection{Creating sudo users}\index{Sudo user!Creating sudo users}
\input{content/chapter6/6.3.2.tex}
\subsection{Practice}\index{Sudo user!Practice}
\input{content/chapter6/6.3.3.tex}

\section{Special permissions}\index{Special permissions}
\input{content/chapter6/6.1.tex}
\subsection{SUID}\index{Special permissions!SUID}
\setlength{\columnsep}{3pt}
\begin{flushleft}
	\paragraph{What is SUID?}
	\begin{itemize}
		\item SUID stands for \textbf{S}et \textbf{U}ser \textbf{ID}.
		\item \textbf{SUID can be applied only on command binaries}.
		\item \textbf{SUID is applicable only on user}.
		\item SUID allows user to run a \textbf{command binary} with the permissions of the \textbf{command binary owner} rather than the user who runs it.
		\item SUID is denoted as \textbf{"s"}, if user has execute permission.
		\begin{figure}[h!]
			\centering
			\includegraphics[scale=0.3]{content/chapter6/images/adv_perm1.png}
			\caption{SUID permission}
			\label{fig:combination_permission3}
		\end{figure}
		\item SUID is denoted as \textbf{"S"}, if no execute permission is applied for user.
		\begin{figure}[h!]
			\centering
			\includegraphics[scale=0.3]{content/chapter6/images/adv_perm2.png}
			\caption{SUID permission}
			\label{fig:combination_permission4}
		\end{figure}
		\item Octal representation of SUID permission is \textbf{"4"}.
	\end{itemize}
	\newpage
	\paragraph{Real world example for SUID}
	The \textbf{passwd} command has SUID applied on it.
	\bigskip
	\begin{tcolorbox}[breakable,notitle,boxrule=-0pt,colback=black,colframe=black]
		\color{green}
		\fontdimen2\font=1em
		\$ ls -ld /usr/bin/passwd
		\color{white}
		\newline
		\fontdimen2\font=0.5em
		-rwsr-xr-x 1 root root 68208 Jul 15  2021 /usr/bin/passwd
		\fontdimen2\font=4pt
	\end{tcolorbox}
	Explaination:
	\begin{itemize}
		\item The \textbf{passwd} command is used to change password.
		\item The command tries to edit files such as \textbf{/etc/passwd, /etc/shadow} etc. while changing the password.
		\item These files \textbf{can be accessed by root \& not local users}.
		\item The \textbf{SUID} permission allows local user to execute passwd command as root \& edit \textbf{/etc/passwd, /etc/shadow} files.
		\item Hence passwd command can be used by local user to change their own password.
	\end{itemize}
	
	\newpage
	
	\paragraph{How to apply SUID permission?}
	Command:
	\begin{tcolorbox}[breakable,notitle,boxrule=0pt,colback=pink,colframe=pink]
		\color{black}
		\fontdimen2\font=1em
		Syntax: chmod u+s command\_binary
		\fontdimen2\font=4pt
	\end{tcolorbox}
	
	Let's take example of \textbf{fdisk} command. 
	\newline
	The \textbf{fdisk command cannot be executed by normal user and is owned by root user}. Is there a way to allow local user execute \textbf{fdisk} commnd?
	\newline
	\textbf{Solution}: 
	\begin{itemize}
		\item Apply SUID on the \textbf{fdisk} command binary.
		\item 	SUID can be set in two ways:
		\bigskip
		\begin{tcolorbox}[breakable,notitle,boxrule=-0pt,colback=black,colframe=black]
			\color{green}
			\fontdimen2\font=1em
			\# chmod u+s  /usr/sbin/fdisk
			\newline
			or
			\newline
			\# chmod 4750 /usr/sbin/fdisk
			\newline
			\newline
			\# ls -ld /usr/sbin/fdisk
			\newline
			\color{white}
			-rwsr-xr-x 1 root root 153880 Jul 21  2020 /usr/sbin/fdisk
			\fontdimen2\font=4pt
		\end{tcolorbox}
		\item Check if \textbf{normal user jack} is able to execute \textbf{fdisk} command:
		\begin{tcolorbox}[breakable,notitle,boxrule=-0pt,colback=black,colframe=black]
			\color{green}
			\fontdimen2\font=1em
			jack@lavatech:~\$ fdisk -l
			\fontdimen2\font=4pt
		\end{tcolorbox}
	\end{itemize}
	\paragraph{How to remove SUID permission?}
	Command:
	\begin{tcolorbox}[breakable,notitle,boxrule=0pt,colback=pink,colframe=pink]
		\color{black}
		\fontdimen2\font=1em
		Syntax: chmod u-s command\_binary
		\fontdimen2\font=4pt
	\end{tcolorbox}
	Eg:
	\begin{tcolorbox}[breakable,notitle,boxrule=-0pt,colback=black,colframe=black]
		\color{green}
		\fontdimen2\font=1em
		\# chmod u-s  /usr/sbin/fdisk
		\newline
		\color{white}
		OR
		\color{green}
		\newline
		\# chmod 0750 /usr/sbin/fdisk
		\fontdimen2\font=4pt
	\end{tcolorbox}

	
\end{flushleft}

\newpage


\subsection{SGID}\index{Special permissions!SGID}
\setlength{\columnsep}{3pt}
\begin{flushleft}
	\bigskip
	\paragraph{What is SGID?}
	\begin{itemize}
		\item SGID stands for \textbf{S}et \textbf{G}roup \textbf{ID}.
		\item \textbf{SGID can be applied only on directories}.
		\item \textbf{SGID is applicable only on group ownership}.
		\item When SGID permission is set on a directory, all the new (future) files/folders created
		under that directory will have the same group owner as that of the parent
		directory.
		\item Subdirectories created in future will also have SGID bit on them.
		\item SGID is denoted as \textbf{"s"} for group, if group has execute permission:
		\begin{figure}[h!]
			\centering
			\includegraphics[scale=0.3]{content/chapter6/images/adv_perm3.png}
			\caption{SGID permission}
			\label{fig:combination_permission5}
		\end{figure}
		\item SGID is denoted as \textbf{"S"} for group, if no execute permission is applied for group:
		\begin{figure}[h!]
			\centering
			\includegraphics[scale=0.3]{content/chapter6/images/adv_perm4.png}
			\caption{SGID permission}
			\label{fig:combination_permission6}
		\end{figure}
		\item Octal representation of SGID is \textbf{"2"}.
	\end{itemize}
	\newpage	
	\paragraph{How to apply SGID permission?}
	Command:
	\begin{tcolorbox}[breakable,notitle,boxrule=0pt,colback=pink,colframe=pink]
		\color{black}
		\fontdimen2\font=1em
		Syntax: chmod g+s directory\_name
		\fontdimen2\font=4pt
	\end{tcolorbox}
	Eg: Create a folder named \textbf{"/project"} with below conditions:
	\begin{itemize}
		\item Owner: raj
		\item Group: devops
		\item SGUID: yes
	\end{itemize}
	\textbf{Solution}:
	\begin{itemize}
		\item Create the folder:
		\bigskip
		\begin{tcolorbox}[breakable,notitle,boxrule=-0pt,colback=black,colframe=black]
			\color{green}
			\fontdimen2\font=1em
			\# mkdir /project
			\fontdimen2\font=4pt
		\end{tcolorbox}
		\item Assign owner as raj and group as devops:
		\begin{tcolorbox}[breakable,notitle,boxrule=-0pt,colback=black,colframe=black]
			\color{green}
			\fontdimen2\font=1em
			\# chown raj:devops /project
			\fontdimen2\font=4pt
		\end{tcolorbox}
		\item Set SGID on /project folder:
		\begin{tcolorbox}[breakable,notitle,boxrule=-0pt,colback=black,colframe=black]
			\color{green}
			\fontdimen2\font=1em
			\# chmod g+s /project
			\newline
			\color{white}
			or
			\color{green}
			\newline
			\# chmod 2770 /project
			\fontdimen2\font=4pt
		\end{tcolorbox}
		\item To confirm the effect of SGID, switch to root user and create a file named \textbf{"/project/sample.txt"}. Confirm the group ownership is set to \textbf{"devops"}.
		\begin{figure}[h!]
			\centering
			\includegraphics[scale=0.5]{content/chapter6/images/123.png}
			\caption{Sample output}
			\label{fig:combination_permission41}
		\end{figure}
	\end{itemize}
	\newpage
	\paragraph{How to remove SGID permission?}
	\bigskip
	Command:
	\begin{tcolorbox}[breakable,notitle,boxrule=0pt,colback=pink,colframe=pink]
		\color{black}
		\fontdimen2\font=1em
		Syntax: chmod g-s directory\_name
		\fontdimen2\font=4pt
	\end{tcolorbox}
	Eg:
	\begin{tcolorbox}[breakable,notitle,boxrule=-0pt,colback=black,colframe=black]
		\color{green}
		\fontdimen2\font=1em
		\# chmod g-s  /tmp/test
		\newline
		or
		\newline
		\# chmod 0750 /tmp/test
		\fontdimen2\font=4pt
	\end{tcolorbox}
	
	
	
	
	
\end{flushleft}

\newpage


\subsection{Sticky bit}\index{Special permissions!Sticky bit}
\setlength{\columnsep}{3pt}
\begin{flushleft}
	\bigskip
	\begin{itemize}
		\item Sticky bit can be applied only on directories.
		\item Content of directory having sticky bit on it \textbf{can be only deleted by root or the user who created that file.}
		\item Sticky bit is applicable only on \textbf{other users.} 
		\item Sticky bit is denoted as \textbf{'t'} for other, if \textbf{other} has execute permission.
		\begin{figure}[h!]
			\centering
			\includegraphics[scale=0.4]{content/chapter6/images/22.png}
			\caption{Sticky bit with execute permission}
			\label{fig:acl_example_9}
		\end{figure}
		\item Sticky bit is denoted as \textbf{'T'} for other, if \textbf{other} has no execute permission.
		\begin{figure}[h!]
			\centering
			\includegraphics[scale=0.4]{content/chapter6/images/23.png}
			\caption{Sticky bit without execute permission}
			\label{fig:acl_example_8}
		\end{figure}
		
		\item Eg: \textbf{/tmp} directory is having sticky bit permission on it, so that only root or the owner of the files under /tmp can delete it's content.
		/tmp/.
		\item Octal representation of sticky bit is \textbf{1}.	
	\end{itemize}
\newpage

\paragraph{How to apply sticky bit permission?}
\bigskip
Command:
\begin{tcolorbox}[breakable,notitle,boxrule=0pt,colback=pink,colframe=pink]
	\color{black}
	\fontdimen2\font=1em
	Syntax: chmod o+t directory\_name
	\fontdimen2\font=4pt
\end{tcolorbox}
Eg: Create a folder named \textbf{/opt/dump} and provide it with permission \textbf{757} and \textbf{sticky bit}.
\begin{tcolorbox}[breakable,notitle,boxrule=-0pt,colback=black,colframe=black]
	\color{green}
	\fontdimen2\font=1em
	\# mkdir /opt/dump/
	\newline
	\$ chmod o+t,u=rwx,g=r-x,o+rwx /opt/dump/
	\newline
	or
	\newline
	\$ chmod 1757 /opt/dump/
	\fontdimen2\font=4pt
\end{tcolorbox}
As local user named "raj", \textbf{create a file named /opt/dump/sample.txt}.
\begin{tcolorbox}[breakable,notitle,boxrule=-0pt,colback=black,colframe=black]
	\color{green}
	\fontdimen2\font=1em
	\# su - raj
	\newline
	\$ touch /opt/dump/sample.txt
	\fontdimen2\font=4pt
\end{tcolorbox}

As local user "ravi", delete the file \textbf{/opt/dump/sample.txt} and confirm below error:
\begin{tcolorbox}[breakable,notitle,boxrule=-0pt,colback=black,colframe=black]
	\color{green}
	\fontdimen2\font=1em
	\# su - ravi
	\newline
	\$ rm /opt/dump/sample.txt
	\color{white}
	\newline
	rm: cannot remove '/opt/test/sample.txt': Operation not permitted
	\fontdimen2\font=4pt
\end{tcolorbox}

\newpage
\paragraph{How to remove sticky bit permission?}
\bigskip
Command:
\begin{tcolorbox}[breakable,notitle,boxrule=0pt,colback=pink,colframe=pink]
	\color{black}
	\fontdimen2\font=1em
	Syntax: chmod o-t directory\_name
	\fontdimen2\font=4pt
\end{tcolorbox}
Eg:
\begin{tcolorbox}[breakable,notitle,boxrule=-0pt,colback=black,colframe=black]
	\color{green}
	\fontdimen2\font=1em
	\$ chmod o-t /opt/dump/
	\newline
	or
	\newline
	\$ chmod 0757 /opt/dump/
	\fontdimen2\font=4pt
\end{tcolorbox}
	
	
\end{flushleft}

\newpage


\subsection{Practice}\index{Special permissions!Practice}
\setlength{\columnsep}{3pt}
\begin{flushleft}

	\paragraph{}
	\bigskip
	
	\begin{figure}[h!]
		\centering
		\includegraphics[scale=.2]{content/practise.jpg}
	\end{figure}	
	\begin{enumerate}
		\item \textbf{Select the correct octal representation for: 
		\begin{itemize}
			\item user: read,write,execute
			\item group: write,execute
			\item other: no permission
			\item SUID for user
		\end{itemize}}
		
		\begin{enumerate}[label=(\alph*)]
			\item 4730 %correct
			\item 2730
			\item 1730
			\item 0730  
		\end{enumerate}
		\bigskip
		\bigskip	
		
		\item \textbf{Select all statement true about SUID permission.}
		\begin{enumerate}[label=(\alph*)]
			\item SUID is used to provide same user ownership to all files in folder.
			\item SUID allows local users to run binaries with the permissions of the binary owner. %correct
			\item SUID provides read, write and execute permission to all users.
			\item SUID can be applied on command binaries. %correct
		\end{enumerate}
		\bigskip
		\bigskip
		\newpage
		\item \textbf{Which of the following commands have SUID applied on themselves by default? (Select all that applies.)}
		\begin{enumerate}[label=(\alph*)]
			\item passwd    %correct
			\item umount    %correct
			\item su       %correct
			\item useradd
		\end{enumerate}
		\bigskip
		\bigskip
		\item \textbf{Which of the following permissions can be applied on directories? (Select all that applies.)}
		\begin{enumerate}[label=(\alph*)]
			\item SUID
			\item SGID                 %correct
			\item Sticky bit          %correct
			\item None of above
		\end{enumerate}	
		\bigskip
		\bigskip
		\item \textbf{Select the correct octal representation for SUID, SGID and sticky bit respectively.}
		\begin{enumerate}[label=(\alph*)]
			\item 4,2,1    %correct
			\item 1,2,3
			\item 1,2,4
			\item 2,1,3
		\end{enumerate}
		\bigskip
		\bigskip
		\item \textbf{Which of the following directory have sticky bit applied on it by default?}
		\begin{enumerate}[label=(\alph*)]
			\item /var
			\item /mnt   
			\item /tmp     %correct
			\item /run
		\end{enumerate}
		\bigskip
		\bigskip
		\item \textbf{State whether true or false. SGID can be applied on files and directories.}
		\begin{enumerate}[label=(\alph*)]
			\item True
			\item False  %correct
		\end{enumerate}
		\bigskip
		\bigskip
		\item \textbf{State whether true or false. Sticky bit can be applied only on directories.}
		\begin{enumerate}[label=(\alph*)]
			\item True %correct
			\item False
		\end{enumerate}
	\end{enumerate}

	
\end{flushleft}

\newpage



\section{Access Control List(ACL)}\index{Access control list(ACL)}
\input{content/chapter6/6.2.tex}
\subsection{What is ACL?}\index{Access control list(ACL)!What is ACL?}
\input{content/chapter6/6.2.1.tex}
\subsection{ACL commands}\index{Access control list(ACL)!ACL commands}
\input{content/chapter6/6.2.2.tex}
\subsection{Practice}\index{Access control list(ACL)!Practice}
\input{content/chapter6/6.2.3.tex}

%-----------------------

%----------------------------------------------------------------------------------------
%	CHAPTER 7
%----------------------------------------------------------------------------------------
\chapterimage{index8.png} % Table of contents heading image
\chapter{Processing Commands}
%%%%%-----------------------

\section{Text processing}\index{Text processing}
\input{content/chapter7/7.2.tex}
\subsection{Commands}\index{Text processing!Commands}
\setlength{\columnsep}{3pt}
\begin{flushleft}
	
	\begin{itemize}
		\item \textbf{locate}: Searches for file names or file paths and returns the results
		instantly.
		\bigskip
		\begin{tcolorbox}[breakable,notitle,boxrule=-0pt,colback=pink,colframe=pink]
			\color{black}
			\fontdimen2\font=1em
			Syntax: locate filename/foldername
			\fontdimen2\font=4pt
		\end{tcolorbox}
		Eg:
		\begin{tcolorbox}[breakable,notitle,boxrule=-0pt,colback=black,colframe=black]
			\color{green}
			\fontdimen2\font=1em
			\# locate passwd
			\fontdimen2\font=4pt
		\end{tcolorbox}		
		\bigskip
		\bigskip
		\item \textbf{find}: Searches the directory tree containing a specific file.
		\bigskip
		\begin{tcolorbox}[breakable,notitle,boxrule=-0pt,colback=pink,colframe=pink]
			\color{black}
			\fontdimen2\font=1em
			Syntax: find directory\_name options expression
			\fontdimen2\font=4pt
		\end{tcolorbox}
		Options with \textbf{find} command:
		\bigskip
		\begin{itemize}
			\item \textbf{-name}: Supply an expression to search the directory for.
			\bigskip
			\begin{tcolorbox}[breakable,notitle,boxrule=-0pt,colback=pink,colframe=pink]
				\color{black}
				\fontdimen2\font=1em
				Syntax: find directory\_name -name expression
				\fontdimen2\font=4pt
			\end{tcolorbox}
			Eg: Search for files having name "sample\_file" in directory \textbf{/root}:
			\bigskip
			\begin{tcolorbox}[breakable,notitle,boxrule=-0pt,colback=black,colframe=black]
				\color{green}
				\fontdimen2\font=1em
				\# find /root -name "sample\_file"
				\fontdimen2\font=4pt
			\end{tcolorbox}		
			Special characters in \textbf{-name} option:
			\begin{itemize}
				\item \textbf{"?" in expression}: The question-mark is a wild card which represents 'any one character'.
				\bigskip
				\begin{tcolorbox}[breakable,notitle,boxrule=-0pt,colback=black,colframe=black]
					\color{green}
					\fontdimen2\font=1em
					\# find /root -name "sample?"
					\fontdimen2\font=4pt
				\end{tcolorbox}		
				\item \textbf{"*" in expression}: Represent any number of multiple characters.
				\bigskip
				\begin{tcolorbox}[breakable,notitle,boxrule=-0pt,colback=black,colframe=black]
					\color{green}
					\fontdimen2\font=1em
					\# find /root -name "sample*"
					\fontdimen2\font=4pt
				\end{tcolorbox}		
			\end{itemize}
			\bigskip
			\bigskip
			\item \textbf{-iname}: Perform a case-insensitive search for a given file name.
			\bigskip
			\begin{tcolorbox}[breakable,notitle,boxrule=-0pt,colback=pink,colframe=pink]
				\color{black}
				\fontdimen2\font=1em
				Syntax: find directory\_name -iname expression
				\fontdimen2\font=4pt
			\end{tcolorbox}
			\bigskip
			\bigskip
			
			\item \textbf{-user}: Find files owned by a specific user.
			\bigskip
			\begin{tcolorbox}[breakable,notitle,boxrule=-0pt,colback=pink,colframe=pink]
				\color{black}
				\fontdimen2\font=1em
				Syntax: find directory\_name -user user\_name
				\fontdimen2\font=4pt
			\end{tcolorbox}
			\bigskip
			Eg: Find all files/directories owned by user \textbf{"root" in /var directory}.
			\bigskip
			\begin{tcolorbox}[breakable,notitle,boxrule=-0pt,colback=black,colframe=black]
				\color{green}
				\fontdimen2\font=1em
				\# find /var -user root
				\fontdimen2\font=4pt
			\end{tcolorbox}		

			\bigskip
			\bigskip		
			\item \textbf{-group}: Find files owned by a specific group.
			\bigskip
			\begin{tcolorbox}[breakable,notitle,boxrule=-0pt,colback=pink,colframe=pink]
				\color{black}
				\fontdimen2\font=1em
				Syntax: find directory\_name -group group\_name
				\fontdimen2\font=4pt
			\end{tcolorbox}
			\bigskip
			Eg: Find all files/directories owned by group \textbf{"root" in /var directory}.
			\bigskip
			\begin{tcolorbox}[breakable,notitle,boxrule=-0pt,colback=black,colframe=black]
				\color{green}
				\fontdimen2\font=1em
				\# find /var -group root
				\fontdimen2\font=4pt
			\end{tcolorbox}		
			
			\bigskip
			\bigskip
			
			\item \textbf{-type}: Find files of specific type like \textbf{file, directory, symbolic link} etc.
			\bigskip
			\begin{tcolorbox}[breakable,notitle,boxrule=-0pt,colback=pink,colframe=pink]
				\color{black}
				\fontdimen2\font=1em
				Syntax: find directory\_name -type [f,d,l,s,c,b]
				\fontdimen2\font=4pt
			\end{tcolorbox}
			\bigskip
			Eg: Find all directories under \textbf{/tmp} directory.
			\bigskip
			\begin{tcolorbox}[breakable,notitle,boxrule=-0pt,colback=black,colframe=black]
				\color{green}
				\fontdimen2\font=1em
				\# find /tmp -type d
				\fontdimen2\font=4pt
			\end{tcolorbox}		
			
			\bigskip
			\bigskip
			
			\item \textbf{-atime}: Find files according to their access time.
			\bigskip
			\begin{tcolorbox}[breakable,notitle,boxrule=-0pt,colback=pink,colframe=pink]
				\color{black}
				\fontdimen2\font=1em
				Syntax: find directory\_name -atime [argument]
				\fontdimen2\font=4pt
			\end{tcolorbox}
			\bigskip
			Eg:
			\bigskip
			\begin{tcolorbox}[breakable,notitle,boxrule=-0pt,colback=black,colframe=black]
				\color{yellow}
				\fontdimen2\font=1em
				\# Find files whose access time 3 days hours ago.
				\color{green}
				\newline
				\$ find /tmp –atime 3 
				\newline
				\newline				
				\color{yellow}
				\# Find files whose access time 3 days hours ago and prior to that.
				\color{green}
				\newline
				\$ find /tmp –atime +3 
				\newline
				\color{yellow}
				\newline
				\# Find files whose access time between now and upto 3 days ago.
				\color{green}
				\newline
				\$ find /tmp –atime -3 
				\fontdimen2\font=4pt
			\end{tcolorbox}		


			\bigskip
			\bigskip
			
			\item \textbf{-perm}: Find files according to specific permission.
			\bigskip
			\begin{tcolorbox}[breakable,notitle,boxrule=-0pt,colback=pink,colframe=pink]
				\color{black}
				\fontdimen2\font=1em
				Syntax: find directory\_name -perm argument
				\fontdimen2\font=4pt
			\end{tcolorbox}
			Eg:
			\bigskip
			\begin{tcolorbox}[breakable,notitle,boxrule=-0pt,colback=black,colframe=black]
				\color{white}
				\fontdimen2\font=1em
				\color{yellow}
				\# Find in the current directory the files having exact permissions of 644.
				\color{green}
				\newline
				\$ find . –perm 644
				\newline
				\color{yellow}
				\newline
				\# Find in the current directory the files having \color{yellow} either rw to user OR r to group OR r to \color{yellow} others. Any one permission match will do.
				\color{green}
				\newline
				\$ find . –perm /644 
				\newline
				\color{yellow}
				\newline
				\# Find in the current directory the files having minimum 664 permissions.
				\color{green}
				\newline
				\$ find . –perm -664
				\fontdimen2\font=4pt
			\end{tcolorbox}		
		\end{itemize}
		
		\newpage
		\item \textbf{grep}: Stands for "\textbf{G}lobal \textbf{R}egular \textbf{E}xpression \textbf{P}rint". Searches the file for lines containing a match to the given expression.
		\newline
			\begin{tcolorbox}[breakable,notitle,boxrule=-0pt,colback=pink,colframe=pink]
				\color{black}
				\fontdimen2\font=1em
				Syntax: grep expression file\_name
				\fontdimen2\font=4pt
			\end{tcolorbox}
		Eg: Search word "shakher" in file \textbf{/etc/passwd}:
			\bigskip
			\begin{tcolorbox}[breakable,notitle,boxrule=-0pt,colback=black,colframe=black]
				\color{green}
				\fontdimen2\font=1em
				\# grep shakher /etc/passwd
				\newline
				\color{white}
				shakher:x:1000:1000::/home/shakher:/bin/bash
				\newline
				shakher\_suman:x:1000:1000::/home/shakher\_suman:/bin/bash
				\fontdimen2\font=4pt
			\end{tcolorbox}
		Options with \textbf{grep} command:
		\begin{itemize}
			\item \textbf{-i}: Force grep to ignore word case.
			\bigskip
			
			\begin{tcolorbox}[breakable,notitle,boxrule=-0pt,colback=black,colframe=black]
				\color{green}
				\fontdimen2\font=1em
				\# grep -i shakher /etc/passwd
				\newline
				\color{white}
				shakher:x:1000:1000::/home/shakher:/bin/bash
				\newline
				shakher\_suman:x:1000:1000::/home/shakher\_suman:/bin/bash
				\fontdimen2\font=4pt
			\end{tcolorbox}
			
			\item \textbf{-r} or \textbf{-R}: Search recursively i.e. search all files under each directory for a string.
			\bigskip
			\begin{tcolorbox}[breakable,notitle,boxrule=-0pt,colback=black,colframe=black]
				\color{green}
				\fontdimen2\font=1em
				\# grep -r "192.168.1.5" /etc/
				\newline
				\color{white}
				/etc/ppp/options:\# ms-wins 192.168.1.50
				\newline
				/etc/ppp/options:\# ms-wins 192.168.1.51
				\fontdimen2\font=4pt
			\end{tcolorbox}
			
			\item \textbf{-w}: Select only those lines containing matches that form whole words.
			\bigskip
			\begin{tcolorbox}[breakable,notitle,boxrule=-0pt,colback=black,colframe=black]
				\color{green}
				\fontdimen2\font=1em
				\# grep -w "shakher" /etc/passwd
				\newline
				\color{white}
				shakher:x:1000:1000::/home/shakher:/bin/bash
				\fontdimen2\font=4pt
			\end{tcolorbox}

			\item \textbf{-c}: Count a particular word in file.
			\bigskip
			\begin{tcolorbox}[breakable,notitle,boxrule=-0pt,colback=black,colframe=black]
				\color{green}
				\fontdimen2\font=1em
				\# grep -c "Error" logfile.txt
				\newline
				\color{white}
				4
				\fontdimen2\font=4pt
			\end{tcolorbox}
			
			\item \textbf{-n}: Display line number of lines matching the word.
			\bigskip
			\begin{tcolorbox}[breakable,notitle,boxrule=-0pt,colback=black,colframe=black]
				\color{green}
				\fontdimen2\font=1em
				\# grep -n "Error" logfile
				\fontdimen2\font=4pt
			\end{tcolorbox}
			Eg:
			\begin{figure}[h!]
				\centering
				\includegraphics[scale=.4]{content/chapter7/images/grep1.png}
				\caption{Sample output}
				\label{fig:path23}
			\end{figure}

			\item \textbf{-v}: Display only those lines that \textbf{do not} contain the given word.
			\bigskip
			\begin{tcolorbox}[breakable,notitle,boxrule=-0pt,colback=black,colframe=black]
				\color{green}
				\fontdimen2\font=1em
				\# grep -v "nep" logfile
				\fontdimen2\font=4pt
			\end{tcolorbox}		

			\item \textbf{-l}: Displays only the file names which matches the given pattern.
			\begin{tcolorbox}[breakable,notitle,boxrule=-0pt,colback=black,colframe=black]
				\color{green}
				\fontdimen2\font=1em
				\# grep -l "main" *.java
				\fontdimen2\font=4pt
			\end{tcolorbox}					
		\end{itemize}
			Special characters in {grep} command:
			\begin{itemize}
				\item \textbf{\^} : Matches only the lines having the starting word mentioned in the expression.
				\bigskip
				\begin{tcolorbox}[breakable,notitle,boxrule=-0pt,colback=black,colframe=black]
					\color{green}
					\fontdimen2\font=1em
					\# grep -v "\textbf{\^}UUID" /etc/fstab
					\fontdimen2\font=4pt
				\end{tcolorbox}			
				\item \textbf{\$} : Matches only the lines having the last word mentioned in the expression.
				\bigskip
				\begin{tcolorbox}[breakable,notitle,boxrule=-0pt,colback=black,colframe=black]
					\color{green}
					\fontdimen2\font=1em
					\# grep -v "UUID\textbf{\$}" /etc/fstab
					\fontdimen2\font=4pt
				\end{tcolorbox}									
			\end{itemize}
		
		\newpage
				
		\item \textbf{head}: Display starting lines of the file. By default, displays the top 10 lines of file.
		\bigskip
		\begin{tcolorbox}[breakable,notitle,boxrule=-0pt,colback=pink,colframe=pink]
			\color{black}
			\fontdimen2\font=1em
			Syntax: head file\_name
			\fontdimen2\font=4pt
		\end{tcolorbox}
		Eg:
		\bigskip
		\begin{tcolorbox}[breakable,notitle,boxrule=-0pt,colback=black,colframe=black]
			\color{green}
			\fontdimen2\font=1em
			\# head /etc/passwd
			\fontdimen2\font=4pt
		\end{tcolorbox}		
		Options for \textbf{head} command:
		\newline
		\textbf{-n}: Provide the number of lines to be displayed from the start of the file.
		\bigskip
		\begin{tcolorbox}[breakable,notitle,boxrule=-0pt,colback=black,colframe=black]
			\color{green}
			\fontdimen2\font=1em
			\# head -n5 /etc/passwd
			\fontdimen2\font=4pt
		\end{tcolorbox}		

		\bigskip
		\bigskip
		\item \textbf{tail}: Display ending lines of the file. By default, displays the bottom 10 lines of file.
		\bigskip
		\begin{tcolorbox}[breakable,notitle,boxrule=-0pt,colback=pink,colframe=pink]
			\color{black}
			\fontdimen2\font=1em
			Syntax: tail file\_name
			\fontdimen2\font=4pt
		\end{tcolorbox}
		Eg:
		\bigskip
		\begin{tcolorbox}[breakable,notitle,boxrule=-0pt,colback=black,colframe=black]
			\color{green}
			\fontdimen2\font=1em
			\# tail /etc/passwd
			\fontdimen2\font=4pt
		\end{tcolorbox}		
		Options for \textbf{tail} command:
		\newline
		\textbf{-n}: Provide the number of lines to be displayed from the end of the file.
		\bigskip
		\begin{tcolorbox}[breakable,notitle,boxrule=-0pt,colback=black,colframe=black]
			\color{green}
			\fontdimen2\font=1em
			\# tail -n5 /etc/passwd
			\fontdimen2\font=4pt
		\end{tcolorbox}		
		
		\bigskip
		\bigskip

		\item \textbf{more}: Easily read a file without using an editor. \textbf{Press "q"} to quit reading the file.
		\bigskip
		\begin{tcolorbox}[breakable,notitle,boxrule=-0pt,colback=pink,colframe=pink]
			\color{black}
			\fontdimen2\font=1em
			Syntax: more file\_name
			\fontdimen2\font=4pt
		\end{tcolorbox}
		Eg:
		\bigskip
		\begin{tcolorbox}[breakable,notitle,boxrule=-0pt,colback=black,colframe=black]
			\color{green}
			\fontdimen2\font=1em
			\# more /etc/passwd
			\fontdimen2\font=4pt
		\end{tcolorbox}		


		\bigskip
		\bigskip
		\item \textbf{wc}: Display number of \textbf{lines, words and bytes} respectively in a file.
		\bigskip
		\begin{tcolorbox}[breakable,notitle,boxrule=-0pt,colback=pink,colframe=pink]
			\color{black}
			\fontdimen2\font=1em
			Syntax: wc filename
			\fontdimen2\font=4pt
		\end{tcolorbox}
		Eg:
		\bigskip
		\begin{tcolorbox}[breakable,notitle,boxrule=-0pt,colback=black,colframe=black]
			\color{green}
			\fontdimen2\font=1em
			\$ wc myfile
			\newline
			\color{white}
			6 7 39 myfile
			\fontdimen2\font=4pt
		\end{tcolorbox}	

		Options with \textbf{wc} command:
		\begin{itemize}
			\item \textbf{-l}: Display number of lines in a file
			\item \textbf{-w}: Display number of words in a file
			\item \textbf{-c}: Display number of characters in a file
			\newline
			Eg:
			\begin{tcolorbox}[breakable,notitle,boxrule=-0pt,colback=black,colframe=black]
				\color{green}
				\fontdimen2\font=1em
				\# wc -l myfile
				\newline
				\# wc -w myfile
				\newline
				\# wc -c myfile
				\fontdimen2\font=4pt
			\end{tcolorbox}	
		\end{itemize}

		\bigskip
		\bigskip

		\item \textbf{sort}: Sorts file content by default in ascending order.
		\bigskip
		\begin{tcolorbox}[breakable,notitle,boxrule=-0pt,colback=pink,colframe=pink]
			\color{black}
			\fontdimen2\font=1em
			Syntax: sort file\_name
			\fontdimen2\font=4pt
		\end{tcolorbox}
		Eg: Notice the file content \textbf{"phonebook"}:
		\bigskip
		\begin{tcolorbox}[breakable,notitle,boxrule=-0pt,colback=black,colframe=black]
			\color{green}
			\fontdimen2\font=1em
			\$ cat phonebook
			\color{white}
			\newline
			Smith,Brett 5554321
			\newline
			Doe,John 5551234
			\newline
			Doe,Jane 5553214
			\newline
			Avery,Cory 5554321
			\newline
			Fogarty,Suzie 5552314
			\fontdimen2\font=4pt
		\end{tcolorbox}		
		Let's sort the content of \textbf{"phonebook"} file.
		\bigskip
		\begin{tcolorbox}[breakable,notitle,boxrule=-0pt,colback=black,colframe=black]
			\color{green}
			\fontdimen2\font=1em
			\$ sort phonebook
			\color{white}
			\newline
			Avery,Cory 5554321
			\newline
			\color{white}
			Doe,Jane 5553214
			\newline
			\color{white}
			Doe,John 5551234
			\newline
			Fogarty,Suzie 5552314
			\newline
			Smith,Brett 5554321
			\fontdimen2\font=4pt
		\end{tcolorbox}		
		Options with \textbf{sort} command:
		\begin{itemize}
			\item \textbf{-r}: Sort file in reverse order.
			\bigskip
			\begin{tcolorbox}[breakable,notitle,boxrule=-0pt,colback=black,colframe=black]
				\color{green}
				\fontdimen2\font=1em
				\$ sort -r phonebook
				\fontdimen2\font=4pt
			\end{tcolorbox}		
			%\item \textbf{-n}: Makes the program to sort according to numerical value.
		\end{itemize}

		\bigskip
		\bigskip

		\item \textbf{uniq}: Display file content by removing consecutive duplicate lines from the file.
		\bigskip
		\begin{tcolorbox}[breakable,notitle,boxrule=-0pt,colback=pink,colframe=pink]
			\color{black}
			\fontdimen2\font=1em
			Syntax: uniq file\_name
			\fontdimen2\font=4pt
		\end{tcolorbox}
		Eg: Find all unique lines in below file:
		\bigskip
		\begin{tcolorbox}[breakable,notitle,boxrule=-0pt,colback=black,colframe=black]
			\color{green}
			\fontdimen2\font=1em
			\# cat cities.txt
			\newline
			\color{white}
			Pune
			\newline
			Kolhapur
			\newline
			Kolhapur
			\newline
			Pune
			\fontdimen2\font=4pt
		\end{tcolorbox}		
		The \textbf{uniq} command is used find uniq consecutive lines:
		\begin{tcolorbox}[breakable,notitle,boxrule=-0pt,colback=black,colframe=black]
			\color{green}
			\fontdimen2\font=1em
			\# uniq cities.txt
			\newline
			\color{white}
			Pune
			\newline
			Kolhapur
			\newline
			Pune
			\fontdimen2\font=4pt
		\end{tcolorbox}		
		\bigskip
		\bigskip

		\item \textbf{cut}: Extract a certain range of characters from a line or file.
		\newline
		Options with \textbf{cut} command:
		\begin{itemize}
		\item  \textbf{-c}: Display specific character or range of characters.
		\bigskip
		\begin{tcolorbox}[breakable,notitle,boxrule=-0pt,colback=pink,colframe=pink]
			\color{black}
			\fontdimen2\font=1em
			Syntax: cut -cn  file\_name
			\fontdimen2\font=4pt
		\end{tcolorbox}
		Eg: Consider below file:
		\bigskip
		\begin{tcolorbox}[breakable,notitle,boxrule=-0pt,colback=black,colframe=black]
			\color{green}
			\fontdimen2\font=1em
			\# cat company.data
			\newline
			\color{white}
			406378:Sales:Itorre:Jan
			\newline
			031762:Marketing:Nasium:Jim
			\newline
			636496:Research:Ancholie:Mel
			\newline
			396082:Sales:Jucacion:Ed
			\fontdimen2\font=4pt
		\end{tcolorbox}		
		\bigskip
		To display the 6th characters of the file:
		\bigskip
		\begin{tcolorbox}[breakable,notitle,boxrule=-0pt,colback=black,colframe=black]
			\color{green}
			\fontdimen2\font=1em
			\# cut -c6 company.data
			\newline
			\color{white}
			8
			\newline
			2
			\newline
			6
			\newline
			2
			\fontdimen2\font=4pt
		\end{tcolorbox}		
		\bigskip
		Display the range of characters like 2nd to 6th character of the file:
		\bigskip
		\begin{tcolorbox}[breakable,notitle,boxrule=-0pt,colback=black,colframe=black]
			\color{green}
			\fontdimen2\font=1em
			\# cut -c2-6 company.data
			\newline
			\color{white}
			06378
			\newline
			31762
			\newline
			36496
			\newline
			96082
			\fontdimen2\font=4pt
		\end{tcolorbox}		
		\bigskip

		Display only 2nd \& 6th character of the file:
		\bigskip
		\begin{tcolorbox}[breakable,notitle,boxrule=-0pt,colback=black,colframe=black]
			\color{green}
			\fontdimen2\font=1em
			\# cut -c2,6 company.data
			\newline
			\color{white}
			08
			\newline
			32
			\newline
			36
			\newline
			92
			\fontdimen2\font=4pt
		\end{tcolorbox}		
		\item \textbf{-f}: Specifies a field list, the line being cut is supposed to be comprised of fields
		\newline
		\textbf{-d}: The ‘fields’ in the line are determined by delimiter specified by ‘–d ’ option
		\bigskip
		\begin{tcolorbox}[breakable,notitle,boxrule=-0pt,colback=pink,colframe=pink]
			\color{black}
			\fontdimen2\font=1em
			Syntax: cut -fn -d[delimiter]  file\_name
			\fontdimen2\font=4pt
		\end{tcolorbox}
		Eg: \textbf{Cut} 3rd field as per delimiter ":" -
		\bigskip
		\begin{tcolorbox}[breakable,notitle,boxrule=-0pt,colback=black,colframe=black]
			\color{green}
			\fontdimen2\font=1em
			\# cut -d":" -f3 company.data
			\newline
			\color{white}
			Itorre
			\newline
			\color{white}
			Nasium
			\newline
			\color{white}
			Ancholie
			\newline
			\color{white}
			Jucacion
			\fontdimen2\font=4pt
		\end{tcolorbox}		
		Eg: If you want to access multiple fields like 1st and 3rd field, you can separate them by comma:
		\bigskip
		\begin{tcolorbox}[breakable,notitle,boxrule=-0pt,colback=black,colframe=black]
			\color{green}
			\fontdimen2\font=1em
			\# cut -d":" -f1,3 company.data
			\newline
			\color{white}
			406378:Itorre
			\newline
			031762:Nasium
			\newline
			636496:Ancholie
			\newline
			396082:Jucacion
			\fontdimen2\font=4pt
		\end{tcolorbox}				 
	\end{itemize}

		
	\end{itemize}

\end{flushleft}

\newpage


\subsection{Practice}\index{Text processing!Practice}
\setlength{\columnsep}{3pt}
\begin{flushleft}
	
	\paragraph{}
	\bigskip
	
	\begin{figure}[h!]
		\centering
		\includegraphics[scale=.2]{content/practise.jpg}
	\end{figure}	
	\begin{enumerate}
		\item \textbf{Which of the following command is used to search a file in a directory?}
		\begin{enumerate}[label=(\alph*)]
			\item grep
			\item sort
			\item cut
			\item find    %correct
 		\end{enumerate}
		\bigskip
		\bigskip	
		
		\item \textbf{Which of the following command is used to search an expression in a file?}
		\begin{enumerate}[label=(\alph*)]
			\item find
			\item locate
			\item grep   %correct
			\item sort          
		\end{enumerate}
		\bigskip
		\bigskip
		
		
		\item \textbf{Which of the following are valid options of find command? (Select all that applies.)}
		\begin{enumerate}[label=(\alph*)]
			\item -user            %correct
			\item -name            %correct
			\item -type           %correct
			\item -atime          %correct
		\end{enumerate}
		\bigskip
		\bigskip
		
		\item \textbf{Which of the following is valid options for grep command? (Select all that applies.)}
		\begin{enumerate}[label=(\alph*)]
			\item \textbf{-i}   %corrrect
  			\item \textbf{-r} or \textbf{-R}   %correct
			\item \textbf{-v}   %correct
			\item \textbf{-n}   %correct
		\end{enumerate}
		\bigskip
		\bigskip
		
		\item \textbf{Which of the following command is used to search all lines in \textbf{/etc/fstab} not having letter \textbf{"\#"}?}
		\begin{enumerate}[label=(\alph*)]
			\item grep -v "\#" /etc/pfstab  %correct
			\item grep -w "\#" /etc/pfstab  
			\item grep -n "\#" /etc/pfstab  
			\item grep -i "\#" /etc/pfstab  
		\end{enumerate}
		\bigskip
		\bigskip
		
		\item \textbf{Which of the following command is used to display first 20 lines of \textbf{/etc/passwd} file?}
		\begin{enumerate}[label=(\alph*)]
			\item head -2 /etc/passwd
			\item tail -20 /etc/passwd
			\item head -20 /etc/passwd   %correct
			\item tail -2 /etc/passwd
		\end{enumerate}
	
		\bigskip
		\bigskip
		\item \textbf{Which of the following command is used to sort all lines of \textbf{/etc/passwd} file in descending order?}
		\begin{enumerate}[label=(\alph*)]
			\item sort -d /etc/passwd
			\item sort -r /etc/passwd   %correct
			\item sort -n /etc/passwd
			\item sort -R /etc/passwd
		\end{enumerate}
	
		\bigskip
		\bigskip
		\newpage
		\item \textbf{Which of the following command is used to count number of words in a file?}
		\begin{enumerate}[label=(\alph*)]
			\item sort
			\item wc   %correct
			\item cut
			\item tail
		\end{enumerate}
	
		\bigskip
		\bigskip
		\item \textbf{Which of the following command is valid to display only UID of all users from \textbf{/etc/passwd} file?}
		\begin{enumerate}[label=(\alph*)]
			\item cut -d":" -f3 /etc/passwd  %correct
 			\item cut -d"." -f2 /etc/passwd
			\item cut -d":" -f1 /etc/passwd
			\item cut -d":" -f4 /etc/passwd
		\end{enumerate}
	\end{enumerate}
	
	
\end{flushleft}

\newpage



\section{I/O Redirection}\index{I/ORedirection}
\input{content/chapter7/7.1.tex}
\subsection{Standard input and standard output device}\index{I/ORedirection!What are standard input and standard output?}
\setlength{\columnsep}{3pt}
\begin{flushleft}
	\bigskip
	\begin{itemize}
		\item Commands read input using standard input device and produce output on standard output device.
		\item Standard input device: \textbf{Keyboard}
		\item Standard output device: \textbf{Screen} or \textbf{Desktop} or \textbf{Terminal window}
		\begin{figure}[h!]
			\centering
			\includegraphics[scale=.5]{content/chapter7/images/std_in_out.png}
			\caption{Standard input and output device}
			\label{fig:path}
		\end{figure}
	\end{itemize}
	
\end{flushleft}

\newpage


\subsection{Output redirection ( > )}
\input{content/chapter7/7.1.2.tex}
\subsection{Output append operator ( \textbf{{$\textgreater$}}\textbf{{$\textgreater$}} )}
\input{content/chapter7/7.1.3.tex}
\subsection{Input Redirection ( < )}
\input{content/chapter7/7.1.4.tex}
\subsection{Error redirection ( 2> )}
\input{content/chapter7/7.1.5.tex}
\subsection{Error append redirection ( 2\textbf{{$\textgreater$}}\textbf{{$\textgreater$}} )}
\input{content/chapter7/7.1.6.tex}
\subsection{Output \& error redirection ( \&\textbf{{$\textgreater$}} )}
\input{content/chapter7/7.1.10.tex}
\subsection{Output \& error append redirection (\&\textbf{{$\textgreater$}}\textbf{{$\textgreater$}} )}
\input{content/chapter7/7.1.11.tex}
\subsection{Redirection Summary}
\input{content/chapter7/7.1.7.tex}
\subsection{The pipe operator}
\input{content/chapter7/7.1.8.tex}
\subsection{Practice}
\input{content/chapter7/7.1.9.tex}
\section{Archives \& Compression}
\input{content/chapter7/7.3.tex}
\subsection{Commands}
\input{content/chapter7/7.3.1.tex}
\subsection{Practice}
\input{content/chapter7/7.3.2.tex}
%%%%
%%%%%-----------------------
%%%%
%%%%
%%%%%----------------------------------------------------------------------------------------
%%%%%	CHAPTER 8
%%%%%----------------------------------------------------------------------------------------
\chapterimage{index9.png} % Table of contents heading image
\chapter{Partition}
%%%%%-----------------------
\section{Introduction to partition}
\input{content/chapter8/8.1.tex}
\subsection{What is \textbf{H}ard \textbf{D}isk \textbf{D}rive (HDD) \& partition?}
\setlength{\columnsep}{3pt}
\begin{flushleft}
	\bigskip
	\begin{itemize}
		\item A \textbf{hard disk drive (HDD) or hard drive}, is a data storage device that stores and retrieves digital data in computer.
		\item HDD brands in market: Seagate, Western Digital, Hitachi etc.
			\begin{figure}[h!]
				\centering
				\includegraphics[scale=.3]{content/chapter8/images/hdd.png}
				\caption{Hard disk}
				\label{hard_disk}
			\end{figure}
		\item \textbf{Partitioning} means to divide a HDD into many logical drives.
		\begin{figure}[h!]
			\centering
			\includegraphics[scale=.3]{content/chapter8/images/part2.png}
			\caption{Hard disk partitions}
			\label{hard_disk_partitions}
		\end{figure}		
	\end{itemize}

\newpage

	
\end{flushleft}

\newpage


\subsection{HDD devices naming in Linux}
\setlength{\columnsep}{3pt}
\begin{flushleft}
	\begin{itemize}
		\item In Linux, both HDD and partitions are represented as \textbf{block device files}.
		\item Block device files are located under \textbf{/dev}. 
		\item Command to find all block device files under /dev:
		\bigskip
		\begin{tcolorbox}[breakable,notitle,boxrule=-0pt,colback=black,colframe=black]
			\color{green}
			\fontdimen2\font=1em
			\# find /dev -type b
			\fontdimen2\font=4pt
		\end{tcolorbox}
	
	\end{itemize}

	
	\paragraph{Disk drives naming}
	\begin{itemize}
	\item \textbf{HDD names starting with "sd"}:
	\begin{enumerate}
		\item The \textbf{IDE/SATA/SCSI type of disk drive} are represented with name starting from \textbf{"sd"}.
		\item Eg: \textbf{/dev/sda}, \textbf{/dev/sdb}, \textbf{/dev/sdc} and so on.
	\end{enumerate}
	
	\item \textbf{HDD names starting with "vd"}:
	\begin{enumerate}
	\item The \textbf{paravirtualizated disk driver} are represented with name starting from \textbf{"vd"}. 
	\item Eg: \textbf{/dev/vda}, \textbf{/dev/vdb}, \textbf{/dev/vdc} and so on.
	\end{enumerate}
	\begin{figure}[h!]
		\centering
		\includegraphics[scale=.38]{content/chapter8/images/name.png}
		\caption{Hard disk partitions}
		\label{HDD_naming}
	\end{figure}		
	\end{itemize}	
\newpage
\paragraph{Partition Names}
\begin{itemize}
	\item The partition are represented as numbers at the end of the HDD names.
	\item Eg: For HDD \textbf{"/dev/sda"}, partitions will be numbered as \textbf{/dev/sda1}, \textbf{/dev/sda2} and so on.
\end{itemize}
	\begin{figure}[h!]
	\centering
	\includegraphics[scale=.38]{content/chapter8/images/partition_name.png}
	\caption{Partition names}
	\label{Partitions_naming}
\end{figure}		


\end{flushleft}
\newpage



\subsection{Commands to check HDD}
\setlength{\columnsep}{3pt}
\begin{flushleft}

\bigskip
\begin{itemize}
	\item \textbf{lsblk}: Lists information about all available block devices.
	
	\bigskip
	\begin{tcolorbox}[breakable,notitle,boxrule=-0pt,colback=pink,colframe=pink]
		\color{black}
		\fontdimen2\font=1em
		Syntax: lsblk
		\fontdimen2\font=4pt
	\end{tcolorbox}
	Eg:
	\begin{figure}[h!]
		\centering
		\includegraphics[scale=0.45]{content/chapter8/images/lsblk.png}
		\caption{lsblk command output}
		\label{fig:lsblk}
	\end{figure}
	
	\item \textbf{fdisk -l}: Display more detailed information about all disk drives.
	\bigskip
	\begin{tcolorbox}[breakable,notitle,boxrule=-0pt,colback=pink,colframe=pink]
		\color{black}
		\fontdimen2\font=1em
		Syntax: fdisk -l
		\fontdimen2\font=4pt
	\end{tcolorbox}
	Eg:
	\begin{figure}[h!]
		\centering
		\includegraphics[scale=0.45]{content/chapter8/images/fdisk.png}
		\caption{"fdisk -l" command output}
		\label{fig:fdisk}
	\end{figure}
	
	\item \textbf{du}: Estimate file space usage.
	\bigskip
	\begin{tcolorbox}[breakable,notitle,boxrule=-0pt,colback=pink,colframe=pink]
		\color{black}
		\fontdimen2\font=1em
		Syntax: du [option] [file/folder]
		\fontdimen2\font=4pt
	\end{tcolorbox}
	Eg: 
	\begin{tcolorbox}[breakable,notitle,boxrule=-0pt,colback=black,colframe=black]
		\color{green}
		\fontdimen2\font=1em
		\# du /home
		\fontdimen2\font=4pt
	\end{tcolorbox}

	Options with \textbf{du} command:
	
	\begin{itemize}
		\item \textbf{-sh}: Display only a total for each argument in human readable format.
		\begin{tcolorbox}[breakable,notitle,boxrule=-0pt,colback=pink,colframe=pink]
			\color{black}
			\fontdimen2\font=1em
			Syntax: du -sh [folder/file]
			\fontdimen2\font=4pt
		\end{tcolorbox}
		Eg:
		\begin{tcolorbox}[breakable,notitle,boxrule=-0pt,colback=black,colframe=black]
			\color{green}
			\fontdimen2\font=1em
			\# du -sh /home
			\newline
			\color{white}
			16K	/home
			\fontdimen2\font=4pt
		\end{tcolorbox}
		\bigskip
		\bigskip		
\end{itemize}
	
\end{flushleft}

\newpage


\subsection{Practice}
\input{content/chapter8/8.1.4.tex}

\section{Partition types}\index{Understanding vi/vim editor}
\setlength{\columnsep}{3pt}
\begin{flushleft}
	\bigskip
	\bigskip
	\begin{tcolorbox}[breakable,notitle,boxrule=1pt,colback=black,colframe=black]
		\color{white}
		\bigskip
		In this section, you are going to learn:
		\begin{enumerate}
			\item \textbf{What is MBR \& partition table?}
			\item \textbf{Types of partition:}
			\begin{itemize}
	 			\item \textbf{Primary partition}
				\item \textbf{Extended partition}
				\item \textbf{Logical partition}
			\end{itemize}
			\item \textbf{What is filesystem? How to implement it on partition?}
			\item \textbf{Temporary \& permanent mounting}
			\item \textbf{Swap partition}
			\item \textbf{Setting up swap partition}
		\end{enumerate}	
		\bigskip
		Finally, there will be a \textbf{small excerise} on these topics to check your knowledge.
		\bigskip
	\end{tcolorbox}
	

	
	\begin{multicols}{2}
		\vspace*{\fill}
		\vspace*{\fill}
		\vspace*{\fill}
		\vspace*{\fill}
		\vspace*{\fill}
		\vspace*{\fill}
		\vspace*{\fill}
		\vspace*{\fill}
		\vspace*{\fill}
		
		\vfill \null
		\columnbreak
		So let's get started....
		\includegraphics[scale=0.08]{content/linux_section.png}
	\end{multicols}	
	
\end{flushleft}

\newpage


\subsection{Partition table in MBR}\index{Understanding vi/vim editor!About vi/vim editor}
\setlength{\columnsep}{3pt}
\begin{flushleft}
	
	\paragraph{What is an MBR?}
	\begin{itemize}
		\item MBR stands for \textbf{Master Boot Recorder}.
		\item MBR is the first \textbf{512 bytes} of HDD.
		\item MBR consists of:
		\begin{itemize}
			\item Bootloader (446 bytes in size) - More on this in chapter 17 under section 17.1.4.
			\item \textbf{Partition Table}  (64 bytes in size) - Stores entry of maximum 4 partitions (16 byte entry for each partition).
			\item Magic number (2 bytes in size) - More on this in chapter 17 under section 17.1.3.
		\end{itemize}
	\end{itemize}	
	 
	
		\begin{figure}[h!]
			\centering
			\includegraphics[scale=.6]{content/chapter8/images/correction3.png}
			\caption{MBR}
			\label{mbr_naming}
		\end{figure}		
	
\end{flushleft}

\newpage


\subsection{Primary partitions}\index{Understanding vi/vim editor!About vi/vim editor}
\input{content/chapter8/8.2.2.tex}
\subsection{Extended partition}\index{Understanding vi/vim editor!About vi/vim editor}
\input{content/chapter8/8.2.3.tex}
\subsection{Logical partitions}
\input{content/chapter8/8.2.4.tex}
\subsection{Standard formatting filesystems}
\input{content/chapter8/8.3.2.tex}
\subsection{Mounting Partitions}
\setlength{\columnsep}{3pt}
\begin{flushleft}
	

		Mounting a partition means attaching it to some directory so that it can be used. 
		
		\begin{figure}[h!]
			\centering
			\includegraphics[scale=.6]{content/chapter8/images/new.png}
			\caption{Mounting filesystem}
			\label{mounting}
		\end{figure}	
	
	\newpage
	\paragraph{Command to mount a partition}
		\begin{itemize}		
		\item Mount a linux partition:
		\bigskip
		\begin{tcolorbox}[breakable,notitle,boxrule=-0pt,colback=pink,colframe=pink]
			\color{black}
			\fontdimen2\font=1em
			Syntax: mount device\_name directory\_name
			\fontdimen2\font=4pt
		\end{tcolorbox}
		
		\item Check mounted partition:
		\bigskip
		\begin{tcolorbox}[breakable,notitle,boxrule=-0pt,colback=pink,colframe=pink]
			\color{black}
			\fontdimen2\font=1em
			Syntax: df -Th
			\fontdimen2\font=4pt
		\end{tcolorbox}
		
		\item Eg: Create a directory to mount a partition -
		\bigskip
		\begin{tcolorbox}[breakable,notitle,boxrule=-0pt,colback=black,colframe=black]
			\color{green}
			\fontdimen2\font=1em
			\# mkdir /mnt/access
			\fontdimen2\font=4pt
		\end{tcolorbox}
		Mount \textbf{/dev/sda2} partition on \textbf{/mnt/access} directory -
		\bigskip
		\begin{tcolorbox}[breakable,notitle,boxrule=-0pt,colback=black,colframe=black]
			\color{green}
			\fontdimen2\font=1em
			\# mount /dev/sda2 /mnt/access
			\fontdimen2\font=4pt
		\end{tcolorbox}
		Check mounted partition:
		\bigskip
		\begin{tcolorbox}[breakable,notitle,boxrule=-0pt,colback=black,colframe=black]
			\color{green}
			\fontdimen2\font=1em
			\# df -Th
			\fontdimen2\font=4pt
		\end{tcolorbox}
		
	\end{itemize}
	\bigskip
	\begin{tcolorbox}[breakable,notitle,boxrule=1pt,colback=yellow,colframe=yellow]
		\color{black}
		Note: Mounting done using \textbf{mount} command is \textbf{temporary}. After system reboot, the partition and directory will get dissociated. This will cause partition to be \textbf{unmounted}.
	\end{tcolorbox}
	
	
	
\end{flushleft}

\newpage


\subsection{Permanent mounting:/etc/fstab}
\input{content/chapter8/8.3.4.tex}
\subsection{Unmounting filesystem}
\setlength{\columnsep}{3pt}
\begin{flushleft}
	

	\begin{itemize}
		\item \textbf{lsof}: 
		\begin{itemize}
			\item Lists all open files and the process accessing them in the mounted directory.
			\item It is useful to identify which processes currently prevent the file system from successful unmounting.
		\end{itemize}
		 
		\begin{tcolorbox}[breakable,notitle,boxrule=-0pt,colback=pink,colframe=pink]
			\color{black}
			\fontdimen2\font=1em
			Syntax: lsof directory-name
			\fontdimen2\font=4pt
		\end{tcolorbox}
		Eg:	
		\bigskip
		\begin{tcolorbox}[breakable,notitle,boxrule=-0pt,colback=black,colframe=black]
			\color{green}
			\fontdimen2\font=1em
			\# lsof /project
			\newline
			\color{white}
			COMMAND  PID USER   FD   TYPE DEVICE SIZE/OFF NODE NAME
			\newline
			bash    5514 jack  cwd    DIR  259,6     4096    2 /project
			\fontdimen2\font=4pt
		\end{tcolorbox}
		\bigskip
		\bigskip
		
		\item \textbf{umount}: Unmounts the filesystem.
		\begin{tcolorbox}[breakable,notitle,boxrule=-0pt,colback=pink,colframe=pink]
			\color{black}
			\fontdimen2\font=1em
			Syntax: umount directory-name
			\fontdimen2\font=4pt
		\end{tcolorbox}
		Eg:	
		\bigskip
		\begin{tcolorbox}[breakable,notitle,boxrule=-0pt,colback=black,colframe=black]
			\color{green}
			\fontdimen2\font=1em
			\# umount /project
			\fontdimen2\font=4pt
		\end{tcolorbox}
	\end{itemize}	
	
	
\end{flushleft}

\newpage


\subsection{Swap partition}\index{Understanding vi/vim editor!About vi/vim editor}
\input{content/chapter8/8.2.5.tex}
\subsection{Setting up swap space}\index{Understanding vi/vim editor!About vi/vim editor}
\input{content/chapter8/8.3.5.tex}
\subsection{Practice}\index{Understanding vi/vim editor!Practice}
\input{content/chapter8/8.2.6.tex}
%%%%%
%%%%%


%%%%%----------------------------------------------------------------------------------------
%%%%%	CHAPTER 9
%%%%%----------------------------------------------------------------------------------------
\chapterimage{index10.png} % Table of contents heading image
\chapter{Logical Volume Management}
%%%%%-----------------------
\section{Introduction to LVM}\index{Understanding vi/vim editor}
\input{content/chapter9/9.1.tex}
\subsection{What is LVM?}\index{Understanding vi/vim editor!About vi/vim editor}
\setlength{\columnsep}{3pt}
\begin{flushleft}
	\bigskip
	\textbf{L}ogical \textbf{V}olume \textbf{M}anagement (LVM) makes it easier to manage disk space. 
\bigskip

\paragraph{Problem with normal partition:}
	\begin{itemize}
		\item Once a partition is created, it cannot be resized. 
		\item If a partition needs to be reduced or expanded, it's not possible.	
	\end{itemize}

\bigskip

\paragraph{LVM is used for the following purposes:}
\begin{itemize}
	\item Combine multiple HDD storage and use them as single HDD.
	\item If existing partition size is 100\% full, you can extend it without the need to unmount the partition.
	\item If existing partition size is very large and resulting in space wasteage, you can reduce it without the need to unmount the partition.
\end{itemize}

	
\end{flushleft}

\newpage


\subsection{LVM organisation}\index{Understanding vi/vim editor!About vi/vim editor}
\input{content/chapter9/9.2.1.tex}
\subsection{Volumes practical}\index{Understanding vi/vim editor!About vi/vim editor}
\input{content/chapter9/9.2.2.tex}
\subsection{Physical extends}\index{Understanding vi/vim editor!About vi/vim editor}
\input{content/chapter9/9.2.3.tex}
\subsection{Practice}\index{Understanding vi/vim editor!Practice}
\input{content/chapter9/9.2.4.tex}
%%%%%-----------------------
%%%%
%%%%
%----------------------------------------------------------------------------------------
%	CHAPTER 10
%----------------------------------------------------------------------------------------
\chapterimage{index11.png} % Table of contents heading image
\chapter{Links in Linux}
%-----------------------
\section{Inodes \& Links}\index{Inodes \& Links}
\input{content/chapter10/10.1.tex}
\subsection{What are inodes?}\index{Inodes \& Links!What are inodes?}
\input{content/chapter10/10.1.1.tex}
\subsection{Inode number and filesystem}\index{Inodes \& Links!Inode number and filesystem}
\input{content/chapter10/10.1.2.tex}
\subsection{Links and it's types}\index{Inodes \& Links!Links and it's types}
\input{content/chapter10/10.1.3.tex}
\subsection{Difference between links}\index{Inodes \& Links!Difference between links}
\input{content/chapter10/10.1.4.tex}
\subsection{Practice}\index{Inodes \& Links!Practice}
\input{content/chapter10/10.1.5.tex}
%%%%%-----------------------
%%%%
%%%%



%%%%%----------------------------------------------------------------------------------------
%%%%%	CHAPTER 11
%%%%%----------------------------------------------------------------------------------------
\chapterimage{index12.png} % Table of contents heading image
\chapter{Package Management}
%%%%%-----------------------
\section{RPM in detail}\index{Understanding vi/vim editor}
\input{content/chapter11/11.1.tex}
\subsection{What is a package?}\index{Understanding vi/vim editor!About vi/vim editor}
\input{content/chapter11/11.1.1.tex}
\subsection{What is an RPM?}\index{Understanding vi/vim editor!About vi/vim editor}
\input{content/chapter11/11.1.2.tex}
\subsection{RPM commands}\index{Understanding vi/vim editor!About vi/vim editor}
\input{content/chapter11/11.1.3.tex}
\subsection{Drawback of RPM}\index{Understanding vi/vim editor!About vi/vim editor}
\input{content/chapter11/11.1.4.tex}
\subsection{Practice}\index{Understanding vi/vim editor!Practice}
\input{content/chapter11/11.1.5.tex}
\section{YUM \& dnf}\index{Understanding vi/vim editor}
\input{content/chapter11/11.2.tex}
\subsection{What is YUM \& dnf?}\index{Understanding vi/vim editor!About vi/vim editor}
\input{content/chapter11/11.2.1.tex}
\subsection{Configuration of YUM server}\index{Understanding vi/vim editor!About vi/vim editor}
\input{content/chapter11/11.2.2.tex}
\subsection{Configuration of YUM client}\index{Understanding vi/vim editor!About vi/vim editor}
\input{content/chapter11/11.2.3.tex}
\subsection{Subscribing RHEL8 server}\index{Understanding vi/vim editor!About vi/vim editor}
\input{content/chapter11/11.2.7.tex}
\subsection{YUM commands}\index{Understanding vi/vim editor!About vi/vim editor}
\input{content/chapter11/11.2.4.tex}
\subsection{Dnf commands}\index{Understanding vi/vim editor!About vi/vim editor}
\setlength{\columnsep}{3pt}
\begin{flushleft}
	\bigskip
	
	\textbf{dnf}: Used to install, update, remove \& search package.
	\begin{tcolorbox}[breakable,notitle,boxrule=-0pt,colback=pink,colframe=pink]
		\color{black}
		\fontdimen2\font=1em
		Syntax: dnf [options] command [package\_name]
		\fontdimen2\font=4pt
	\end{tcolorbox}
	Command and options with \textbf{dnf} command:
	
	\begin{itemize}
		\item \textbf{install}: Install package along with it's dependency.
		\newline
		\textbf{-y}: Automatically answer yes for all questions.
		\begin{tcolorbox}[breakable,notitle,boxrule=-0pt,colback=pink,colframe=pink]
			\color{black}
			\fontdimen2\font=1em
			Syntax: dnf install package\_name -y
			\fontdimen2\font=4pt
		\end{tcolorbox}
		Eg:
		\begin{tcolorbox}[breakable,notitle,boxrule=-0pt,colback=black,colframe=black]
			\color{green}
			\fontdimen2\font=1em
			\# dnf install httpd -y
			\fontdimen2\font=4pt
		\end{tcolorbox}
		\bigskip
		\bigskip
		\item \textbf{remove}: Uninstall a package along with this dependency.
		\begin{tcolorbox}[breakable,notitle,boxrule=-0pt,colback=pink,colframe=pink]
			\color{black}
			\fontdimen2\font=1em
			Syntax: dnf remove package\_name 
			\fontdimen2\font=4pt
		\end{tcolorbox}
		Eg:
		\begin{tcolorbox}[breakable,notitle,boxrule=-0pt,colback=black,colframe=black]
			\color{green}
			\fontdimen2\font=1em
			\# dnf remove httpd
			\fontdimen2\font=4pt
		\end{tcolorbox}
		\bigskip
		\bigskip		
		
		\item \textbf{info}: Display the information of a package.
		\begin{tcolorbox}[breakable,notitle,boxrule=-0pt,colback=pink,colframe=pink]
			\color{black}
			\fontdimen2\font=1em
			Syntax: dnf info package\_name 
			\fontdimen2\font=4pt
		\end{tcolorbox}
		Eg:
		\begin{tcolorbox}[breakable,notitle,boxrule=-0pt,colback=black,colframe=black]
			\color{green}
			\fontdimen2\font=1em
			\# dnf info httpd
			\fontdimen2\font=4pt
		\end{tcolorbox}
		\bigskip
		\bigskip		

		\item \textbf{groupinfo}: List all packages provided by the group.
		\begin{tcolorbox}[breakable,notitle,boxrule=-0pt,colback=pink,colframe=pink]
			\color{black}
			\fontdimen2\font=1em
			Syntax: dnf groupinfo group\_name 
			\fontdimen2\font=4pt
		\end{tcolorbox}
		Eg:
		\begin{tcolorbox}[breakable,notitle,boxrule=-0pt,colback=black,colframe=black]
			\color{green}
			\fontdimen2\font=1em
			\# dnf groupinfo "Development Tools"
			\fontdimen2\font=4pt
		\end{tcolorbox}
		\bigskip
		\bigskip	
		
		\item \textbf{groupinstall}: Install all packages provided by the group.
		\begin{tcolorbox}[breakable,notitle,boxrule=-0pt,colback=pink,colframe=pink]
			\color{black}
			\fontdimen2\font=1em
			Syntax: dnf groupinstall group\_name 
			\fontdimen2\font=4pt
		\end{tcolorbox}
		Eg:
		\begin{tcolorbox}[breakable,notitle,boxrule=-0pt,colback=black,colframe=black]
			\color{green}
			\fontdimen2\font=1em
			\# dnf groupinstall "Development Tools"
			\fontdimen2\font=4pt
		\end{tcolorbox}
		\bigskip
		\bigskip	
		
		\item \textbf{groupremove}: Remove all packages provided by the group.
		\begin{tcolorbox}[breakable,notitle,boxrule=-0pt,colback=pink,colframe=pink]
			\color{black}
			\fontdimen2\font=1em
			Syntax: dnf groupremove group\_name 
			\fontdimen2\font=4pt
		\end{tcolorbox}
		Eg:
		\begin{tcolorbox}[breakable,notitle,boxrule=-0pt,colback=black,colframe=black]
			\color{green}
			\fontdimen2\font=1em
			\# dnf groupremove "Development Tools"
			\fontdimen2\font=4pt
		\end{tcolorbox}
		\bigskip
		\bigskip	



		\item \textbf{update}: Update all packages and their dependencies.
		\begin{tcolorbox}[breakable,notitle,boxrule=-0pt,colback=pink,colframe=pink]
			\color{black}
			\fontdimen2\font=1em
			Syntax: dnf update
			\fontdimen2\font=4pt
		\end{tcolorbox}
		\bigskip
		\bigskip	

		\item \textbf{history}: Display a list of all the latest dnf transactions
		\begin{tcolorbox}[breakable,notitle,boxrule=-0pt,colback=pink,colframe=pink]
			\color{black}
			\fontdimen2\font=1em
			Syntax: dnf history
			\fontdimen2\font=4pt
		\end{tcolorbox}
		\bigskip
		\bigskip			

		\item \textbf{history info}: Examine a particular dnf transaction.
		\begin{tcolorbox}[breakable,notitle,boxrule=-0pt,colback=pink,colframe=pink]
			\color{black}
			\fontdimen2\font=1em
			Syntax: dnf history info transactionID
			\fontdimen2\font=4pt
		\end{tcolorbox}
		Eg:
		\begin{tcolorbox}[breakable,notitle,boxrule=-0pt,colback=black,colframe=black]
			\color{green}
			\fontdimen2\font=1em
			\# dnf history info 8
			\fontdimen2\font=4pt
		\end{tcolorbox}
		\bigskip
		\bigskip			

		\item \textbf{history undo}: Revert a particular transaction.
		\begin{tcolorbox}[breakable,notitle,boxrule=-0pt,colback=pink,colframe=pink]
			\color{black}
			\fontdimen2\font=1em
			Syntax: dnf history undo transactionID
			\fontdimen2\font=4pt
		\end{tcolorbox}
		Eg:
		\begin{tcolorbox}[breakable,notitle,boxrule=-0pt,colback=black,colframe=black]
			\color{green}
			\fontdimen2\font=1em
			\# dnf history undo 8
			\fontdimen2\font=4pt
		\end{tcolorbox}
		\bigskip
		\bigskip			
		\newpage
		\item \textbf{history redo}: Repeat a particular transaction.
		\begin{tcolorbox}[breakable,notitle,boxrule=-0pt,colback=pink,colframe=pink]
			\color{black}
			\fontdimen2\font=1em
			Syntax: dnf history redo transactionID
			\fontdimen2\font=4pt
		\end{tcolorbox}
		Eg:
		\begin{tcolorbox}[breakable,notitle,boxrule=-0pt,colback=black,colframe=black]
			\color{green}
			\fontdimen2\font=1em
			\# dnf history redo 8
			\fontdimen2\font=4pt
		\end{tcolorbox}
		\bigskip
		\bigskip		
		
		
		\item \textbf{list all}: List all installed packages.
		\begin{tcolorbox}[breakable,notitle,boxrule=-0pt,colback=pink,colframe=pink]
			\color{black}
			\fontdimen2\font=1em
			Syntax: dnf list all
			\fontdimen2\font=4pt
		\end{tcolorbox}
		Eg: List all installed pacakges associated with httpd.
		\begin{tcolorbox}[breakable,notitle,boxrule=-0pt,colback=black,colframe=black]
			\color{green}
			\fontdimen2\font=1em
			\# dnf list all | grep httpd
			\fontdimen2\font=4pt
		\end{tcolorbox}
		\bigskip
		\bigskip					

		\item \textbf{localinstall}: Install locally downloaded package.
		\begin{tcolorbox}[breakable,notitle,boxrule=-0pt,colback=pink,colframe=pink]
			\color{black}
			\fontdimen2\font=1em
			Syntax: dnf localinstall package.rpm
			\fontdimen2\font=4pt
		\end{tcolorbox}
		Eg: Download cvs package using command \textbf{"wget https://rpmfind.net/linux/epel/8/Everything/x86\_64/Packages/c/cvs-1.11.23-52.el8.x86\_64.rpm"}
		\newline
		Install downloaded package using command:		
		\begin{tcolorbox}[breakable,notitle,boxrule=-0pt,colback=black,colframe=black]
			\color{green}
			\fontdimen2\font=1em
			\# dnf localinstall cvs-1.11.23-52.el8.x86\_64.rpm -y
			\fontdimen2\font=4pt
		\end{tcolorbox}
		\bigskip
		\bigskip					
		
					
	\end{itemize}
	
	
\end{flushleft}
\newpage




\subsection{Practice}\index{Understanding vi/vim editor!Practice}
\input{content/chapter11/11.2.5.tex}
%%%%%-----------------------
%%%%
%%%%%----------------------------------------------------------------------------------------
%%%%%	CHAPTER 12
%%%%%----------------------------------------------------------------------------------------
\chapterimage{index13.png} % Table of contents heading image
\chapter{Process, CPU \& Memory}
%%%%%-----------------------
\section{Process management}
\input{content/chapter12/12.1.tex}
\subsection{What is a process?}
\input{content/chapter12/12.1.1.tex}
\subsection{Types of process}
\input{content/chapter12/12.1.2.tex}
\subsection{Linux process states}
\input{content/chapter12/12.1.3.tex}
\subsection{How to check Linux process?}
\input{content/chapter12/12.1.4.tex}
\subsection{Killing process}
\input{content/chapter12/12.1.6.tex}
\subsection{Process priority}
\input{content/chapter12/12.1.13.tex}
\subsection{Practice}
\input{content/chapter12/12.1.5.tex}
\section{CPU management}
\input{content/chapter12/12.3.tex}
\subsection{What is CPU?}
\input{content/chapter12/12.1.7.tex}
\subsection{CPU core \& CPU threads}
\input{content/chapter12/12.1.7.1.tex}
\subsection{Types of CPU}
\input{content/chapter12/12.1.7.2.tex}
\subsection{Undertanding CPU calculation}
\input{content/chapter12/12.1.7.3.tex}
\subsection{Commands for CPU}
\input{content/chapter12/12.1.7.4.tex}
\subsection{Understanding CPU consumption}
\input{content/chapter12/12.1.7.6.tex}
\subsection{Practice}\index{Understanding vi/vim editor!Practice}
\input{content/chapter12/12.1.7.5.tex}
\section{Memory management}\index{Understanding vi/vim editor}
\input{content/chapter12/12.2.tex}
\subsection{What is RAM?}\index{Understanding vi/vim editor!About vi/vim editor}
\input{content/chapter12/12.1.11.tex}
\subsection{What is cache?}\index{Understanding vi/vim editor!About vi/vim editor}
\input{content/chapter12/12.1.8.tex}
\subsection{What is buffer?}\index{Understanding vi/vim editor!About vi/vim editor}
\input{content/chapter12/12.1.9.tex}
\subsection{Commands for memory}\index{Understanding vi/vim editor!About vi/vim editor}
\input{content/chapter12/12.1.10.tex}
\subsection{Practice}\index{Understanding vi/vim editor!Practice}
\input{content/chapter12/12.1.12.tex}

%%%%%-----------------------
%%%%
%%%%%----------------------------------------------------------------------------------------
%%%%%	CHAPTER 13
%%%%%----------------------------------------------------------------------------------------
\chapterimage{index14.png} % Table of contents heading image
\chapter{Scheduling Jobs}
%%%%%-----------------------
\section{Crontab in detail}\index{Understanding vi/vim editor}
\input{content/chapter13/13.1.tex}
\subsection{Introduction to crontab}\index{Understanding vi/vim editor!About vi/vim editor}
\input{content/chapter13/13.1.1.tex}
\subsection{Command to set crontab}\index{Understanding vi/vim editor!Practice}
\input{content/chapter13/13.1.3.tex}
\subsection{Examples of cronjob}\index{Understanding vi/vim editor!About vi/vim editor}
\input{content/chapter13/13.1.2.tex}
\subsection{Practice}\index{Understanding vi/vim editor!Practice}
\input{content/chapter13/13.1.4.tex}
%\input{content/chapter13/13.1.4.tex}
%%%%\subsection{Soft link}\index{Understanding vi/vim editor!About vi/vim editor}
%%%%\setlength{\columnsep}{3pt}
\begin{flushleft}
	\bigskip
	There are 2 major types of editors in Linux:
	\begin{itemize}
		\item \textbf{Grahpical editor}
		\begin{itemize}
			\item \textbf{gedit}: 	Gedit application is a full-featured text editor.
			\begin{tcolorbox}[breakable,notitle,boxrule=0pt,colback=pink,colframe=pink]
				\color{black}
				\fontdimen2\font=1em
				Syntax: gedit filename
				\fontdimen2\font=4pt
			\end{tcolorbox}
			Eg:
			\begin{figure}[h!]
				\centering
				\includegraphics[scale=0.4]{content/chapter3/images/gedit.png}
				\caption{Sample output}
				\label{fig:cal32}
			\end{figure}
			
		\end{itemize}
		\item \textbf{Text-based editor}
		\begin{itemize}
			\item \textbf{vi or vim}:  Text based editor used in Linux and Mac OS. Let's see more on this.
		\end{itemize}
	\end{itemize}
\end{flushleft}

\newpage






%%%%\subsection{Practice}\index{Understanding vi/vim editor!Practice}
%%%%\input{content/chapter3/3.2.1.tex}
%%%%%-----------------------
%%%%
%%%%%----------------------------------------------------------------------------------------
%%%%%	CHAPTER 14
%%%%%----------------------------------------------------------------------------------------
\chapterimage{index15.png} % Table of contents heading image
\chapter{Networking}
%%%%%-----------------------
\section{Networking essentials}\index{Understanding vi/vim editor}
\input{content/chapter14/14.1.tex}
\subsection{Networking \& it's types}\index{Understanding vi/vim editor!About vi/vim editor}
\input{content/chapter14/14.1.1.tex}
\subsection{Network connecting device}\index{Understanding vi/vim editor!About vi/vim editor}
\input{content/chapter14/14.1.3.tex}
\subsection{Network interface names}\index{Understanding vi/vim editor!About vi/vim editor}
\input{content/chapter14/14.1.3.3.tex}
\subsection{Practice}\index{Understanding vi/vim editor!Practice}
\input{content/chapter14/14.1.4.tex}
\section{Introduction to IP}\index{Understanding vi/vim editor}
\input{content/chapter14/14.2.tex}
\subsection{What is Internet Protocol (IP)?}\index{Understanding vi/vim editor!About vi/vim editor}
\input{content/chapter14/14.2.1.tex}
\subsection{What is an IP address?}\index{Understanding vi/vim editor!About vi/vim editor}
\input{content/chapter14/14.2.2.tex}
\subsection{What is DNS server? }\index{Understanding vi/vim editor!About vi/vim editor}
\input{content/chapter14/14.2.3.4.tex}
\subsection{What is DHCP server? }\index{Understanding vi/vim editor!About vi/vim editor}
\input{content/chapter14/14.2.3.3.tex}
\subsection{Types of IP address}\index{Understanding vi/vim editor!About vi/vim editor}
\input{content/chapter14/14.2.3.tex}
\subsection{ISO-OSI model}\index{Understanding vi/vim editor!About vi/vim editor}
\input{content/chapter14/14.2.4.tex}
\subsection{Practice}\index{Understanding vi/vim editor!Practice}
\input{content/chapter14/14.2.5.tex}
\section{IPv4 in detail}\index{Understanding vi/vim editor}
\input{content/chapter14/14.3.tex}
\subsection{IPv4 address structure}\index{Understanding vi/vim editor!About vi/vim editor}
\input{content/chapter14/14.3.1.tex}
\subsection{Network bit \& Host bit}\index{Understanding vi/vim editor!About vi/vim editor}
\input{content/chapter14/14.3.2.tex}
\subsection{Netmask}\index{Understanding vi/vim editor!About vi/vim editor}
\input{content/chapter14/14.3.3.tex}
\subsection{What is classful addressing?}\index{Understanding vi/vim editor!About vi/vim editor}
\input{content/chapter14/14.3.4.tex}
\subsection{Loopback \& 0.0.0.0 address}\index{Understanding vi/vim editor!About vi/vim editor}
\input{content/chapter14/14.3.4.1.tex}
\subsection{Network address}\index{Understanding vi/vim editor!About vi/vim editor}
\input{content/chapter14/14.3.5.tex}
\subsection{Broadcast address}\index{Understanding vi/vim editor!About vi/vim editor}
\input{content/chapter14/14.3.6.tex}
\subsection{Calculating number of host IP}\index{Understanding vi/vim editor!About vi/vim editor}
\input{content/chapter14/14.3.7.tex}
\subsection{Practice}\index{dsdsd sdsdsd sdsd sdsd}
\input{content/chapter14/14.3.9.tex}
\section{Subnetting}\index{Understanding vi/vim editor!About vi/vim editor}
\input{content/chapter14/14.4.tex}
\subsection{Drawback of classful addressing}\index{Understanding vi/vim editor!About vi/vim editor}
\input{content/chapter14/14.4.1.tex}
\subsection{Subnetting/CIDR}\index{Understanding vi/vim editor!About vi/vim editor}
\input{content/chapter14/14.4.2.tex}
\subsection{Working of CIDR}\index{Understanding vi/vim editor!About vi/vim editor}
\input{content/chapter14/14.4.3.tex}
\subsection{Practice}\index{Understanding vi/vim editor!Practice}
\input{content/chapter14/14.4.4.tex}
\section{Networking in action}\index{Understanding vi/vim editor!About vi/vim editor}
\input{content/chapter14/14.5.tex}
\subsection{Assigning hostname}\index{}
\input{content/chapter14/14.5.1.tex}
\subsection{/etc/hosts file}\index{}
\input{content/chapter14/14.5.4.tex}
\subsection{Setting up IP address}\index{Understanding vi/vim editor!About vi/vim editor}
\input{content/chapter14/14.5.2.tex}
\subsection{Network monitoring commands}\index{Understanding vi/vim editor!About vi/vim editor}
\input{content/chapter15/15.2.1.tex}
\subsection{Practice}\index{Understanding vi/vim editor!Practice}
\input{content/chapter14/14.5.3.tex}


%%%%%-----------------------
%%%%
%%%%
%%%%%----------------------------------------------------------------------------------------
%%%%%	CHAPTER 15
%%%%%----------------------------------------------------------------------------------------
%\chapterimage{index16.png} % Table of contents heading image
%\chapter{System \& network monitoring}
%%%%%-----------------------
%\section{System monitoring}\index{Understanding vi/vim editor}
%\input{content/chapter15/15.1.tex}
%\subsection{Commands}\index{Understanding vi/vim editor!About vi/vim editor}
%\input{content/chapter15/15.1.1.tex}
%\subsection{Practice}\index{Understanding vi/vim editor!Practice}
%\input{content/chapter15/15.1.2.tex}
%\section{Network monitoring}\index{Understanding vi/vim editor}
%\input{content/chapter15/15.2.tex}
%\subsection{Commands}\index{Understanding vi/vim editor!About vi/vim editor}
%\input{content/chapter15/15.2.1.tex}
%\subsection{Practice}\index{Understanding vi/vim editor!Practice}
%\input{content/chapter15/15.2.2.tex}
%%%%%-----------------------
%%%%
%%%%%----------------------------------------------------------------------------------------
%%%%%	CHAPTER 16
%%%%%----------------------------------------------------------------------------------------
\chapterimage{index17.png} % Table of contents heading image
\chapter{Logs in Linux}
%%%%%-----------------------
\section{Logging in Linux}\index{Understanding vi/vim editor}
\input{content/chapter16/16.1.tex}
\subsection{Introduction to logs}\index{Understanding vi/vim editor!About vi/vim editor}
\input{content/chapter16/16.1.1.tex}
\subsection{Syslog \& rsyslog}\index{Understanding vi/vim editor!About vi/vim editor}
\input{content/chapter16/16.1.2.tex}
\subsection{Rules in /etc/rsyslog.conf}\index{Understanding vi/vim editor!About vi/vim editor}
\input{content/chapter16/16.1.3.tex}
\subsection{Important log files}\index{Understanding vi/vim editor!About vi/vim editor}
\input{content/chapter16/16.1.4.tex}
\subsection{Logger command}\index{Understanding vi/vim editor!About vi/vim editor}
\input{content/chapter16/16.1.5.tex}
\subsection{Viewing logs}
\input{content/chapter16/16.1.7.tex}
\subsection{Practice}\index{Understanding vi/vim editor!Practice}
\setlength{\columnsep}{3pt}
\begin{flushleft}
	
	\paragraph{}
	\bigskip
	
	\begin{figure}[h!]
		\centering
		\includegraphics[scale=.2]{content/practise.jpg}
	\end{figure}	
	\begin{enumerate}
		
		\item \textbf{Which of the following directory stored all logs in Linux?}
		\begin{enumerate}[label=(\alph*)]
			\item /var/spool/log
			\item /var/log   %correct
			\item /var/tmp  
			\item /var/tmp/log
		\end{enumerate}
		\bigskip
		\bigskip	
		

		\item \textbf{Which of the following is the configuration file for rsyslog?}
		\begin{enumerate}[label=(\alph*)]
			\item /etc/syslog.config
			\item /etc/rsyslog
			\item /etc/rsyslog.conf     %correct 
			\item /etc/rsyslog.config
		\end{enumerate}
		\bigskip
		\bigskip	
		
		\item \textbf{Which of the following are valid facility level? (Select all that applies.)}
		\begin{enumerate}[label=(\alph*)]
			\item kern   %correct
			\item cron  %correct
			\item mail  %correct
			\item auth   %correct
		\end{enumerate}
		\bigskip
		\bigskip	
		
		\item \textbf{Which of the following are valid priority level? (Select all that applies.)}
		\begin{enumerate}[label=(\alph*)]
			\item emerg   %correct
			\item crit  %correct
			\item error  %correct
			\item info   %correct
		\end{enumerate}
		\bigskip
		\bigskip	

		\item \textbf{Which of the following log file is used to store booting \& startup logs?}
		\begin{enumerate}[label=(\alph*)]
			\item /var/log/secure
			\item /var/log/messages
			\item /var/log/boot.log  %correct
			\item /var/log/boot   
		\end{enumerate}
		\bigskip
		\bigskip	

		\item \textbf{Which of the following log file is used to store authentication logs?}
		\begin{enumerate}[label=(\alph*)]
			\item /var/log/secure     %correct
			\item /var/log/messages
			\item /var/log/boot.log  
			\item /var/log/boot   
		\end{enumerate}
		\bigskip
		\bigskip	

		\item \textbf{Which of the following log file is used to store yum logs?}
		\begin{enumerate}[label=(\alph*)]
			\item /var/log/secure    
			\item /var/log/messages
			\item /var/log/yum.log 
			\item /var/log/dnf.log     %correct
		\end{enumerate}
		\bigskip
		\bigskip	

		\item \textbf{Which of the following log file is used to store cronjobs related logs?}
		\begin{enumerate}[label=(\alph*)]
			\item /var/log/crontab
			\item /var/log/crony
			\item /var/log/cron   %correct 
			\item /var/log/cron.log
		\end{enumerate}
		\bigskip
		\bigskip	

		\item \textbf{Which of the following command is used to create a log with facility kern \& priority info?}
		\begin{enumerate}[label=(\alph*)]
			\item logger -p kern.info  log\_message  %correct
			\item logger kern.info  log\_message
			\item logger -s kern.info  log\_message
			\item logger -d kern.info  log\_message
		\end{enumerate}
		\bigskip
		\bigskip
		
	\end{enumerate}
\end{flushleft}

\newpage


%%%%%-----------------------
%%%%
%%%%
%%%%%----------------------------------------------------------------------------------------
%%%%%	CHAPTER 17
%%%%%----------------------------------------------------------------------------------------
\chapterimage{index18.png} % Table of contents heading image
\chapter{Boot process}
%%%%%-----------------------
\section{Booting in Linux}\index{Understanding vi/vim editor}
\input{content/chapter17/17.1.tex}
\subsection{What is boot process?}\index{Understanding vi/vim editor!About vi/vim editor}
\input{content/chapter17/17.1.1.tex}
\subsection{BIOS \& POST}\index{Understanding vi/vim editor!About vi/vim editor}
\input{content/chapter17/17.1.2.tex}
\subsection{MBR}\index{Understanding vi/vim editor!Practice}
\input{content/chapter17/17.1.3.tex}
\subsection{Bootloader}\index{Understanding vi/vim editor!Practice}
\input{content/chapter17/17.1.4.tex}
\subsection{Run levels/Targets}\index{Understanding vi/vim editor!Practice}
\input{content/chapter17/17.1.5.tex}
\subsection{Systemd}\index{Understanding vi/vim editor!Practice}
\input{content/chapter17/17.1.6.tex}
\subsection{Practice}\index{Understanding vi/vim editor!Practice}
\input{content/chapter17/17.1.7.tex}
%%%%%-----------------------
%%%%
%%%%%----------------------------------------------------------------------------------------
%%%%%	CHAPTER 17
%%%%%----------------------------------------------------------------------------------------
\chapterimage{index19.png} % Table of contents heading image
\chapter{Linux OS Installation}
%%%%%-----------------------
\section{Installation in detail}\index{Understanding vi/vim editor}
\input{content/chapter18/18.1.tex}
\subsection{Minimum system requirements for RHEL}\index{Understanding vi/vim editor!About vi/vim editor}
\input{content/chapter18/18.1.1.tex}
\subsection{Partition requirement}\index{Understanding vi/vim editor!About vi/vim editor}
\input{content/chapter18/18.1.2.tex}
\subsection{Creating a virtual machine}\index{Understanding vi/vim editor!Practice}
\input{content/chapter18/18.1.3.tex}
\subsection{RHEL OS using Anaconda}\index{Understanding vi/vim editor!Practice}
\input{content/chapter18/18.1.4.tex}
%%%%%-----------------------
%%%%
%%%%
%%%%%----------------------------------------------------------------------------------------
%%%%%	CHAPTER 19
%%%%%----------------------------------------------------------------------------------------
\chapterimage{index20.png} % Table of contents heading image
\chapter{Remote Access in Linux}
%%%%%-----------------------
\section{SSH, scp \& rsync}
\input{content/chapter19/19.1.tex}
\subsection{What is SSH?}
\input{content/chapter19/19.1.1.tex}
\subsection{What is a server \& client?}
\input{content/chapter19/19.1.8.tex}
\subsection{Setting up SSH server \& client on RHEL}
\input{content/chapter19/19.1.2.tex}
\subsection{Password based SSH connection}\index{Understanding vi/vim editor!Practice}
\input{content/chapter19/19.1.3.tex}
\subsection{Passwordless SSH connection}\index{Understanding vi/vim editor}
\input{content/chapter19/19.1.4.tex}
\subsection{What is scp command?}\index{Understanding vi/vim editor!Practice}
\input{content/chapter19/19.1.5.tex}
\subsection{What is rsync command?}\index{Understanding vi/vim editor!Practice}
\setlength{\columnsep}{3pt}
\begin{flushleft}

\bigskip
\begin{itemize}
	\item \textbf{rsync}: Securely copy files from one system to another. It differs from scp in that if two files or directories are similar between two systems, rsync only needs to copy the differences between the systems, while scp would need to copy everything.
	\bigskip
		
	\item \textbf{Copy file using rsync:}
	\begin{tcolorbox}[breakable,notitle,boxrule=0pt,colback=pink,colframe=pink]
		\color{black}

		Syntax: rsync [option] [user@]SRC\_HOST:]file1 [user@]DEST\_HOST:]file2

	\end{tcolorbox}
	
	\bigskip
	
	Eg: Copy file from remote SSH server to SSH client's /tmp folder.
	\begin{tcolorbox}[breakable,notitle,boxrule=-0pt,colback=black,colframe=black]
		\color{green}
		\fontdimen2\font=1em
		\# rsync jack@192.168.0.108:~/one.txt  /tmp
		\fontdimen2\font=4pt
	\end{tcolorbox}

	\bigskip
	
	Eg: Copy file from SSH client's /home/jim/data.txt to remote SSH server's /tmp folder.
	\begin{tcolorbox}[breakable,notitle,boxrule=-0pt,colback=black,colframe=black]
		\color{green}
		\fontdimen2\font=1em
		\# rsync  /home/jim/data.txt  jack@192.168.0.108:/tmp
		\fontdimen2\font=4pt
	\end{tcolorbox}

	\bigskip
	\bigskip
	\item \textbf{Copy folder using rsync:}
	\begin{tcolorbox}[breakable,notitle,boxrule=0pt,colback=pink,colframe=pink]
		\color{black}
		\fontdimen2\font=1em
		Syntax: rsync -r [user@]SRC\_HOST:]folder1 [user@]DEST\_HOST:]folder2
		\fontdimen2\font=4pt
	\end{tcolorbox}

	\bigskip

	Eg: Copy folder from remote SSH server to SSH client's /tmp folder.
	\begin{tcolorbox}[breakable,notitle,boxrule=-0pt,colback=black,colframe=black]
		\color{green}
		\fontdimen2\font=1em
		\# rsync -r jack@192.168.0.108:~/data  /tmp
		\fontdimen2\font=4pt
	\end{tcolorbox}
	
	\bigskip
	Eg: Copy folder /home/jim/data of SSH client to remote SSH server's /tmp folder.
	\begin{tcolorbox}[breakable,notitle,boxrule=-0pt,colback=black,colframe=black]
		\color{green}
		\fontdimen2\font=1em
		\# rsync -r /home/jim/data  jack@192.168.0.108:/tmp
		\fontdimen2\font=4pt
	\end{tcolorbox}
		
\end{itemize}

\end{flushleft}
\newpage



\subsection{Practice}\index{Understanding vi/vim editor!Practice}
\input{content/chapter19/19.1.6.tex}
%%%%%-----------------------



%%%%
%%%%
%%%%
%%%%
%%%%%----------------------------------------------------------------------------------------
%%%%%	PART
%%%%%----------------------------------------------------------------------------------------
%%%%
%%%%\part{System Admin Level II}
%%%%
%%%%%----------------------------------------------------------------------------------------
%%%%%	CHAPTER 3
%%%%%----------------------------------------------------------------------------------------
%%%%
%%%%\chapterimage{index4.pdf} % Chapter heading image
%%%%
%%%%\chapter{Presenting Information}
%%%%
%%%%\section{Table}\index{Table}
%%%%
%%%%\begin{table}[h]
%%%%\centering
%%%%\begin{tabular}{l l l}
%%%%\toprule
%%%%\textbf{Treatments} & \textbf{Response 1} & \textbf{Response 2}\\
%%%%\midrule
%%%%Treatment 1 & 0.0003262 & 0.562 \\
%%%%Treatment 2 & 0.0015681 & 0.910 \\
%%%%Treatment 3 & 0.0009271 & 0.296 \\
%%%%\bottomrule
%%%%\end{tabular}
%%%%\caption{Table caption}
%%%%\label{tab:example} % Unique label used for referencing the table in-text
%%%%%\addcontentsline{toc}{table}{Table \ref{tab:example}} % Uncomment to add the table to the table of contents
%%%%\end{table}
%%%%
%%%%Referencing Table \ref{tab:example} in-text automatically.
%%%%
%%%%%------------------------------------------------
%%%%
%%%%\section{Figure}\index{Figure}
%%%%
%%%%\begin{figure}[h]
%%%%\centering\includegraphics[scale=0.5]{placeholder.jpg}
%%%%\caption{Figure caption}
%%%%\label{fig:placeholder} % Unique label used for referencing the figure in-text
%%%%%\addcontentsline{toc}{figure}{Figure \ref{fig:placeholder}} % Uncomment to add the figure to the table of contents
%%%%\end{figure}
%%%%
%%%%Referencing Figure \ref{fig:placeholder} in-text automatically.

%----------------------------------------------------------------------------------------
%	BIBLIOGRAPHY
%----------------------------------------------------------------------------------------

%\chapter*{Bibliography}
%\addcontentsline{toc}{chapter}{\textcolor{ocre}{Bibliography}} % Add a Bibliography heading to the table of contents
%
%%------------------------------------------------
%
%\section*{Articles}
%\addcontentsline{toc}{section}{Articles}
%\printbibliography[heading=bibempty,type=article]
%
%%------------------------------------------------
%
%\section*{Books}
%\addcontentsline{toc}{section}{Books}
%\printbibliography[heading=bibempty,type=book]
%
%%----------------------------------------------------------------------------------------
%%	INDEX
%%----------------------------------------------------------------------------------------
%
%\cleardoublepage % Make sure the index starts on an odd (right side) page
%\phantomsection
%\setlength{\columnsep}{0.75cm} % Space between the 2 columns of the index
%\addcontentsline{toc}{chapter}{\textcolor{ocre}{Index}} % Add an Index heading to the table of contents
%\printindex % Output the index

%----------------------------------------------------------------------------------------

\end{document}


