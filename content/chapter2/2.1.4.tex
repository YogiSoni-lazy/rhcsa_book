\setlength{\columnsep}{5pt}

\begin{flushleft}
	\paragraph{}
	%\begin{tabulary}{1.0\textwidth}{C|C|C|p{10em}}
	
	\textbf{Useful command line-editing shortcuts}	
	\begin{tabulary}{1.0\textwidth}{|p{10em}|p{18em}|}
		\toprule
		\textbf{Shortcut} & \textbf{Description}\\
		\midrule
		\textbf{Ctrl + l} & Clear the terminal screen. \\
		\hline
		\textbf{Ctrl + a} & Jump to the beginning of the command line. \\
		\hline
		\textbf{Ctrl + e} & Jump to the beginning of the command line. \\
		\hline
		\textbf{Ctrl + u} & Clear from the cursor to the beginning of the command line. \\
		\hline
		\textbf{Ctrl + k} & Clear from the cursor to the end of the command line. \\
		\hline
		\textbf{Ctrl + r} & Search the history list of commands for a pattern. \\
		\hline
		\textbf{Ctrl + Shift + c} & Copy text from terminal. \\
		\hline
		\textbf{Ctrl + Shift + v} & Paste copied text on terminal. \\
		\hline
		\textbf{Ctrl + Shift + t} & Open new terminal. \\
		\hline
		\textbf{Alt + Tab} & Switch between applications. \\
		\bottomrule
	\end{tabulary}

	
	\label{tab:example} % Unique label used for referencing the table in-text
	%\addcontentsline{toc}{table}{Table \ref{tab:example}} % Uncomment to add the table to the table of contents
	
	
	
\end{flushleft}

\newpage