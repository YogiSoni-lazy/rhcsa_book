
\begin{flushleft}
	
	There are two modes using which you can navigate \& edit a file in vim:
	\begin{itemize}

		\item \textbf{Command mode}: 
		\begin{itemize}
			\item \textbf{Press Esc} to enter command mode.
			\item This is default mode.
			\item File cannot be edited under this mode.
			\item Below keystrokes will move the cursor:
			\begin{itemize}	
				\item \textbf{\$}: Move cursor to the end of the line.
				\item \textbf{17G}: Move cursor to line 17 (i.e type 17 and press shift + g).
				\item \textbf{G}: Move cursor to the last line (i.e press shift + g).
			\end{itemize}
			
		\end{itemize}

		\newline
		\bigskip
		
		\bigskip
		\bigskip
		\item \textbf{Insert mode}: 
		\begin{itemize}
			\item \textbf{Press i or a or o} to enter insert mode.
			\item This mode is used to edit the file.
			\item Below keystrokes will allow you to enter this mode:
			\begin{itemize}
				\item \textbf{i}: Insert text just \textbf{before} the current cursor position.
				\item \textbf{a}: Insert text just \textbf{after} the current cursor position.
				\item \textbf{o}: Insert text into a \textbf{new line} below current line.
				\item \textbf{I}: Insert text at the \textbf{beginning} of the current line.
				\item \textbf{A}: Insert text at the \textbf{end} of the current line.
				\item \textbf{O}: Insert text into a \textbf{new line} above current line.
			\end{itemize}
			
		\end{itemize}

					
	
		
		

	\end{itemize}
	
	
\end{flushleft}

\newpage
