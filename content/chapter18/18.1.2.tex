\setlength{\columnsep}{3pt}
\begin{flushleft}

\bigskip

\textbf{/}: This is the compulsory partition. Minimum size of this partition is \textbf{20 GB}. 
\newline
Below partition can be created separately to have more structured filesystem:
\begin{itemize}
	\item \textbf{/boot}: Minimum size for this partition is 512 MB.
	\item \textbf{swap}: Recommended swap partition size:
	\begin{tabulary}{1.0\textwidth}{|p{10em}|p{10em}|}
		\toprule
		\textbf{Amount of RAM installed in system} & \textbf{Recommended swap space}\\
		\midrule
		≤ 2GB & 2xRAM \\
		\hline
		2GB – 8GB & = RAM \\
		\hline
		8GB – 64GB & 4G to 0.5xRAM \\
		\hline
		>64GB & Minimum 4G, depends on type of application installed, number of user etc.\\
		\bottomrule
	\end{tabulary}
	
	\bigskip\bigskip
	\item \textbf{/var}: It is recommended to have \textbf{/var} as separate partition to have a separate space for logs, libraries \& temporary files. Minimum space required is \textbf{5GB}.
	\item \textbf{/home}: It is recommended to have \textbf{/home} as separate partition. Minimum recommended size of \textbf{8GB}.
	
\end{itemize}
 
 
 \begin{tcolorbox}[breakable,notitle,boxrule=-0pt,colback=yellow,colframe=yellow]
 	\color{black}
 	Note: It is recommended to install Linux OS using LVM rather than normal partitions.
 \end{tcolorbox}
 
 
 

\end{flushleft}
\newpage


