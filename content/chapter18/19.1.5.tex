\setlength{\columnsep}{3pt}
\begin{flushleft}

\bigskip
\begin{itemize}
	\item \textbf{scp}: Stands for \textbf{S}ecure \textbf{C}o\textbf{P}y. It is used to copy data securely from one Linux server to another.
	\newline
	scp command will ask for remote SSH server's user account password or key passphrase.
	\bigskip
		
	\item Copy file using scp:
	\begin{tcolorbox}[breakable,notitle,boxrule=0pt,colback=pink,colframe=pink]
		\color{black}

		Syntax: scp [user@]SRC\_HOST:]file1 [user@]DEST\_HOST:]file2

	\end{tcolorbox}
	
	\bigskip
	
	Eg: Copy file from remote SSH server to SSH client's /tmp folder.
	\begin{tcolorbox}[breakable,notitle,boxrule=-0pt,colback=black,colframe=black]
		\color{green}
		\fontdimen2\font=1em
		\# scp jack@192.168.0.108:~/one.txt  /tmp
		\fontdimen2\font=4pt
	\end{tcolorbox}

	\bigskip
	
	Eg: Copy file from SSH client's /home/jim/data.txt to remote SSH server's /tmp folder.
	\begin{tcolorbox}[breakable,notitle,boxrule=-0pt,colback=black,colframe=black]
		\color{green}
		\fontdimen2\font=1em
		\# scp  /home/jim/data.txt  jack@192.168.0.108:/tmp
		\fontdimen2\font=4pt
	\end{tcolorbox}

	\bigskip
	
	\item Copy folder using scp:
	\begin{tcolorbox}[breakable,notitle,boxrule=0pt,colback=pink,colframe=pink]
		\color{black}
		\fontdimen2\font=1em
		Syntax: scp -R [user@]SRC\_HOST:]folder1 [user@]DEST\_HOST:]folder2
		\fontdimen2\font=4pt
	\end{tcolorbox}

	\bigskip

	Eg: Copy folder from remote SSH server to SSH client's /tmp folder.
	\begin{tcolorbox}[breakable,notitle,boxrule=-0pt,colback=black,colframe=black]
		\color{green}
		\fontdimen2\font=1em
		\# scp -R jack@192.168.0.108:~/data  /tmp
		\fontdimen2\font=4pt
	\end{tcolorbox}
	
	\bigskip
	Eg: Copy folder /home/jim/data of SSH client to remote SSH server's /tmp folder.
	\begin{tcolorbox}[breakable,notitle,boxrule=-0pt,colback=black,colframe=black]
		\color{green}
		\fontdimen2\font=1em
		\# scp -R /home/jim/data  jack@192.168.0.108:/tmp
		\fontdimen2\font=4pt
	\end{tcolorbox}
		
\end{itemize}

\end{flushleft}
\newpage


